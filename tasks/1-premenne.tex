\subsection{Premenné}
\underline{\textbf{Premenná}} je taká krabička na odkladanie čísel alebo slov, ktoré si potrebujeme zapamätať na dokončenie činnosti. Premenné sa líšia svojím \underline{\textbf{dátovým typom}}. Premenná dostane svoj typ cez \underline{\textbf{priradenie}}, čiže vtedy keď prvýkrát do nej niečo uložíme. Typ hovorí o tom, čo sa vo vnútri premennej nachádza.

Základné typy premenných sú:
\begin{itemize}[noitemsep,topsep=0pt]
\item \underline{\textbf{Logická hodnota}} (\textit{bool}) - môže mať len dve hodnoty - pravda (\textit{True}) alebo nepravda (\textit{False})
\item \underline{\textbf{Celé číslo}} (\textit{int}) - ukladáme sem ľubovoľné kladné a záporné celé čísla (\textit{97})
\item \underline{\textbf{Desatinné číslo}} (\textit{float}) - Líšia sa od celých čísel spôsobom uloženia (\textit{3.14159})
\item \underline{\textbf{Reťazec}} (\textit{str}) - Označujeme ich úvodzovkami alebo apostrofmi a väčšinou predstavujú text napísaný na klávesnici alebo zobrazený na obrazovke \textit{``Učím sa programovať!''}).
\end{itemize}

\subsubsection*{1. Pozdrav}
Skladáš si stavebnicu robotického domáceho miláčika, ktorý je takmer dokonalý. Má telo, končatiny, hlavu a vie kráčať po stole. Keby však sa naučil aj hovoriť, to by bol potom poriadny spoločník. Každý dobrý rozhovor sa začína pozdravom. Napíš program, ktorý ťa pozdraví po napísaní mena na klávesnici. Doplň tiež, aby sa program s tebou aj rozlúčil.

\begin{tabular}{@{}p{0.15\linewidth}p{0.75\linewidth}}
\textbf{\small Vstup:} &
\vspace{-3em}
\begin{code}
Ako sa voláš?: @\fbox{\phantom{meno}}@
\end{code}
\end{tabular}

\vspace{-2em}
\begin{tabular}{@{}p{0.15\linewidth}p{0.75\linewidth}}
\textbf{\small Výstup:} &
\vspace{-3em}
\begin{code}
Ahoj @\fbox{\phantom{meno}}@
\end{code}
\end{tabular}
\vspace{-2em}

\subsubsection*{2. Básnik}
Rozkríklo sa, že píšeš pekné básničky na rôzne príležitosti. Prichádza ti čím ďalej viac prosieb od kamarátov a známych, či im nevytvoríš peknú rýmovačku. Vymýšľať kreatívne texty je niekedy veľká námaha, tak ti napadne, že stačí meniť len rým. Napíš program, ktorý za teba slovo vloží do predlohy básne.

\begin{tabular}{@{}p{0.15\linewidth}p{0.75\linewidth}}
\textbf{\small Vstup:} &
\vspace{-3em}
\begin{code}
Napíš slovo, ktoré sa rýmuje so slovom strach: @\fbox{\phantom{slovo}}@
\end{code}
\end{tabular}

\vspace{-2em}
\begin{tabular}{@{}p{0.15\linewidth}p{0.75\linewidth}}
\textbf{\small Výstup:} &
\vspace{-3em}
\begin{code}
Tu je báseň:
Z počítačov mával som vždy strach,
teraz som však šťastný ako @\fbox{\phantom{slovo}}@.
\end{code}
\end{tabular}
\vspace{-2em}

\subsubsection*{3. Pozvánka}
Budúci víkend organizuješ veľkolepú narodeninovú párty a rozposielaš na ňu pozvánky. Okrem mena hosťa potrebuješ meniť aj čas konania oslavy. Máš totiž skúsenosti, že nie všetci chodia načas. Každý pozvaný si má priniesť okrem darčeku aj jednu špeciálnu vec. Napíš program, ktorý doplní takúto pozvánku na mieru.

\begin{tabular}{@{}p{0.15\linewidth}p{0.75\linewidth}}
\textbf{\small Vstup:} &
\vspace{-3em}
\begin{code}
Meno kamaráta: @\fbox{\phantom{vstup}}@
Čas oslavy: @\fbox{\phantom{vstup}}@
Prinesie okrem darčeku: @\fbox{\phantom{vstup}}@
\end{code}
\end{tabular}

\vspace{-2em}
\begin{tabular}{@{}p{0.15\linewidth}p{0.75\linewidth}}
\textbf{\small Výstup:} &
\vspace{-3em}
\begin{code}
Ahoj @\fbox{\phantom{vstup}}@,
pozývam ťa na moju narodeninovú párty.
Bude sa konať 12.4. o @\fbox{\phantom{vstup}}@.
Nezabudni priniesť @\fbox{\phantom{vstup}}@ a pekný darček.
Teším sa na teba! :)
\end{code}
\end{tabular}
\vspace{-2em}

\subsubsection*{4. Teplota vo Farenheitoch}
Prišiel si na dovolenku do Spojených štátov amerických. Chystáš sa na krátky výlet von, ale nevieš ako sa máš obliecť. Na teplomeroch sú napísané len stupne Fahrenheita. Napíš program, ktorý ich premení na stupne Celzia s presnosťou na dve desatinné miesta.

\begin{tabular}{@{}p{0.15\linewidth}p{0.75\linewidth}}
\textbf{\small Vstup:} &
\vspace{-3em}
\begin{code}
Vonku je °F: @\fbox{\phantom{vstup}}@
\end{code}
\end{tabular}

\vspace{-2em}
\begin{tabular}{@{}p{0.15\linewidth}p{0.75\linewidth}}
\textbf{\small Výstup:} &
\vspace{-3em}
\begin{code}
Doma by bolo na teplomeri @\fbox{\phantom{vstup}}@°C.
\end{code}
\end{tabular}
\vspace{-2em}

\subsubsection*{5. Hlboká roklina}
Stojíš nad hlbokým údolím za zábradlím a uvažuješ ako odmerať jeho hĺbku. Vtom si spomenieš na svoje vedomosti z fyziky. Zoberieš si do ruky váľajúci sa kameň a pustíš ho priamo do rokliny. Zároveň stlačíš stopky, ktorými zmeriaš čas do dopadu v sekundách. Kameň padá nadol voľným pádom. Stopky zastavíš pri započutí rachotu z nárazu. Pri výpočte zanedbáme rýchlosť zvuku, ktorou sa rachot rozšíri až k nám.

\begin{tabular}{@{}p{0.15\linewidth}p{0.75\linewidth}}
\textbf{\small Vstup:} &
\vspace{-3em}
\begin{code}
Čas do dopadu kameňa: @\fbox{\phantom{vstup}}@
\end{code}
\end{tabular}

\vspace{-2em}
\begin{tabular}{@{}p{0.15\linewidth}p{0.75\linewidth}}
\textbf{\small Výstup:} &
\vspace{-3em}
\begin{code}
Hĺbka rokliny je @\fbox{\phantom{vstup}}@ metrov.
\end{code}
\end{tabular}
\vspace{-2em}

\subsubsection*{6. Vedro s vodou}
V rodinnom dome ste ekologicky uvedomelí, lebo zachytávate dažďovú vodu z odkvapu na polievanie záhrady. Minulú noc vám napršalo do nádrže veľa vody. Keď bude o pár dní suchšie mama ťa pošle poliať rastliny uzavretým vedrom valcového tvaru. To naplníš vždy až po okraj. Zaujíma ťa, aký objem naberieš na jedno naplnenie. Rozmery valcového vedra vieš odmerať pravítkom. Napíš program, ktorý zráta koľko sa zmestí vody do rôzne veľkých vedier.

\begin{tabular}{@{}p{0.15\linewidth}p{0.75\linewidth}}
\textbf{\small Vstup:} &
\vspace{-3em}
\begin{code}
Výška vedra (cm): @\fbox{\phantom{vstup}}@
Priemer dna (cm): @\fbox{\phantom{vstup}}@
\end{code}
\end{tabular}

\vspace{-2em}
\begin{tabular}{@{}p{0.15\linewidth}p{0.75\linewidth}}
\textbf{\small Výstup:} &
\vspace{-3em}
\begin{code}
Do vedra sa zmestí @\fbox{\phantom{vstup}}@ litrov vody.
\end{code}
\end{tabular}
\vspace{-2em}

\subsubsection*{7. Cesta autom}
Tešíš sa na očakávaný výlet autom po Európe. Pri plánovaní trasy chceš zistiť akou rýchlosťou musíte priemerne cestovať, aby ste od rána stihli navštíviť všetky miesta. Večer však musíte prísť včas do hotela, aby vás ubytovali. Napíš program, ktorý ti s tým pomôže.

\begin{tabular}{@{}p{0.15\linewidth}p{0.75\linewidth}}
\textbf{\small Vstup:} &
\vspace{-3em}
\begin{code}
Dĺžka cesty (km): @\fbox{\phantom{vstup}}@
Odchod z domu (hodina): @\fbox{\phantom{vstup}}@
Príchod do hotela (hodina): @\fbox{\phantom{vstup}}@
\end{code}
\end{tabular}

\vspace{-2em}
\begin{tabular}{@{}p{0.15\linewidth}p{0.75\linewidth}}
\textbf{\small Výstup:} &
\vspace{-3em}
\begin{code}
Auto pôjde priemernou rýchlosťou @\fbox{\phantom{vstup}}@ km/h.
\end{code}
\end{tabular}
\vspace{-2em}

\subsubsection*{8. Kúpalisko}
Začína sa horúca letná sezóna. Prevádzka kúpaliska musí pred otvorením napustiť bazény. Všetky bazény v areáli sú kvádrového tvaru, ktorých rozmery poznáme. Vedúceho kúpaliska zaujíma spotrebovaná voda pre bazén, keď bude napustený pod okraj. Voda nie je zadarmo, preto si chcú pripraviť dosť peňazí, aby za ňu zaplatili.

\begin{tabular}{@{}p{0.15\linewidth}p{0.75\linewidth}}
\textbf{\small Vstup:} &
\vspace{-3em}
\begin{code}
Dĺžka bazéna (m): @\fbox{\phantom{vstup}}@
Šírka bazéna (m): @\fbox{\phantom{vstup}}@
Hĺbka bazéna (m): @\fbox{\phantom{vstup}}@
Hĺbka hladiny pod okrajom (cm): @\fbox{\phantom{vstup}}@
Cena za m^3 vody v eurách: @\fbox{\phantom{vstup}}@
\end{code}
\end{tabular}

\vspace{-2em}
\begin{tabular}{@{}p{0.15\linewidth}p{0.75\linewidth}}
\textbf{\small Výstup:} &
\vspace{-3em}
\begin{code}
Na bazén sa minie @\fbox{\phantom{vstup}}@ litrov vody.
Voda bude stáť @\fbox{\phantom{vstup}}@ eur.
\end{code}
\end{tabular}
\vspace{-2em}

\subsubsection*{9. Maľovanie}
S rodičmi sa sťahuješ do nového bytu. Dali ti za úlohu kúpiť si farbu na vymaľovanie izby. Nástroj na rýchle počítanie množstva farby by sa hodil asi aj profesionálnym maliarom. Vytvor program na vypočítanie plochy stien a stropu bez okna a podlahy.

\begin{tabular}{@{}p{0.15\linewidth}p{0.75\linewidth}}
\textbf{\small Vstup:} &
\vspace{-3em}
\begin{code}
Rozmery miestnosti
Dĺžka (cm): @\fbox{\phantom{vstup}}@
Šírka (cm): @\fbox{\phantom{vstup}}@
Výška (cm): @\fbox{\phantom{vstup}}@
Rozmery okna
Šírka (cm): @\fbox{\phantom{vstup}}@
Výška (cm): @\fbox{\phantom{vstup}}@
Výdatnosť farby (m^2/kg): @\fbox{\phantom{vstup}}@
\end{code}
\end{tabular}

\vspace{-2em}
\begin{tabular}{@{}p{0.15\linewidth}p{0.75\linewidth}}
\textbf{\small Výstup:} &
\vspace{-3em}
\begin{code}
Maľovať budeš plochu @\fbox{\phantom{vstup}}@ m^2.
Kúp @\fbox{\phantom{vstup}}@ kg farby.
\end{code}
\end{tabular}
\vspace{-2em}

\subsubsection*{10. Chemikálie}
Chemici v laboratóriu bežne zmiešavajú roztoky, aby dosiahli správny pomer želanej látky. Roztoky sú opísané svojou hmotnosťou ($m$) a hmotnostným zlomkom rozpustenej látky v rozpúšťadle ($w$). Viaceré látky odlíšime dolným indexom ($m_1$).  Hmotnosť sa uvádza v gramoch a hmotnostný zlomok v percentách. Napíš program na opísanie vlastností výsledného roztoku. Na výpočet použi tieto rovnice:
\begin{align*}
m_3 &= m_1 + m_2 \\
m_3 \cdot w_3 &= m_1 \cdot w_1 +  m_2 \cdot w_2
\end{align*}

\begin{tabular}{@{}p{0.15\linewidth}p{0.75\linewidth}}
\textbf{\small Vstup:} &
\vspace{-3em}
\begin{code}
Hmotnosť roztoku č.1 (m1)? @\fbox{\phantom{vstup}}@
Hmotnostný zlomok roztoku č.1 (w1)? @\fbox{\phantom{vstup}}@
Hmotnosť roztoku č.2 (m2)? @\fbox{\phantom{vstup}}@
Hmotnostný zlomok roztoku č.2 (w2)? @\fbox{\phantom{vstup}}@
\end{code}
\end{tabular}

\vspace{-2em}
\begin{tabular}{@{}p{0.15\linewidth}p{0.75\linewidth}}
\textbf{\small Výstup:} &
\vspace{-3em}
\begin{code}
Výsledný roztok má hmotnosť @\fbox{\phantom{vstup}}@ g.
Hmotnostný zlomok rozpustenej látky je @\fbox{\phantom{vstup}}@ %.
\end{code}
\end{tabular}
\vspace{-2em}

\subsubsection*{11. Brzdenie}
V poslednej dobe sa objavuje na trati viac nebezpečných zrážok. Rušňovodiči ťa požiadali, aby si zistil ako rýchlo a ďaleko pred prekážkou dokáže vlak zastaviť. Vlaková súprava ide pred brzdením svojou stálou rýchlosťou v kilometroch za hodinu. Hmotnosť vlaku tvorí súčet hmotností lokomotívy a všetkých vagónov. Brzdy na kolesách majú spoločnú brzdnú silu uvedenú v Newtonoch na tonu. V programe využiješ nasledovné fyzikálne vzťahy:

\begin{itemize}
\itemsep0pt
\item Kinetická energia pohybujúceho sa vlaku (práca potrebná na zabrzdenie): \\ $ W = E_k = \frac{1}{2} \cdot m \cdot v^2 $
\item Brzdná dráha pri brzdnej sile $F_b$: $s = \frac{W}{F_b \cdot m} $
\item Čas na zastavenie vlaku pri rovnomernom spomalenom pohybe: $ t = \sqrt{\frac{2 \cdot s}{F / m}} $
\end{itemize}

\begin{tabular}{@{}p{0.15\linewidth}p{0.75\linewidth}}
\textbf{\small Vstup:} &
\vspace{-3em}
\begin{code}
Vlaková súprava
- Rýchlosť (km/h): @\fbox{\phantom{vstup}}@
- Hmotnosť lokomotívy (t): @\fbox{\phantom{vstup}}@
- Hmotnosť vagóna (t): @\fbox{\phantom{vstup}}@
- Počet vagónov: @\fbox{\phantom{vstup}}@
- Brzdná sila (N/t): @\fbox{\phantom{vstup}}@
\end{code}
\end{tabular}

\vspace{-2em}
\begin{tabular}{@{}p{0.15\linewidth}p{0.75\linewidth}}
\textbf{\small Výstup:} &
\vspace{-3em}
\begin{code}
Vlaková súprava má hmotnosť @\fbox{\phantom{vstup}}@ ton.
V rýchlosti @\fbox{\phantom{vstup}}@ km/h zabrzdí na vzdialenosť @\fbox{\phantom{vstup}}@ metrov.
Brzdenie bude trvať @\fbox{\phantom{vstup}}@ sekúnd.
\end{code}
\end{tabular}
\vspace{-2em}

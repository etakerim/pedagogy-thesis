\subsection{Predhovor}
\setlength\epigraphwidth{13cm}
\setlength\epigraphrule{0pt}
\epigraph{\small\itshape Umenie programátora je najmä o schopnosti meniť úrovne abstrakcie, z nízkej na vysokú úroveň. Niečo vidieť v malej a niečo vidieť vo veľkej mierke.}{--- \textup{Donald Knuth}}

Každý je raz na začiatku a stojí pred výzvou ako zvládnuť kostrbatú cestu, ktorá ho čaká.  Programovanie nie je v tomto ohľade výnimkou. Vedieť požiadať neživý predmet, počítač, aby  spravil to, čo od neho chceme, stojí nemalé úsilie. Často sa pri tom zasekneme na rôznych  chybách objavujúcich sa medzi zadaním a riešením problému. 

Zo začiatku si prejdeme cez množstvo vzájomných nedorozumení. Rečou stroja sú totiž mystické  postupnosti binárnych čísel. Ľudia našťastie vymysleli programovacíe jazyky, vďaka ktorým sa  dokážeme lepšie pochopiť. Naučiť sa plynulo rozprávať s týmto cudzincom si aj napriek tomu  vyžaduje veľa času a hlavne neustáleho prekonávania nových výziev.

Táto zbierka úloh si kladie za cieľ byť tvojim spoločníkom pralesom kódu od prvých pozdravov až k rozsiahlym esejám. Krása textov nebude spočívať v rýmoch básne, ale v presnej a usporiadanej logickej štruktúre. Naša činnosť bude podobná kuchárovi, keď objaví chutnú kombináciu prísad. Dá ich dokopy presným postupom a následne svoje majstrovstvo premení do receptu, aby si nové unikátne jedlo mohli uvariť a vychutnať všetci.

Pri ochutnávke sveta softvéru prejdeme od priamej postupnosti príkazov, cez rozhodnutia spracovania viacerých údajov v cykloch, až po využívanie súborov na ukladanie celých databáz. Našimi prísadami budú dáta a postupom algoritmy, alebo teda programy. Nezaháľajme teda a vydajme sa na púť. 

\subsection{Reťazce a zoznamy}
\underline{\textbf{Zoznam}} (tiež aj \underline{\textbf{Pole}}) je množina údajov zaznamenaných spolu pod jedným menom. Každý údaj poľa sa nazýva \underline{\textbf{prvok}} a poradie jeho pozície sa nazýva \underline{\textbf{index}}. \underline{\textbf{Reťazce}} sa správajú podobne ako zoznamy, ale ich prvkami sú jednotlivé \underline{\textbf{znaky}}.

\subsubsection*{1. Vymeň písmeno}
Niekto ti posiela správy s diakritikou, ale po ceste sa vždy prekrúti jedno písmeno. Texty obsahujú aj pekné básne, ktoré si chceš vytlačiť a pripnúť na nástenku. Pokazený znak však kazí celkový dojem z diela. Zameň zadané chybné písmeno v celom reťazci.

\begin{tabular}{@{}p{0.15\linewidth}p{0.75\linewidth}}
\textbf{\small Vstup:} &
\vspace{-3em}
\begin{code}
Správa: @\fbox{\phantom{dlhý text}}@
Za chybné písmeno: @\fbox{\phantom{a}}@
Vymeň: @\fbox{\phantom{b}}@
\end{code}
\end{tabular}

\vspace{-2em}
\begin{tabular}{@{}p{0.15\linewidth}p{0.75\linewidth}}
\textbf{\small Výstup:} &
\vspace{-3em}
\begin{code}
Opravené!
@\fbox{\phantom{dlhý text}}@
\end{code}
\end{tabular}
\vspace{-2em}


\subsubsection*{2. Cenzúra}
Prišla tvrdá cenzúra s nariadením, že nikto už nesmie vidieť žiadnu samohlásku. Nahraď každý priestupok vo vstupnom texte iným špeciálnym znakom.

\begin{tabular}{@{}p{0.15\linewidth}p{0.75\linewidth}}
\textbf{\small Vstup:} &
\vspace{-3em}
\begin{code}
Správa: @\fbox{Ja som tvoj kamarat}@
Samohlásku nahraď: @\fbox{*}@
\end{code}
\end{tabular}

\vspace{-2em}
\begin{tabular}{@{}p{0.15\linewidth}p{0.75\linewidth}}
\textbf{\small Výstup:} &
\vspace{-3em}
\begin{code}
Cenzurované: @\fbox{J* s*m tv*j k*m*r*t}@
\end{code}
\end{tabular}
\vspace{-2em}


\subsubsection*{3. Počítanie slov}
Do redakcie miestnych novín chodia dennodenne články, vtipy, poviedky a príbehy zo života od verných čitateľov. Aby mohli byť uverejnené potrebujú sa zmestiť do vyhradeného priestoru. Vypíš počet znakov, slov, viet a normostrán (=\emph{1800 znakov}), aby sa príhody rýchlejšie rozšírili medzi ľudí.

\begin{tabular}{@{}p{0.15\linewidth}p{0.75\linewidth}}
\textbf{\small Vstup:} &
\vspace{-3em}
\begin{code}
Článok: @\fbox{\phantom{Dlhý text článku s veľa slovami}}@
\end{code}
\end{tabular}

\vspace{-2em}
\begin{tabular}{@{}p{0.15\linewidth}p{0.75\linewidth}}
\textbf{\small Výstup:} &
\vspace{-3em}
\begin{code}
Znaky: @\fbox{\phantom{123}}@
Slová: @\fbox{\phantom{123}}@
Vety: @\fbox{\phantom{123}}@
Normostrany: @\fbox{\phantom{123}}@
\end{code}
\end{tabular}
\vspace{-2em}


\subsubsection*{4. Najdlhšie slovo}
Debatný spolok usporiadal súťaž o nájdenie najdlhšieho slova, ktoré sa kedy vyskytlo v historických prejavoch. Zaujali ťa odmeny, ale nechce sa ti prehrabávať knižnicou starých záznamníkov. Prácu si preto uľahčíš. Nájdi najdlhšie slovo v ľubovoľnom reťazci.

\begin{tabular}{@{}p{0.15\linewidth}p{0.75\linewidth}}
\textbf{\small Vstup:} &
\vspace{-3em}
\begin{code}
Rečnícky prejav: @\fbox{\phantom{Dlhý text článku s veľa slovami}}@
\end{code}
\end{tabular}

\vspace{-2em}
\begin{tabular}{@{}p{0.15\linewidth}p{0.75\linewidth}}
\textbf{\small Výstup:} &
\vspace{-3em}
\begin{code}
Najdlhšie slovo v ňom: @\fbox{\phantom{slovo}}@
\end{code}
\end{tabular}
\vspace{-2em}

\subsubsection*{5. Výskyt písmen}
Dlho do noci čítaš časopisy o umelej inteligencii a fascinuje ťa jej schopnosť rozprávať sa s človekom. Na vytvorenie viet na danú tému potrebuje mať prehľad o percentuálnom výskyte hlások v texte. Spočítaj a vypíš zoznam početnosti písmen v reťazci.

\begin{tabular}{@{}p{0.15\linewidth}p{0.75\linewidth}}
\textbf{\small Vstup:} &
\vspace{-3em}
\begin{code}
Článok: @\fbox{\phantom{Dlhý text článku s veľa slovami}}@
\end{code}
\end{tabular}

\vspace{-2em}
\begin{tabular}{@{}p{0.15\linewidth}p{0.75\linewidth}}
\textbf{\small Výstup:} &
\vspace{-3em}
\begin{code}
A: @\fbox{23.2}@ %
B: @\fbox{11.5}@ %
C: @\fbox{8.9}@ %
...
Z: @\fbox{0.3}@ %
\end{code}
\end{tabular}
\vspace{-2em}


\subsubsection*{6. Histogram}
Počas predošlého pokusu s početnosťou písmen si všimneš, že každé ďalšie písmeno v zozname sa objavuje oveľa menej než očakávaš. Vykresli hviezdičky namiesto počtu percent. Over si tak svoje pozorovanie graficky.

\begin{tabular}{@{}p{0.15\linewidth}p{0.75\linewidth}}
\textbf{\small Vstup:} &
\vspace{-3em}
\begin{code}
Článok: @\fbox{\phantom{Dlhý text článku s veľa slovami}}@
\end{code}
\end{tabular}

\vspace{-2em}
\begin{tabular}{@{}p{0.15\linewidth}p{0.75\linewidth}}
\textbf{\small Výstup:} &
\vspace{-3em}
\begin{code}
A: @\fbox{****}@
E: @\fbox{*******}@
I: @\fbox{****}@
...
X: @\fbox{*}@
\end{code}
\end{tabular}
\vspace{-2em}


\subsubsection*{7. Nákupný košík}
Na veľkých nákupoch sa často zíde prehľadný zoznam s tým, čo doma treba. Pýtaj si položky s ich cenami až kým sa nerozhodneš, že máš spísané všetko. Zobraz prehľadnú orámovanú tabuľku s údajmi, podobne ako na pokladničnom bločku. To sú názov tovaru, DPH tovaru, cena tovaru s DPH a cena spolu za nákup.

\begin{tabular}{@{}p{0.15\linewidth}p{0.75\linewidth}}
\textbf{\small Vstup:} &
\vspace{-3em}
\begin{code}
Čo kúpiť?: @\fbox{\phantom{vstup}}@
Cena @\fbox{\phantom{vstup}}@?: @\fbox{\phantom{vstup}}@
\end{code}
\end{tabular}

\vspace{-2em}
\begin{tabular}{@{}p{0.15\linewidth}p{0.75\linewidth}}
\textbf{\small Výstup:} &
\vspace{-3em}
\begin{code}
+----------+--------+--------------+
| Tovar    |  DPH   |  Cena s DPH  |
+----------+--------+--------------+
| Chlieb   |  0,20  |      0,98    |
+----------+--------+--------------+
|    ...   |  ...   |     ...      |
+----------+--------+--------------+
| CELKOM   |  0,20  |      0,98    |
+----------+--------+--------------+
\end{code}
\end{tabular}
\vspace{-2em}

\subsubsection*{8. Akronym}
SMS-ky rapídne zdraželi. Napadlo ti, že bude lepšie posielať slovné spojenia ako skratky. Zo zadaných slov vytvor akronym, ktorý vznikne ponechaním len začiatočných písmen každého slova.

\begin{tabular}{@{}p{0.15\linewidth}p{0.75\linewidth}}
\textbf{\small Vstup:} &
\vspace{-3em}
\begin{code}
Slovné spojenie: @\fbox{Slovenské národné divadlo}@
\end{code}
\end{tabular}

\vspace{-2em}
\begin{tabular}{@{}p{0.15\linewidth}p{0.75\linewidth}}
\textbf{\small Výstup:} &
\vspace{-3em}
\begin{code}
Skratka: @\fbox{SND}@
\end{code}
\end{tabular}
\vspace{-2em}


\subsubsection*{9. Veľa opakovania}
Roboti rozvážajú pizzu po meste. Popri tom si zapisujú zmenu smeru pre postupné vylepšovanie trás k častým zákazníkom. Keďže sa firme darí, nachodili roboti toho už riadne veľa. Všetky záznamy o ich cestách sa im ani nezmestia do pamäti. Všimneš si, že si značia každý jeden krok, čiže sa často opakujú. Nahraď postupnosť za sebou idúceho písmena, písmenom a počtom jeho výskytov.

\begin{tabular}{@{}p{0.15\linewidth}p{0.75\linewidth}}
\textbf{\small Vstup:} &
\vspace{-3em}
\begin{code}
Cesta robota: @\fbox{NNNNNNSSSSSSSSSSSWWWWNNN}@
\end{code}
\end{tabular}

\vspace{-2em}
\begin{tabular}{@{}p{0.15\linewidth}p{0.75\linewidth}}
\textbf{\small Výstup:} &
\vspace{-3em}
\begin{code}
Skomprimované: @\fbox{6N11S4W3N}@
\end{code}
\end{tabular}
\vspace{-2em}

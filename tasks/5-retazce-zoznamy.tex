\subsection{Reťazce a zoznamy}
\textbf{Zoznam} (tiež aj Pole) je množina údajov zaznamenaných spolu pod jedným menom. Každý údaj poľa sa nazýva *prvok* a poradie jeho pozície sa nazýva \textit{index}. \textbf{Reťazce} sa správajú podobne ako zoznamy, ale ich prvkami sú jednotlivé znaky.


\subsubsection*{1. Vymeň písmeno}
Niekto ti posiela správy s diakritikou, ale po ceste sa vždy prekrúti jedno písmeno. Texty obsahujú aj pekné básne, ktoré si chceš vytlačiť a pripnúť na nástenku. Pokazený znak však kazí celkový dojem z diela. Zameň zadané chybné písmeno v celom reťazci.

\begin{code}
Správa: ________
Za chybné písmeno: ____
Vymeň: ____

Opravené!
__________
\end{code}


\subsubsection*{2. Cenzúra}
Prišla tvrdá cenzúra s nariadením, že nikto už nesmie vidieť žiadnu samohlásku. Nahraď každý prečin vo vstupnom texte ľubovoľným iným špeciálnym znakom.

\begin{code}
Správa: Ja som tvoj kamarat
Samohlásku nahraď: *

Cenzurované: J* s*m tv*j k*m*r*t
\end{code}


\subsubsection*{3. Počítanie slov}
Do redakcie miestnych novín chodia denno denne články, vtipy, poviedky a príbehy zo života od verných čitateľov. Aby mohli byť uverejnené potrebujú sa zmestiť do vyhradeného priestoru. Vypíš počet znakov, slov, viet a normostrán (\textbf{1800 znakov}) pre rýchlejšie spracovanie textov.

\begin{code}
Článok: _________

Znaky: ___
Slová: ___
Vety: ___
Normostrany: ____
\end{code}


\subsubsection*{4. Najdlhšie slovo}
Hra staršia ako ľudstvo samo. Debatný spolok usporiadal súťaž o nájdenie najdlhšieho slova, ktoré sa kedy vyskytlo v historických prejavoch. Zaujali ťa odmeny, ale nechce sa ti prehrabávať knižnicou starých záznamníkov a preto si prácu uľahčíš. Nájdi najdlhšie slovo v reťazci.

\begin{code}
Rečnícky prejav: ________

Najdlhšie slovo v ňom: _____
\end{code}

\subsubsection*{5. Výskyt písmen}
Dlho do noci čítaš časopisy o umelej inteligencii a fascinuje ťa jej schopnosť rozprávať sa s človekom. Na vytvorenie viet na danú tému potrebuje mať prehľad o percentuálnom výskyte hlások v texte. Spočítaj a vypíš zoznam frekvencie písmen v reťazci.

\begin{code}
Článok: _______

A: 23.2 %
B: 11.5 %
C: 8.9 %
...
Z: 0.3 %
\end{code}


\subsubsection*{6. Histogram}
Pri svojom predchádzajúceho pokuse s početnosťou písmen si všimneš, že každé ďaľšie písmeno v zozname sa oveľa menej objavuje ako očakávaš. Vykresli hviezdičky namiesto počtu percent a over si tak svoje pozorovanie graficky.

\begin{code}
Článok: _______

A: ****
E: *******
I: ****
...
X: *
\end{code}


\subsubsection*{7. Nákupný košík}
Pri veľkých nákupoch sa často zíde prehľadný zoznam s tým, čo doma treba. Pýtaj si položky s ich cenami až kým sa nerozhodneš, že máš spísané všetko. Zobraz prehľadnú orámovanú tabuľku s údajmi podobne ako na pokladničom bločku (názov tovaru, DPH tovaru, cena tovaru s DPH, celková suma na zaplatenie).

\begin{code}
Čo kúpiť?: ______
Cena ______?: _______
....

+----------+--------+--------------+
| Tovar    |  DPH   |  Cena s DPH  |
+----------+--------+--------------+
| Chlieb   |  0,20e |      0,98e   |
+----------+--------+--------------+
|    ...   |  ...   |     ...      |
+----------+--------+--------------+
| CELKOM   |  0,20e |      0,98e   |
+----------+--------+--------------+
\end{code}

\subsubsection*{8. Akronym}
SMS-ky rapídne zdraželi a napadlo ti, že bude lepšie posielať slovné spojenia ako skratky. Zo zadaných slov vytvor akronym. Vezmi začiatočné písmená každého slova a vytvor skratku, ktorá bude pozostávať len z týchto písmen.

\begin{code}
Slovné spojenie: Slovenské národné divadlo
Skratka: SND
\end{code}


\subsubsection*{9. Veľa opakovania}
Roboti rozvážajúci pizzu po meste si zaznamenávajú zmenu smeru pre postupné vylepšovanie trás na lokality k častým zákazníkom. Keďže sa firme darí, prešli roboti už toľko, že sa im všetky záznamy o ich cestách nezmestia do pamäti. Všimneš si, že si značia každý krok a to vedie k častému opakovaniu. Nahraď postupnosť za sebou idúceho písmena, počtom výskytu a písmenom (\textit{Run-length encoding})

\begin{code}
Cesta robota: NNNNNNSSSSSSSSSSSWWWWNNN
Skomprimované: 6N11S4W3N
\end{code} 

\subsection{Cykly}
Obrovský potenciál počítačov tkvie v bezchybnom neúnavnom vykonávaní presne zadaných inštrukcií. \underline{\textbf{Cykly}} umožňujú opakovať rovnaký postup ľubovoľný počet krát a tým efektívne odstraňovať rutinnú prácu.


\subsubsection*{1. 100-krát napíš}
Za vyrušovanie na hodinách sa stalo populárnym trestom ručné prepisovanie mravoučnej vety stokrát. Stalo sa to tak neznesiteľné, že si zhotovil robota na pomoc záškodníkom. Chýbajú mu len príkazy, čo má vlastne robiť.

\begin{tabular}{@{}p{0.15\linewidth}p{0.75\linewidth}}
\textbf{\small Vstup:} &
\vspace{-3em}
\begin{code}
Musím napísať: @\fbox{\phantom{vstup}}@
Toľkoto krát: @\fbox{\phantom{vstup}}@
\end{code}
\end{tabular}

\vspace{-2em}
\begin{tabular}{@{}p{0.15\linewidth}p{0.75\linewidth}}
\textbf{\small Výstup:} &
\vspace{-3em}
\begin{code}
@\fbox{\phantom{vstup}}@
@\fbox{\phantom{vstup}}@
@\fbox{\phantom{vstup}}@
...
\end{code}
\end{tabular}
\vspace{-2em}


\subsubsection*{2. Hodnotenie}
Filmoví a gastonomickí kritici zavŕšia namáhavý deň udelením číselného skóre k ich recenziam. Pre lepší efekt v časopise potrebujú nakresliť hviezdničky namiesto čísla. Pomôž im programom.

\begin{tabular}{@{}p{0.15\linewidth}p{0.75\linewidth}}
\textbf{\small Vstup:} &
\vspace{-3em}
\begin{code}
Skóre: @\fbox{5}@
\end{code}
\end{tabular}

\vspace{-2em}
\begin{tabular}{@{}p{0.15\linewidth}p{0.75\linewidth}}
\textbf{\small Výstup:} &
\vspace{-3em}
\begin{code}
@\fbox{\textit{*****}}@
\end{code}
\end{tabular}
\vspace{-2em}


\subsubsection*{3. Pyramída}
Hviezdičky zoskup do tvaru pyramídy zadanej výšky.

\begin{tabular}{@{}p{0.15\linewidth}p{0.75\linewidth}}
\textbf{\small Vstup:} &
\vspace{-3em}
\begin{code}
Výška pyramídy: @\fbox{4}@
\end{code}
\end{tabular}

\vspace{-2em}
\begin{tabular}{@{}p{0.15\linewidth}p{0.75\linewidth}}
\textbf{\small Výstup:} &
\vspace{-3em}
\begin{code}
   *
  ***
 *****
*******
\end{code}
\end{tabular}
\vspace{-2em}

\subsubsection*{4. Smaragd}
Na pyramídu pripoj zo spodu ďaľšiu obrátene, aby vznikol smaragd z hviezdičiek.

\begin{tabular}{@{}p{0.15\linewidth}p{0.75\linewidth}}
\textbf{\small Vstup:} &
\vspace{-3em}
\begin{code}
Veľkosť smaragdu: @\fbox{5}@
\end{code}
\end{tabular}

\vspace{-2em}
\begin{tabular}{@{}p{0.15\linewidth}p{0.75\linewidth}}
\textbf{\small Výstup:} &
\vspace{-3em}
\begin{code}
  *
 ***
*****
 ***
  *
\end{code}
\end{tabular}
\vspace{-2em}

\subsubsection*{5. Duté vnútro}
Nakresli duté pyramídu a smaragd podľa prechádzajúcich úloh.

\begin{tabular}{@{}p{0.15\linewidth}p{0.75\linewidth}}
\textbf{\small Vstup:} &
\vspace{-3em}
\begin{code}
Výška pyramídy: @\fbox{4}@
\end{code}
\end{tabular}

\vspace{-2em}
\begin{tabular}{@{}p{0.15\linewidth}p{0.75\linewidth}}
\textbf{\small Výstup:} &
\vspace{-3em}
\begin{code}
    *
   * *
  *   *
 *******
\end{code}
\end{tabular}
\vspace{-2em}


\subsubsection*{6. Mriežka slov}
Načítaj veľkosť tabuľky a slovo, ktoré sa v nej bude na každom riadku v stĺpci opakovať.

\begin{tabular}{@{}p{0.15\linewidth}p{0.75\linewidth}}
\textbf{\small Vstup:} &
\vspace{-3em}
\begin{code}
Počet riakov a stĺpcov: @\fbox{4}@
Opakovať slovo: @\fbox{ano}@
\end{code}
\end{tabular}

\vspace{-2em}
\begin{tabular}{@{}p{0.15\linewidth}p{0.75\linewidth}}
\textbf{\small Výstup:} &
\vspace{-3em}
\begin{code}
ano ano ano ano
ano ano ano ano
ano ano ano ano
ano ano ano ano
\end{code}
\end{tabular}
\vspace{-2em}


\subsubsection*{7. Rám}
Prvý a posledný riadok a stĺpec bude tvoriť rám pre mriežku slov.

\begin{tabular}{@{}p{0.15\linewidth}p{0.75\linewidth}}
\textbf{\small Vstup:} &
\vspace{-3em}
\begin{code}
Počet riakov a stĺpcov: @\fbox{4}@
Opakovať slovo: @\fbox{ano}@
\end{code}
\end{tabular}

\vspace{-2em}
\begin{tabular}{@{}p{0.15\linewidth}p{0.75\linewidth}}
\textbf{\small Výstup:} &
\vspace{-3em}
\begin{code}
### ### ### ###
### ano ano ###
### ano ano ###
### ### ### ###
\end{code}
\end{tabular}
\vspace{-2em}


\subsubsection*{8. Malá násobilka}
K výbave každého žiaka základnej školy patrí tabuľky malej násobilky. Vytvor takúto tabuľku obsahujúcu každý násobok od 1x1 po 10x10, aby si pomohol všetkým malým matematikom.


\vspace{-2em}
\begin{tabular}{@{}p{0.15\linewidth}p{0.75\linewidth}}
\textbf{\small Výstup:} &
\vspace{-3em}
\begin{code}
   1   2   3   4   5   6   7   8   9  10
   2   4   6   8  10  12  14  16  18  20
   3   6   9  12  15  18  21  24  27  30
   4   8  12  16  20  24  28  32  36  40
   5  10  15  20  25  30  35  40  45  50
   6  12  18  24  30  36  42  48  54  60
   7  14  21  28  35  42  49  56  63  70
   8  16  24  32  40  48  56  64  72  80
   9  18  27  36  45  54  63  72  81  90
  10  20  30  40  50  60  70  80  90 100
\end{code}
\end{tabular}
\vspace{-2em}


\subsubsection*{9. Sporenie}
Na letnej brigáde si zarobil peniaze, ktoré chceš usporiť. Porovnáš ponuky bánk a hľadáš najvýhodnejší plán. Vytvor si sporiacu kalkulačku, ktorá na základe nemenného počiatčného vkladu, ročnej úrokovej sadzby, typu úročenia a žiadanej konečnej sumy, vypíše vývoj tvojich finančných prostriedkov do budúcnosti.

\begin{tabular}{@{}p{0.15\linewidth}p{0.75\linewidth}}
\textbf{\small Vstup:} &
\vspace{-3em}
\begin{code}
Počiatočný vklad v eurách: @\fbox{\phantom{vstup}}@
Úroková sadzba p.a. v %: @\fbox{\phantom{vstup}}@
Typ úročenia (jednoduché / zložené): @\fbox{\phantom{vstup}}@
Žiadaná suma v eurách: @\fbox{\phantom{vstup}}@
\end{code}
\end{tabular}

\vspace{-2em}
\begin{tabular}{@{}p{0.15\linewidth}p{0.75\linewidth}}
\textbf{\small Výstup:} &
\vspace{-3em}
\begin{code}
Rok      Suma						Úrok
  1.		@\fbox{\phantom{vstup}}@ Eur	@\fbox{\phantom{vstup}}@ Eur
  2.		@\fbox{\phantom{vstup}}@ Eur	@\fbox{\phantom{vstup}}@ Eur
\end{code}
\end{tabular}
\vspace{-2em}

\subsection{Cykly}
Obrovský potenciál počítačov tkvie v bezchybnom neúnavnom vykonávaní presne zadaných inštrukcií. \underline{\textbf{Cykly}} umožňujú opakovať rovnaký postup ľubovoľný počet krát a tým efektívne odstraňovať rutinnú prácu.


\subsubsection*{1. 100-krát napíš}
Za prehrešky proti školskému poriadku sa stalo populárnym trestom ručné prepisovanie mravoučnej vety stokrát. Stalo sa to tak neznesiteľné, že si zhotovil robota na pomoc záškodníkom. Chýbajú mu len príkazy, čo má vlastne robiť.

\begin{tabular}{@{}p{0.15\linewidth}p{0.75\linewidth}}
\textbf{\small Vstup:} &
\vspace{-3em}
\begin{code}
Musím napísať: @\fbox{\phantom{vstup}}@
Toľkoto krát: @\fbox{\phantom{vstup}}@
\end{code}
\end{tabular}

\vspace{-2em}
\begin{tabular}{@{}p{0.15\linewidth}p{0.75\linewidth}}
\textbf{\small Výstup:} &
\vspace{-3em}
\begin{code}
@\fbox{\phantom{vstup}}@
@\fbox{\phantom{vstup}}@
@\fbox{\phantom{vstup}}@
...
\end{code}
\end{tabular}
\vspace{-2em}


\subsubsection*{2. Hodnotenie}
Filmoví a gastonomickí kritici zavŕšia namáhavý deň udelením číselného skóre k ich recenziam. Pre lepší efekt v časopise potrebujú nakresliť hviezdničky namiesto čísla. Pomôž im programom.

\begin{tabular}{@{}p{0.15\linewidth}p{0.75\linewidth}}
\textbf{\small Vstup:} &
\vspace{-3em}
\begin{code}
Skóre: @\fbox{5}@
\end{code}
\end{tabular}

\vspace{-2em}
\begin{tabular}{@{}p{0.15\linewidth}p{0.75\linewidth}}
\textbf{\small Výstup:} &
\vspace{-3em}
\begin{code}
@\fbox{\textit{*****}}@
\end{code}
\end{tabular}
\vspace{-2em}


\subsubsection*{3. Pyramída}
Mayská civilizácia sa mohla pýšiť v čase svojho najväčšieho rozmachu všelijakými na tú dobu pokrokovými vymoženosťami. Doteraz sa ospevuje ich písmo, sofistikovaný kalendár a znalosti z astronómie. V mestách stavali mohutné chrámové pyramídy na náboženské obrady. Preniesol si sa späť v čase a ocitol si sa pri plánovaní pyramídy. Stavitelia chcú nakresliť jej plány, aby vedeli ako majú poskladať kamenné bloky. Napíš program, ktorý vypíše hviezdičky do tvaru pyramídy podľa jej výšky.

\begin{tabular}{@{}p{0.15\linewidth}p{0.75\linewidth}}
\textbf{\small Vstup:} &
\vspace{-3em}
\begin{code}
Výška pyramídy: @\fbox{4}@
\end{code}
\end{tabular}

\vspace{-2em}
\begin{tabular}{@{}p{0.15\linewidth}p{0.75\linewidth}}
\textbf{\small Výstup:} &
\vspace{-3em}
\begin{code}
   *
  ***
 *****
*******
\end{code}
\end{tabular}
\vspace{-2em}


\subsubsection*{4. Smaragd}
Nie všetko, čo sa blyští je zlato. Drahokamy ako rýzdy zelený smaragd však ulahodia oku podobne. Hruda horniny sa najprv musí vybrúsiť napríklad do amuletu, ktorý sa môže stať parádou náhrdelníku. Prešibaný zlatník nakupuje pre zákazníkov smaragdové amulety v tvare osemstenu. Ten z boku vyzerá takmer ako kosoštvorec. Zlatník ho chce porovnávať s ideálnym tvarom, aby mohol dohodnúť nižšiu cenu, keď ho bude chcieť dodávateľ podviesť. Napíš program na vykreslenie ,,smaragdu'' z hviezdičiek podľa zadanej veľkosti.

\begin{tabular}{@{}p{0.15\linewidth}p{0.75\linewidth}}
\textbf{\small Vstup:} &
\vspace{-3em}
\begin{code}
Veľkosť smaragdu: @\fbox{5}@
\end{code}
\end{tabular}

\vspace{-2em}
\begin{tabular}{@{}p{0.15\linewidth}p{0.75\linewidth}}
\textbf{\small Výstup:} &
\vspace{-3em}
\begin{code}
  *
 ***
*****
 ***
  *
\end{code}
\end{tabular}
\vspace{-2em}

\subsubsection*{5. Duté vnútro}
Staviteľov pyramíd začalo zaujímať zariaďovanie ich vnútra. Do posvätného chrámu sa predsa musia zmesiť všetky bohatstvá, ktorými si budú uctievať božstvá. Program tentokrát vykreslí hviezdičkovú pyramídu bez výplne.

\begin{tabular}{@{}p{0.15\linewidth}p{0.75\linewidth}}
\textbf{\small Vstup:} &
\vspace{-3em}
\begin{code}
Výška pyramídy: @\fbox{4}@
\end{code}
\end{tabular}

\vspace{-2em}
\begin{tabular}{@{}p{0.15\linewidth}p{0.75\linewidth}}
\textbf{\small Výstup:} &
\vspace{-3em}
\begin{code}
   *
  * *
 *   *
*******
\end{code}
\end{tabular}
\vspace{-2em}


\subsubsection*{6. Mriežka slov}
Tapety na stenu sa objavujú v najrozmanitejších podobách od hypnotických špirál cez kvetinové lúky až po hotové umelecké diela. Ešte nikoho nenapadlo si v obývačke natapetovať nekonečný zástup slov. Načítaj v programe šírku tapety a slovo, ktoré sa bude na každom riadku a v stĺpci na nej opakovať.

\begin{tabular}{@{}p{0.15\linewidth}p{0.75\linewidth}}
\textbf{\small Vstup:} &
\vspace{-3em}
\begin{code}
Počet riakov a stĺpcov: @\fbox{4}@
Opakovať slovo: @\fbox{ano}@
\end{code}
\end{tabular}

\vspace{-2em}
\begin{tabular}{@{}p{0.15\linewidth}p{0.75\linewidth}}
\textbf{\small Výstup:} &
\vspace{-3em}
\begin{code}
ano ano ano ano
ano ano ano ano
ano ano ano ano
ano ano ano ano
\end{code}
\end{tabular}
\vspace{-2em}


\subsubsection*{7. Rám}
Moderné umenie má svojich bezbrehých obdivovateľov aj zásadových neznalcov. Krásny obraz môžu tvoriť hoc opakujúce sa slová. Na zosilnenie dojmu by mali by byť pekne zarámované. Na prvý a posledný riadok a stĺpec doplní program symboly ,,\#''. Tie poskytnú rám pre zo mriežku slov.

\begin{tabular}{@{}p{0.15\linewidth}p{0.75\linewidth}}
\textbf{\small Vstup:} &
\vspace{-3em}
\begin{code}
Počet riakov a stĺpcov: @\fbox{4}@
Opakovať slovo: @\fbox{ano}@
\end{code}
\end{tabular}

\vspace{-2em}
\begin{tabular}{@{}p{0.15\linewidth}p{0.75\linewidth}}
\textbf{\small Výstup:} &
\vspace{-3em}
\begin{code}
### ### ### ###
### ano ano ###
### ano ano ###
### ### ### ###
\end{code}
\end{tabular}
\vspace{-2em}


\subsubsection*{8. Malá násobilka}
K výbave každého žiaka základnej školy patrí tabuľky malej násobilky. Vytvor takúto tabuľku obsahujúcu každý násobok od 1x1 po 10x10, aby si pomohol všetkým malým počtárom.

\begin{tabular}{@{}p{0.15\linewidth}p{0.75\linewidth}}
\textbf{\small Výstup:} &
\vspace{-3em}
\begin{code}
  1   2   3   4   5   6   7   8   9  10
  2   4   6   8  10  12  14  16  18  20
  3   6   9  12  15  18  21  24  27  30
  4   8  12  16  20  24  28  32  36  40
  5  10  15  20  25  30  35  40  45  50
  6  12  18  24  30  36  42  48  54  60
  7  14  21  28  35  42  49  56  63  70
  8  16  24  32  40  48  56  64  72  80
  9  18  27  36  45  54  63  72  81  90
 10  20  30  40  50  60  70  80  90 100
\end{code}
\end{tabular}
\vspace{-2em}


\subsubsection*{9. Sporenie}
Na letnej brigáde si zarobil peniaze, ktoré si chceš usporiť. Porovnáš ponuky bánk a hľadáš najvýhodnejší plán. Vytvor si sporiacu kalkulačku, ktorá vypíše vývoj tvojich finančných prostriedkov do budúcnosti. Bude vychádzať z tvojho počiatočného vkladu, ročnej úrokovej sadzby, typu úročenia a penazí, ktoré chceš mať na konci.

\begin{tabular}{@{}p{0.15\linewidth}p{0.75\linewidth}}
\textbf{\small Vstup:} &
\vspace{-3em}
\begin{code}
Počiatočný vklad v eurách: @\fbox{\phantom{vstup}}@
Úroková sadzba p.a. v %: @\fbox{\phantom{vstup}}@
Typ úročenia (jednoduché / zložené): @\fbox{\phantom{vstup}}@
Žiadaná suma v eurách: @\fbox{\phantom{vstup}}@
\end{code}
\end{tabular}

\vspace{-2em}
\begin{tabular}{@{}p{0.15\linewidth}p{0.75\linewidth}}
\textbf{\small Výstup:} &
\vspace{-3em}
\begin{code}
Rok      Suma						Úrok
 1.		@\fbox{\phantom{vstup}}@ Eur	@\fbox{\phantom{vstup}}@ Eur
 2.		@\fbox{\phantom{vstup}}@ Eur	@\fbox{\phantom{vstup}}@ Eur
\end{code}
\end{tabular}
\vspace{-2em}

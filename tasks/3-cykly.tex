\subsection{Cykly}
Obrovský potenciál počítačov tkvie v bezchybnom neúnavnom vykonávaní presne zadaných inštrukcií. Cykly umožňujú opakovať rovnaký postup ľubovoľný počet krát a tým efektívne odstraňovať rutinnú prácu.


\subsubsection*{1. 100-krát napíš}
Za vyrušovanie na hodinách sa stalo populárnym trestom ručné prepisovanie mravoučnej vety stokrát. Stalo sa to tak neznesiteľné, že si zhotovil robota, ktorý vie pomocť záškodníkom. Chýbajú mu len príkazy, čo má vlastne robiť.

\begin{code}
Musím napísať: _____
Toľkoto krát: ____

______
______
...
\end{code}


\subsubsection*{2. Hodnotenie}
Filmový kritici a hodnotitelia reštauracií zapíšu po namáhavom dni číselné skóre k ich recenziam. Pre lepší efekt potrebujú vykresliť hviezdničky namiesto čísla. Pomôž im.

\begin{code}
Skóre: 5

*****
\end{code}

\subsubsection*{3. Pyramída}
Hviezdičky zoskup do tvaru pyramídy zadanej výšky.

\begin{code}
Výška pyramídy: 4

   *
  ***
 *****
*******
\end{code}

\subsubsection*{4. Smaragd}
Na pyramídu pripoj zo spodu ďaľšiu obrátene, aby vznikol smaragd z hviezdičiek.

\begin{code}
Veľkosť: 5

  *
 ***
*****
 ***
  *
\end{code}

\subsubsection*{5. Duté vnútro}
Nakresli duté pyramídu a smaragd podľa prechádzajúcich úloh.

\begin{code}
Výška pyramídy: 4

    *
   * *
  *   *
 *******
\end{code}


\subsubsection*{6. Mriežka slov}
Načítajte veľkosť tabuľky a slovo, ktoré sa v nej bude na každom riadku v stĺpci opakovať.

\begin{code}
Počet riakov a stĺpcov: 4
Opakovať slovo: ano

ano ano ano ano
ano ano ano ano
ano ano ano ano
ano ano ano ano
\end{code}



\subsubsection*{7. Rám}
Prvý a posledný riadok a stĺpec bude tvoriť rám pre mriežku slov.

\begin{code}
Počet riakov a stĺpcov: 4
Opakovať slovo: ano

### ### ### ###
### ano ano ###
### ano ano ###
### ### ### ###
\end{code}


\subsubsection*{8. Malá násobilka}
K výbave každého žiaka základnej školy patrí tabuľky malej násobilky. Vytvor takúto tabuľku obsahujúcu každý násobok od 1x1 po 10x10, aby si pomohol všetkým malým matematikom.

\begin{code}
   1   2   3   4   5   6   7   8   9  10
   2   4   6   8  10  12  14  16  18  20
   3   6   9  12  15  18  21  24  27  30
   4   8  12  16  20  24  28  32  36  40
   5  10  15  20  25  30  35  40  45  50
   6  12  18  24  30  36  42  48  54  60
   7  14  21  28  35  42  49  56  63  70
   8  16  24  32  40  48  56  64  72  80
   9  18  27  36  45  54  63  72  81  90
   10  20  30  40  50  60  70  80  90 100
\end{code}


\subsubsection*{9. Sporenie}
Na letnej brigáde si zarobil peniaze, ktoré chceš usporiť. Porovnáš ponuky bánk a hľadáš najvýhodnejší plán. Vytvor si sporiacu kalkulačku, ktorá na základe nemenného počiatčného vkladu, ročnej úrokovej sadzby, typu úročenia a žiadanej konečnej sumy, vypíše vývoj tvojich finančných prostriedkov do budúcnosti.

\begin{code}
Vklad v Eur: ____
Úroková sadzba p.a. v %: _____
Typ úročenia (jednoduché / zložené): _____
Žiadaná suma v Eur:

Rok      Suma         Úrok
  1.	     ______ Eur   _____ Eur
  2.     ______ Eur   _____ Eur
....
\end{code} 

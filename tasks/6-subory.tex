\subsection{Súbory}
\underline{\textbf{Súbor}} je zoskupením súvisiacich údajov, ktoré sú uložené na disku počítača. Oproti načítavaniu vstupu z klávesnice majú výhodu hlavne pri spracovaní a uchovaní veľkého množstva dát. Súbory sa dajú: 
\begin{itemize}[noitemsep]
\item \underline{vytvoriť} alebo \underline{vymazať}
\item \underline{otvoriť} alebo \underline{zatvoriť}
\item \underline{čítať} alebo \underline{zapisovať}
\end{itemize}.

Podľa typu uchovávaných údajov (označované \underline{\textbf{príponou}}) súbory rozdeľujeme na:
\begin{itemize}[noitemsep]
\item \textbf{Textové súbory} - .txt, .csv, .html, .py
\item \textbf{Obrazové súbory} - .bmp, .png, .jpg, .gif, .svg, .pdf
\item \textbf{Zvukové súbory} - .wav, .mp3, .midi
\item \textbf{Video súbory} - .avi, .mp4, .mkv
\item \textbf{Spustiteľné súbory} - .exe, .elf
\end{itemize}
V tejto kapitole budeme pre jednoduchosť pracovať s textovými súbormi.

\subsubsection*{1. Prepisovanie}
Príde ti zbytočne prepisovať dlhé články na vstup programu a vždy sa pomýliš. Načítaj články pre každú úlohu z predošlej kapitoly zo súboru. Uprav programy tak, aby si najprv vypýtali názov súboru. V úlohe ,,veľa opakovania'' ulož záznam o ceste robota do nového súboru.


\subsubsection*{2. Turistika}
Na víkend sa črtajú ideálne podmienky na horskú turistiku. Nenecháš nič na náhodu a pripravíš si detailný plán s výškovým profilom trasy. Na každých desať metrov trasy si do súboru poznačíš nadmorskú výšku z mapy. Zisti celkové stúpanie a klesanie počas celého výletu spolu s najvyššou a najnižšou nadmorskou výškou. Vypíš aj celkovú dĺžku túry v kilometroch a trvanie prechodu horami v hodinách.

\begin{tabular}{@{}p{0.2\linewidth}p{0.7\linewidth}}
\textbf{\small Obsah súboru:} &
\vspace{-3em}
\begin{code}
348
351
379
384
395
401
396
\end{code}
\end{tabular}

\vspace{-2em}
\begin{tabular}{@{}p{0.2\linewidth}p{0.7\linewidth}}
\textbf{\small Vstup:} &
\vspace{-3em}
\begin{code}
Trasa je v súbore s názvom: @\fbox{\phantom{vstup}}@
\end{code}
\end{tabular}

\vspace{-2em}
\begin{tabular}{@{}p{0.2\linewidth}p{0.7\linewidth}}
\textbf{\small Výstup:} &
\vspace{-3em}
\begin{code}
Trasa: @\fbox{0.140 km}@ - @\fbox{0}@ h @\fbox{21}@ min
Stúpanie: @\fbox{53}@ m
Klesanie: @\fbox{40}@ m
Najnižšie miesto trasy: @\fbox{361}@ m
Najvyššie miesto trasy: @\fbox{401}@ m
\end{code}
\end{tabular}
\vspace{-2em}


\subsubsection*{3. Vedomostný kvíz}
Bifľovanie ti vôbec nepríde prínosné. Keby existoval spôsob, akým si opakovanie učiva spríjemniť. Včera si zo smútku nad vidinou takto premárneho času, pri jedení čokolády a čipsov, pozeral kvízovú reláciu. Prišlo ti to neuveriteľne poučné. Polož náhodnú otázku s možnostami zo súboru kvízových otázok a bodovo ohodnoť správnu odpoveď. Všetky kvízové otázky s možnosťami sa však nezmestia do pamäti programu. Náhodnu otázku vyber priamo zo súboru.

\begin{tabular}{@{}p{0.2\linewidth}p{0.7\linewidth}}
\textbf{\small Obsah súboru:} &
\vspace{-3em}
\begin{code}
Otázka: V ktorom roku začala Francúzska revolúcia?
  A: 1763
  B: 1813
  C: 1789
  D: 1654
Odpoveď: C
Otázka: Al2O3 je?
  A: hydroxid vápenatý
  B: oxid hlinitý
  C: hydroxid sodný
Odpoveď: B
\end{code}
\end{tabular}

\vspace{-2em}
\begin{tabular}{@{}p{0.2\linewidth}p{0.7\linewidth}}
\textbf{\small Ukážka:} &
\vspace{-3em}
\begin{code}
Trasa je v súbore s názvom: @\fbox{\phantom{vstup}}@
\end{code}
\end{tabular}

\vspace{-2em}
\begin{tabular}{@{}p{0.2\linewidth}p{0.7\linewidth}}
\textbf{\small Kvíz:} &
\vspace{-3em}
\begin{code}
Súbor s kvízovými otázkami: @\fbox{kviz.txt}@
Kvízové otázky pripravené.
Ideme na to!

V ktorom roku sa začala Francúzska revolúcia?
A: 1763
B: 1813
C: 1789
D: 1654
Aká je správna odpoveď?: @\fbox{C}@
Správne! Máš 1 bodov. 
(alebo) Nabudúce si to lepšie premysli. Skúsime niečo iné.
\end{code}
\end{tabular}
\vspace{-2em}


\subsubsection*{4. Narodeniny}
Darčeky k narodeninám zvykneš kupovať na poslednú chvílu. Potrebuješ mať prehľad aspoň na mesiac dopredu, kto bude mať narodeniny, aby si stihol vybrať niečo výnimočné. Zo súboru načítaj ľudí, ktorí majú sviatok v požadovaný mesiac v roku.

\begin{tabular}{@{}p{0.2\linewidth}p{0.7\linewidth}}
\textbf{\small Obsah súboru:} &
\vspace{-3em}
\begin{code}
Jožko Mrkvička, 15.3.2002
Katka Krátka, 2.7.1993
Martinko Klingáč, 12.11.1995
Iveta Novotná, 27.2.2001
\end{code}
\end{tabular}

\vspace{-2em}
\begin{tabular}{@{}p{0.2\linewidth}p{0.7\linewidth}}
\textbf{\small Vstup:} &
\vspace{-3em}
\begin{code}
Zobraz narodeniny pre mesiac v roku: @\fbox{3.2019}@
\end{code}
\end{tabular}

\vspace{-2em}
\begin{tabular}{@{}p{0.2\linewidth}p{0.7\linewidth}}
\textbf{\small Výstup:} &
\vspace{-3em}
\begin{code}
Narodeniny: @\fbox{Marec 2019}@
@\fbox{15.3. - Jožko Mrkvička - 17 rokov}@
\end{code}
\end{tabular}
\vspace{-2em}


\subsubsection*{5. Cestovné poriadky}
Z celoštátneho rýchlika prestupujú cestujúci v okresných mestách na miestne autobusy. Podľa času odchodu a trvania cesty zisti, ktorý autobus stihnú. Vypíš najbližší spoj s najmenším čakaním medzi vlakom a autobusom. Daj pozor! Prvý časový údaj v riadku s odchodom autobusu je trvanie cesty vlakom  do stanice, odkiaľ odchádza ten autobus.

\begin{tabular}{@{}p{0.2\linewidth}p{0.7\linewidth}}
\textbf{\small Obsah súboru:} &
\vspace{-3em}
\begin{code}
vlak,9:15,10:45,12:15,14:30,16:15,18:20
bus,1:00,11:00,13:00,15:00,17:00
bus,1:45,9:30,12:08,16:33
\end{code}
\end{tabular}

\vspace{-2em}
\begin{tabular}{@{}p{0.2\linewidth}p{0.7\linewidth}}
\textbf{\small Vstup:} &
\vspace{-3em}
\begin{code}
Čas: @\fbox{10:00}@
Trvanie cesty vlakom: @\fbox{1:00}@
\end{code}
\end{tabular}

\vspace{-2em}
\begin{tabular}{@{}p{0.2\linewidth}p{0.7\linewidth}}
\textbf{\small Výstup:} &
\vspace{-3em}
\begin{code}
Najbližší spoje (vlak, autobus):
@\fbox{12:15 - 13:15, 15:00 -}@
\end{code}
\end{tabular}
\vspace{-2em}

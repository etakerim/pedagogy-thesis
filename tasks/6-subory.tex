

\subsection{Súbory}
\textbf{Súbor} je zoskupením súvisiacich údajov, ktoré sú uložené na disku počítača. Oproti načítavaniu vstupu z klávesnice majú výhodu hlavne pri spracovaní a uchovaní veľkého množstva dát. Súbory sa dajú: \textit{vytvoriť} / \textit{vymazať}, \textit{otvoriť} / \textit{zatvoriť}, \textit{čítať} / \textit{zapisovať}. Podľa typu uchovávaných údajov (označované \textit{príponou}) súbory rozdeľujeme na:

\begin{itemize}
\itemsep0pt
\item \textbf{Textové súbory} - .txt, .csv, .html, .py
\item \textbf{Obrazové súbory} - .bmp, .png, .jpg, .gif, .svg, .pdf
\item \textbf{Zvukové súbory} - .wav, .mp3, .midi
\item \textbf{Video súbory} - .avi, .mp4, .mkv
\item \textbf{Spustiteľné súbory} - .exe, .elf
\end{itemize}

V tejto kapitole budeme pre jednoduchosť pracovať s textovými súbormi:

\subsubsection*{1. Prepisovanie}
Pri prepisovaní dlhých textov na vstup programu sa často mýliš a príde ti to zbytočne zdĺhavé. Načítaj články u zadaní z predchádajúcej kapitoly zo súboru, ktorého názov si na začiatku vypýtaš. Pri úlohe "veľa opakovania" ulož záznam o ceste robota do nového súboru.


\subsubsection*{2. Turistika}
Na víkend sa črtajú ideálne podmienky na horskú turistiku. Nenecháš nič na náhodu a pripravíš si detailný plán s výškovým profilom trasy. Na každých desať metrov trasy si do súboru poznačíš aktuálnu nadmorskú výšku. Zisti celkové stúpanie a klesanie počas celého výletu spolu s najvyššou a najnižšou nadmorskou výškou. Vypíš aj celkovú dĺžku túry v kilometroch a trvanie prechodu horami v hodinách.

\begin{minipage}{.35\textwidth}
\textbf{Vzorový obsah súboru (trasa.txt):}
\begin{code}
348
351
379
384
395
401
396
383
381
367
361
\end{code}
\end{minipage}
\hfill
\begin{minipage}{.45\textwidth}
\textbf{Turistika:}
\begin{code}
Trasa je v súbore: ______

Trasa: 0.140 km - 0 h 21 min
Stúpanie: 53 m
Klesanie: 40 m
Najnižšie miesto trasy: 361 m
Najvyššie miesto trasy: 401 m
\end{code}
\end{minipage}

\subsubsection*{3. Vedomostný kvíz}
Bifľovanie ti vôbec nepríde ako zábava. Keby existoval spôsob, ktorým si opakovanie poznatkov spríjemniť. Včera si zo smútku nad vidinou takto premárneho času pri jedení čokolády a čipsov pozeral kvízovú reláciu. Prišlo ti to neuveriteľne poučné. Polož náhodnú otázku s možnostami zo súboru kvízových otázok a bodovo ohodnoť správnu odpoveď. Všetky kvízové otázky s možnosťami sa však nezmestia do pamäti programu, preto vždy vyber náhodnu otázku priamo zo súboru.

\paragraph{Obsah súboru (kviz.txt):}

\begin{code}
Otázka: V ktorom roku sa začala Francúzska revolúcia?
  A: 1763
  B: 1813
  C: 1789
  D: 1654
Odpoveď: C
Otázka: Al2O3 je?
  A: hydroxid vápenatý
  B: oxid hlinitý
  C: hydroxid sodný
Odpoveď: B
\end{code}

\paragraph{Kvíz:}

\begin{code}
Súbor s kvízovými otázkami: kviz.txt
Kvízové otázky pripravené.
Ideme na to!

V ktorom roku sa začala Francúzska revolúcia?
A: 1763
B: 1813
C: 1789
D: 1654
Aká je správna odpoveď?: C
Správne! Máš 1 bodov. / Nabudúce si to lepšie premysli. Skúsime niečo iné.
\end{code}


\subsubsection*{4. Narodeniny}
Darčeky k narodeninám zvykneš kupovať na poslednú chvílu. Potrebuješ mať prehľad aspoň na mesiac dopredu, kto bude mať narodeniny, aby si stihol vybrať niečo výnimočné. Zo súboru načítaj ľudí, ktorí majú sviatok v požadovaný mesiac v roku.

\paragraph{Obsah súboru (narodeniny.csv):}

\begin{code}
Jožko Mrkvička, 15.3.2002
Katka Krátka, 2.7.1993
Martinko Klingáč, 12.11.1995
Iveta Novotná, 27.2.2001
...
\end{code}

\paragraph{Oslavy:}

\begin{code}
Zobraz narodeniny pre mesiac v roku: 3.2019

Narodeniny: Marec 2019
15.3. - Jožko Mrkvička - 17 rokov
\end{code}


\subsubsection*{5. Cestovné poriadky}
Z celoštátneho rýchlika prestupujú v okresných mestách cestujúci na miestne autobusy.  Podľa času odchodu a trvania cesty zisti, ktorý autobus stihnú a vypíš najbližší spoj s najmenším čakaním medzi vlakom a autobusom. Daj pozor, pretože prvý časový údaj v riadku s odchodmi autobusu je v skutočnosti trvanie cesty vlakom, kým sa dostaneš do stanice, odkiaľ odchádza ten autobus.

\paragraph{Obsah súboru (cp.csv):}
\begin{code}
vlak,9:15,10:45,12:15,14:30,16:15,18:20
bus,1:00,11:00,13:00,15:00,17:00
bus,1:45,9:30,12:08,16:33
...
\end{code}

\paragraph{Cestovné poriadky:}
\begin{code}
Čas: 10:00
Trvanie cesty vlakom: 1:00

Najbližší spoje (vlak, autobus):
12:15 - 13:15, 15:00 -
\end{code}

\subsubsection*{6. Pripomienky v kalendári}
Po čase zistíš, že jednoduchšie by bolo, ak by sa ti týždeň pred kamarátovými narodeninami objavila pripomienka v tvojom osobnom elektronickom kalendári. Máš veľa kontaktov, nechceš ich však všetky prepisovať ručne. Zistiš, že zoznam narodenín môžeš do kalendárovej aplikácie vložiť vo formáte \textit{iCalendar (.ics)}. Preveď súbor s menami a dátumami narodenia do tejto podoby.

\textbf{Pozri:}
\begin{itemize}
\itemsep0pt
\item \textbf{iCalendar - súborový formát}: \url{https://cs.wikipedia.org/wiki/ICalendar}, 
\item \textbf{iCalendar - podrobný popis [EN]}: \url{https://icalendar.org/RFC-Specifications/iCalendar-RFC-5545/}
\end{itemize}
 

\paragraph{Pripomienky (narodeniny.ics)}
\begin{code}
BEGIN:VCALENDAR
PRODID:Programatorsky kruzok
VERSION:2.0
...
BEGIN:VEVENT
DTSTAMP:20190811T100534Z
UID:1
SUMMARY:Jožko Mrkvička narodeniny
CATEGORIES:Narodeniny
RRULE:FREQ=YEARLY
DTSTART;VALUE=DATE:20020315
DTEND;VALUE=DATE:20020316
TRANSP:TRANSPARENT
BEGIN:VALARM
DESCRIPTION:
ACTION:DISPLAY
TRIGGER:-P7D
END:VALARM
END:VEVENT
...
END:VCALENDAR
\end{code}

\subsubsection*{7. Spisovateľ}
Každý nemôže mať doma vlastného Hviezdoslava. Nebolo by ale úžastné, keby si mohol tvoriť básne alebo prózu s podobným štýlom ako jeden z velikánov literatúry? Vzrušujúcejšie by bolo naučiť počítač umeleckému cíteniu. Najprv musíš zhromaždiť, čo najväčší počet ukážok tvorby autora, a tým zhromaždiť pravdepodobnosti následnosti \textit{n-gramov} (písmen, slabík, slov) do \textit{Markovovho reťazca}. Potom náhodne vygeneruj nový text v štýle autora. Žiaľ, vytvorené myšlienky zrejme nebudú dávať poväčšinou významovo zmysel.

\textbf{Pozri:}
\begin{itemize}
\itemsep0pt
\item \textbf{Diela slovenskej literatúry}: \url{https://zlatyfond.sme.sk/}, 
\item \textbf{Anglické texty}: \url{https://archive.org/search.php?query=subject%3A%22Literature%22}, 
\item \textbf{Stavové automaty vizuálne [EN]} \url{http://setosa.io/ev/markov-chains/}, 
\item \textbf{Tvorba slov pravdepodobnosťou - str.7 [EN]} \url{http://math.harvard.edu/~ctm/home/text/others/shannon/entropy/entropy.pdf}
\end{itemize}

\begin{code}
Chcem písať ako: Dostojevskij
Dĺžka n-gramu: 2
Počet znakov výsledného textu: 100

Spracúvam korpus tvorby autora ...
Spočítavam maticu prechodových stavov ...
Generujem originálny text ...
Ani v tmi, že páliciu neď si predtým opohľadíka do do nia nehľadík, hľadal nediva ulic
\end{code} 

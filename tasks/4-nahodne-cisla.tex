\subsection{Náhodné čísla}
Pri tvorbe simulácií sú náhodné čísla nepostrádateľné. Umožňujú vniesť variabilitu a rôznorodosť do inak statických scén. Nesmierne poslúžia v hrách, kde dovoľujú modelovať napríklad pravdepodobnosť výskytu monštier, či pokladov.


\subsubsection*{1. Hádzanie kockou}
Vytvorte simuláciu hodu kockou. Po stlačení klávesy Enter sa nakreslí kocka s padnutým číslom.

\begin{tabular}{@{}p{0.15\linewidth}p{0.75\linewidth}}
\textbf{\small Vstup:} &
\vspace{-3em}
\begin{code}
HOĎ@\fbox{<ENTER>}@
\end{code}
\end{tabular}

\vspace{-2em}
\begin{tabular}{@{}p{0.15\linewidth}p{0.75\linewidth}}
\textbf{\small Výstup:} &
\vspace{-3em}
\begin{code}
+-------+
| #   # |
|   #   |
| #   # |
+-------+
\end{code}
\end{tabular}
\vspace{-2em}

\subsubsection*{2. Hádaj číslo}
Náhodne vyber číslo s rozsahu medzi 0 a 100 a nechaj hráča hádať dokým neuhádne. Pri tom mu poskytni nápovedy, či je jeho tip priveľa alebo primalo. Zakomponuj rôzne obtiažnosti s možnosťou nastavenia rozsahu alebo maximálnym počtom tipov.

\begin{tabular}{@{}p{0.15\linewidth}p{0.75\linewidth}}
\textbf{\small Vstup:} &
\vspace{-3em}
\begin{code}
Hádaj číslo: @\fbox{8}@
Hádaj číslo: @\fbox{18}@
Hádaj číslo: @\fbox{13}@
\end{code}
\end{tabular}

\vspace{-2em}
\begin{tabular}{@{}p{0.15\linewidth}p{0.75\linewidth}}
\textbf{\small Výstup:} &
\vspace{-3em}
\begin{code}
Málo
Veľa
Výborne. Uhádol si!
\end{code}
\end{tabular}
\vspace{-2em}

\subsubsection*{3. Opakovanie násobilky}
Vďaka tvojej tabuľke malej násobilky sa malý školáci mohli naučiť násobiť. Ako dobre to vedia, musíš teraz odtestovať. Vygeneruj dve čísla od 1 do 10 do príkladu na násobenie. Over správnosť žiačikovej odpovede.

\begin{tabular}{@{}p{0.15\linewidth}p{0.75\linewidth}}
\textbf{\small Vstup:} &
\vspace{-3em}
\begin{code}
Koľko je @\fbox{\phantom{vstup}}@ x @\fbox{\phantom{vstup}}@?
= @\fbox{\phantom{vstup}}@
Chceš ďalší príklad (a / n)?  @\fbox{\phantom{vstup}}@
\end{code}
\end{tabular}

\vspace{-2em}
\begin{tabular}{@{}p{0.15\linewidth}p{0.75\linewidth}}
\textbf{\small Výstup:} &
\vspace{-3em}
\begin{code}
Správne - len tak ďalej / Nesprávne - hádaj znovu
\end{code}
\end{tabular}
\vspace{-2em}

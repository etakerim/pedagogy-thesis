
\subsection{Náhodné čísla}
Pri tvorbe simulácií sú náhodné čísla nepostrádateľné. Umožňujú vniesť variabilitu a rôznorodosť do inak statických scén. Nesmierne poslúžia v hrách, kde dovoľujú modelovať napríklad pravdepodobnosť výskytu monštier, či pokladov.


\subsubsection*{1. Hádzanie kockou}
Vytvorte simuláciu hodu kockou. Po stlačení klávesy Enter sa nakreslí kocka s padnutým číslom.

\begin{code}
HOĎ<ENTER>
+-------+
| #   # |
|   #   |
| #   # |
+-------+
\end{code}

\subsubsection*{2. Hádaj číslo}
Náhodne vyber číslo s rozsahu medzi 0 a 100 a nechaj hráča hádať dokým neuhádne. Pri tom mu poskytni nápovedy, či je jeho tip priveľa alebo primalo. Zakomponuj rôzne obtiažnosti s možnosťou nastavenia rozsahu alebo maximálnym počtom tipov.

\begin{code}
Hádaj číslo: 8
Málo
Hádaj číslo: 18
Veľa
Hádaj číslo: 13
Výborne. Uhádol si!
\end{code}

\subsubsection*{3. Opakovanie násobilky}
Vďaka tvojej tabuľke malej násobilky sa malý školáci mohli naučiť násobiť. Ako dobre to vedia, musíš teraz odtestovať. Vygeneruj dve čísla od 1 do 10 do príkladu na násobenie. Over správnosť žiačikovej odpovede.

\begin{code}
Koľko je ____ x _____?
= _____
Správne - len tak ďalej / Nesprávne - hádaj znovu
Chceš ďalší príklad?
\end{code}
 

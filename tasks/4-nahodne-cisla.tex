\subsection{Náhodné čísla}
Pri tvorbe simulácií sa bez náhodné čísla nezaobídeme. Umožňujú vniesť nečakané javy a rôznorodosť do inak nemeniacich sa scén. Nesmierne poslúžia v hrách, kde dovoľujú meniť napríklad výskyt monštier, či pokladov.

\subsubsection*{1. Hádzanie kockou}
Hranie človeče nehnevaj sa zaberie pokojne celé popoludnie. Chvíľa nepozornosti stačí, aby sa kocka nadobro zatúlala pod ťažký gauč. Vytvor si namiesto zapadnutej kocky program, ktorý napodobní jej hod. Po stlačení klávesy Enter sa nakreslí kocka s padnutým číslom. Hodené číslo je po každom spustení programu iné.

\begin{tabular}{@{}p{0.15\linewidth}p{0.75\linewidth}}
\textbf{\small Vstup:} &
\vspace{-3em}
\begin{code}
HOĎ@\fbox{<ENTER>}@
\end{code}
\end{tabular}

\vspace{-2em}
\begin{tabular}{@{}p{0.15\linewidth}p{0.75\linewidth}}
\textbf{\small Výstup:} &
\vspace{-3em}
\begin{code}
+-------+
| #   # |
|   #   |
| #   # |
+-------+
\end{code}
\end{tabular}
\vspace{-2em}

\subsubsection*{2. Hádaj číslo}
Hádaj na čo práve myslím bude až do vynálezu telepatie zábavná kratochvíľa. Okrem osobností, vecí a miest sa zvyknú tipovať aj čísla. Nechaj program náhodne vybrať číslo od 0 po 100. Hráč bude ho hádať až pokým neuhádne. Poskytni mu po každom pokuse nápoveď, či povedal priveľa alebo primálo. Potom doplň do programu rôzne náročnosti. Môže ísť o napríklad s možnosť nastaviť rozsah čísel alebo maximálny počet tipov.

\begin{tabular}{@{}p{0.15\linewidth}p{0.75\linewidth}}
\textbf{\small Vstup:} &
\vspace{-3em}
\begin{code}
Hádaj číslo: @\fbox{8}@
Hádaj číslo: @\fbox{18}@
Hádaj číslo: @\fbox{13}@
\end{code}
\end{tabular}

\vspace{-2em}
\begin{tabular}{@{}p{0.15\linewidth}p{0.75\linewidth}}
\textbf{\small Výstup:} &
\vspace{-3em}
\begin{code}
Málo
Veľa
Výborne. Uhádol si!
\end{code}
\end{tabular}
\vspace{-2em}

\subsubsection*{3. Opakovanie násobilky}
Vďaka tvojej tabuľke malej násobilky sa malí školáci mohli naučiť násobiť. Ako dobre to vedia, musíš teraz otestovať. Vygeneruj dve čísla od 1 do 10 do príkladu na násobenie. Over správnosť žiačikovej odpovede.

\begin{tabular}{@{}p{0.15\linewidth}p{0.75\linewidth}}
\textbf{\small Vstup:} &
\vspace{-3em}
\begin{code}
Koľko je @\fbox{\phantom{vstup}}@ x @\fbox{\phantom{vstup}}@?
= @\fbox{\phantom{vstup}}@
Chceš ďalší príklad (a / n)?  @\fbox{\phantom{vstup}}@
\end{code}
\end{tabular}

\vspace{-2em}
\begin{tabular}{@{}p{0.15\linewidth}p{0.75\linewidth}}
\textbf{\small Výstup:} &
\vspace{-3em}
\begin{code}
Správne - len tak ďalej / Nesprávne - hádaj znovu
\end{code}
\end{tabular}
\vspace{-2em}

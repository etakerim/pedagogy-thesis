\subsection{Podmienky}
\textbf{Podmienky} sú ako križovatky na ceste. Podľa toho kam chceme ísť, sa rozhodneme, ktorou cestou pôjdeme ďalej. Aby sme sa uistili, že máme ten správny smer (*vetva podmienky*) pýtame sa vždy logickú otázku pomocou už získaných údajov uložených v premenných.

\subsubsection*{1. Heslo}
Tvoj dom na strome už vykradlo pár nezvaných návštevníkov a preto si vymyslel spôsob ako dovoliť návštevu len povoleným osobám, ktoré poznajú tajné heslo.

\begin{code}
Stoj! Povedz Heslo!
> _____

Vstúp, priateľ /   Zmizni kade ľahšie
\end{code}


\subsubsection*{2. Najväčšie číslo}
Zisti, ktoré z troch zadaných čísel je najväčšie.

\begin{code}
1.číslo: ____
2.číslo: ____
3.číslo: ____

Najväčie je ____.číslo a to je ____.
\end{code}


\subsubsection*{3. Vhodné oblečenie}
Módni poradcovia vyšli z módy a ich prácu prebrali počítače. Na základe počasia a príležitosti odporúčajú vhodný outfit. Vymysli pár tipov pre rôzne situácie a začni radiť.

\begin{code}
Ako je vonku?: _____
Kam ideš?: ____

Určite si nezabudni _______ a tiež si vezmi _______.
\end{code}


\subsubsection*{4. Pokazený rozpis}
Podnik spracujúci rudu dostal časový rozpis trvania jednotlivých krokov vylepšeného technologického procesu. Činnosti zvyčajne trvajú dlhšie ako hodinu, nehodí sa im teda mať časy napísané iba ako údaj v minútach. Tvojou úlohou je rozpísať minúty na dni, hodiny, minúty pre jednoduchšie čítanie rozpisu. Vynechajte nepotrebné časové údaje.

\begin{code}
Trvanie (min.): ____
= ___ d. ____ hod. ___ min
\end{code}


\subsubsection*{5. Hovoriaca kalkulačka}
Výpočty neboli nikdy väčšia zábava, teda aspoň s kalkulačkou, ktorá namiesto čudných matematických znamienok hovorí ľudskou rečou. Vytvorte kalkulačku, ktorá si vypýta dve čísla a vie ich sčítať alebo odčítať.

\begin{code}
Som hovorica kalkulačka a rada počítam!
Povedz mi prvé číslo: ____
Potrebujem ďašie číslo: ____
Chceš ich sčítať alebo odčítať: ____ (sčítať / odčítať)

Výsledok tvojho príkladu: ____ plus/mínus _____ je ________.
\end{code}

\subsubsection*{6. Kvadratická rovnica}
Pre zadané koeficienty $a$, $b$, $c$ kvadratickej rovnice $ax^2 + bx + c = 0$  vypočítajte jej korene v obore reálnych čísel a vrchol paraboly daného predpisu.

\begin{code}
a = ____
b = ____
c = ____

___x^2 + ___x + ___ = 0
x1 = ____
x2 = ____
V[___; ___]
\end{code}

\subsubsection*{7. Trojuholníky}
Mýtická bytosť stredoškolskej matematiky, o ktorej je vždy treba zistiť, čo najviac bez rysovania, aj keď chýbajú rozmery.

\begin{enumerate}[label=\alph*)]
\item Ak je možné, doplň chýbajúce informácie pre ľubovoľný trojuholník (zadaný ako SSS) ako sú dĺžky strán a výšok, veľkosti uhlov, obsah a obvod. Využite trojuholníkoú nerovnosť, sínus(ovú) vetu, kosínus(ovú) vetu a vzorec na výpočet obsahu trojuholníkov.
\item Rozšírte vypočet aj pre ostatné vety o trojuholníkoch: SUS, USU, UUS
\end{enumerate}

\begin{code}
Zadajte strany ľubovolného trojuholníka:
a = ___
b = ___
c = ___

Strany: a = ___; b = ___; c = ___
Uhly: alpha = ___°; beta = ___°; gamma = ___°
Výšky: v(a) = ___; v(b) = ___; v(c) = ___
O = ___
S = ___
Trojuholník je: ____, _____
\end{code}
 

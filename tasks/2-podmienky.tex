\subsection{Podmienky}
\underline{\textbf{Podmienky}} sú ako križovatky na ceste. Podľa toho kam chceme ísť, sa rozhodneme, ktorou cestou pôjdeme ďalej. Aby sme sa uistili, že máme ten správny smer (\underline{\textbf{vetva podmienky}}) pýtame sa vždy logickú otázku. Otázka používa údaje uložené v premenných.

\subsubsection*{1. Heslo}
Tvoj dom na strome už vykradlo pár nezvaných návštevníkov. Vymyslel si preto spôsob ako povoliť návštevu len overeným osobám. Tie musia poznať tajné heslo. Napíš program, ktorý slovne privíta členov a odoženie zlodejov.

\begin{tabular}{@{}p{0.15\linewidth}p{0.75\linewidth}}
\textbf{\small Vstup:} &
\vspace{-3em}
\begin{code}
Stoj! Povedz Heslo!
? @\fbox{\phantom{vstup}}@
\end{code}
\end{tabular}

\vspace{-2em}
\begin{tabular}{@{}p{0.15\linewidth}p{0.75\linewidth}}
\textbf{\small Výstup:} &
\vspace{-3em}
\begin{code}
Vstúp, priateľ
(alebo Zmizni kade ľahšie)
\end{code}
\end{tabular}
\vspace{-2em}

\subsubsection*{2. Najväčšie číslo}
Na lúke sa hrajú šípky. Hráči si zapisujú dosiahnuté skóre na tabuľu. Dnes proti sebe hrali v partii traja protihráči. Napíš program, ktorý označí hráča s najväčším získaným počtom bodov.

\begin{tabular}{@{}p{0.15\linewidth}p{0.75\linewidth}}
\textbf{\small Vstup:} &
\vspace{-3em}
\begin{code}
1.skóre: @\fbox{\phantom{vstup}}@
2.skóre: @\fbox{\phantom{vstup}}@
3.skóre: @\fbox{\phantom{vstup}}@
\end{code}
\end{tabular}

\vspace{-2em}
\begin{tabular}{@{}p{0.15\linewidth}p{0.75\linewidth}}
\textbf{\small Výstup:} &
\vspace{-3em}
\begin{code}
Najväčie skóre @\fbox{\phantom{vstup}}@ bodov má @\fbox{\phantom{vstup}}@ hráč.
\end{code}
\end{tabular}
\vspace{-2em}


\subsubsection*{3. Vhodné oblečenie}
Módni poradcovia vyšli z módy a ich prácu prebrali počítače. Na základe počasia a príležitosti odporúčajú vhodný outfit. Vymysli pár tipov pre rôzne situácie a začni radiť.

\begin{tabular}{@{}p{0.15\linewidth}p{0.75\linewidth}}
\textbf{\small Vstup:} &
\vspace{-3em}
\begin{code}
Ako je vonku?: @\fbox{\phantom{vstup}}@
Kam ideš?: @\fbox{\phantom{vstup}}@
\end{code}
\end{tabular}

\vspace{-2em}
\begin{tabular}{@{}p{0.15\linewidth}p{0.75\linewidth}}
\textbf{\small Výstup:} &
\vspace{-3em}
\begin{code}
Určite si nezabudni @\fbox{\phantom{vstup}}@ a tiež si vezmi @\fbox{\phantom{vstup}}@.
\end{code}
\end{tabular}
\vspace{-2em}

\subsubsection*{4. Morský vánok}
Kapitán plachetnice na otvorenom oceáne musí mať vždy prehľad odkiaľ fúka vietor, aby odkormidloval do vytúženého cieľa. Príliš silné závany vetra môžu byť nebezpečné pre posádku. Polámať lodné sťažne, potrhať plachty, či zaplaviť palubu. Cez rádio dostáva plavidlo každý deň správy o predpovedi sily vetra v Beafortovej stupnici. Sila vetra je ňou vyjadrená do dvanástich stupňov od bezvetria až po orkán. Napíš program, ktorý kapitánovi vysvetlí stupeň vetra. Podľa stupnice určíme jeho pomenovanie, rýchlosti v námorných uzloch a očakávanej výšky vĺn.

\begin{tabular}{@{}p{0.15\linewidth}p{0.75\linewidth}}
\textbf{\small Vstup:} &
\vspace{-3em}
\begin{code}
Sila vetra na Beaufortovej stupnici: @\fbox{\phantom{12}}@
\end{code}
\end{tabular}

\vspace{-2em}
\begin{tabular}{@{}p{0.15\linewidth}p{0.75\linewidth}}
\textbf{\small Výstup:} &
\vspace{-3em}
\begin{code}
Vietor sa nazýva @\fbox{\phantom{vstup}}@.
Vietor má rýchlosť @\fbox{\phantom{vstup}}@ kt.
Očakávaná výška vĺn je @\fbox{\phantom{vstup}}@ m.
\end{code}
\end{tabular}
\vspace{-2em}


\subsubsection*{5. Pokazený rozpis}
Továreň na železnú rudu dostala nový časový rozpis vylepšeného technologického procesu. Spracovanie zvyčajne trvá dlhšie ako hodinu. Nehodí sa im teda mať časy napísané iba v minútach. Rozpíš programom minúty na dni, hodiny, minúty pre jednoduchšie čítanie rozpisu. Vynechaj nepotrebné časové údaje.

\begin{tabular}{@{}p{0.15\linewidth}p{0.75\linewidth}}
\textbf{\small Vstup:} &
\vspace{-3em}
\begin{code}
Trvanie (min.): @\fbox{\phantom{vstup}}@
\end{code}
\end{tabular}

\vspace{-2em}
\begin{tabular}{@{}p{0.15\linewidth}p{0.75\linewidth}}
\textbf{\small Výstup:} &
\vspace{-3em}
\begin{code}
= @\fbox{\phantom{vstup}}@ d. @\fbox{\phantom{vstup}}@ hod. @\fbox{\phantom{vstup}}@ min.
\end{code}
\end{tabular}
\vspace{-2em}


\subsubsection*{6. Hovoriaca kalkulačka}
Výpočty neboli nikdy väčšia zábava. Teda aspoň s kalkulačkou, ktorá namiesto čudných matematických čmáraníc hovorí ľudskou rečou. Vytvor program pre kalkulačku, ktorá si vypýta dve čísla. Tie bude ich vedieť sčítať alebo odčítať podľa slovného pokynu.

\begin{tabular}{@{}p{0.15\linewidth}p{0.75\linewidth}}
\textbf{\small Vstup:} &
\vspace{-3em}
\begin{code}
Som hovorica kalkulačka a rada počítam!
Povedz mi prvé číslo: @\fbox{\phantom{vstup}}@
Potrebujem ďašie číslo: @\fbox{\phantom{vstup}}@
Chceš ich sčítať alebo odčítať: @\fbox{\phantom{vstup}}@ (sčítať alebo odčítať)
\end{code}
\end{tabular}

\vspace{-2em}
\begin{tabular}{@{}p{0.15\linewidth}p{0.75\linewidth}}
\textbf{\small Výstup:} &
\vspace{-3em}
\begin{code}
Výsledok tvojho príkladu: @\fbox{\phantom{vstup}}@ (plus alebo mínus) @\fbox{\phantom{vstup}}@ je @\fbox{\phantom{vstup}}@.
\end{code}
\end{tabular}
\vspace{-2em}

\subsubsection*{7. Chaos v lístkoch}
Vyznať sa v linkách mestskej hromadnej dopravy si vyžaduje dlhoročné skúsenosti. Treba oplývať aj riadnou dávkou trpezlivosti. Ľahko sa nám stane, že omylom nasadneme do autobusu a hneď sa vydáme na okružnú jazdu po siedmich divoch sídliska. Horší zážitok je stretnutie revízora po zistení, že máme nesprávny lístok alebo že nemáme žiaden ... Postávaš pri automate na lístky a nevieš sa vysomáriť z množstva časov a zón v ponuke. Napíš program, ktorý podľa počtu zónu a trvania ceny vypíše cenu zľavneného lístka. Nájdi na internete aktuálnu tarifu MHD v tvojom meste.

\begin{tabular}{@{}p{0.15\linewidth}p{0.75\linewidth}}
\textbf{\small Vstup:} &
\vspace{-3em}
\begin{code}
Popíš mi svoju cestu s MHD
Koľko zón prejdeš?: @\fbox{\phantom{vstup}}@
Koľko minút má trvať cesta?: @\fbox{\phantom{vstup}}@
\end{code}
\end{tabular}

\vspace{-2em}
\begin{tabular}{@{}p{0.15\linewidth}p{0.75\linewidth}}
\textbf{\small Výstup:} &
\vspace{-3em}
\begin{code}
Zlavnený lístok stojí @\fbox{\phantom{vstup}}@ eur.
\end{code}
\end{tabular}
\vspace{-2em}


\subsubsection*{8. Kvadratická rovnica}
Matematika v škole dokáže byť poriadna otrava. Hlavne, keď od rána do večera nič iné nerobíš ako počítaš príklady na kvadratické rovnice. ,,Načo mám ten počítač'', pomyslíš si večer vo svetle stolnej lampy. Pre zadané koeficienty $a$, $b$, $c$ predpisu $ax^2 + bx + c = 0$ napíš program, ktorý vypočíta jej korene a vrchol paraboly.

\begin{tabular}{@{}p{0.15\linewidth}p{0.75\linewidth}}
\textbf{\small Vstup:} &
\vspace{-3em}
\begin{code}
Koeficienty kvadratickej rovnice:
a = @\fbox{\phantom{vstup}}@
b = @\fbox{\phantom{vstup}}@
c = @\fbox{\phantom{vstup}}@
\end{code}
\end{tabular}

\vspace{-2em}
\begin{tabular}{@{}p{0.15\linewidth}p{0.75\linewidth}}
\textbf{\small Výstup:} &
\vspace{-3em}
\begin{code}
@\fbox{\phantom{a}}@x^2 + @\fbox{\phantom{b}}@x + @\fbox{\phantom{c}}@ = 0
x1 = @\fbox{\phantom{abc}}@
x2 = @\fbox{\phantom{abc}}@
V[@\fbox{\phantom{abc}}@; @\fbox{\phantom{abc}}@]
\end{code}
\end{tabular}
\vspace{-2em}


\subsubsection*{9. Trojuholníky}
Trojuholník je mýtická bytosť, o ktorej je vždy treba zistiť. Nesmieme použiť pravítko, lebo to by nás čakala príliš jednoduchá výzva. Veď bez rysovania zistíme o tejto trojcípej paráde všeličo. Hoci aj keď jej chýbajú niektoré rozmery.

\begin{enumerate}[label=\alph*)]
\item Ak je to možné, doplň chýbajúce informácie pre ľubovoľný trojuholník (zadaný ako SSS) ako sú dĺžky strán a výšok, veľkosti uhlov, obsah a obvod. Využi trojuholníkovú nerovnosť, sínusovú vetu, kosínusovú vetu a vzorec na výpočet obsahu trojuholníkov.
\item Rozšír program aj pre ostatné vety o trojuholníkoch: SUS, USU, UUS.
\end{enumerate}

\begin{tabular}{@{}p{0.15\linewidth}p{0.75\linewidth}}
\textbf{\small Vstup:} &
\vspace{-3em}
\begin{code}
Zadajte strany ľubovoľného trojuholníka:
a = @\fbox{\phantom{vstup}}@
b = @\fbox{\phantom{vstup}}@
c = @\fbox{\phantom{vstup}}@
\end{code}
\end{tabular}

\vspace{-2em}
\begin{tabular}{@{}p{0.15\linewidth}p{0.75\linewidth}}
\textbf{\small Výstup:} &
\vspace{-3em}
\begin{code}
Strany: a = @\fbox{\phantom{abc}}@; b = @\fbox{\phantom{abc}}@; c = ___
Uhly: alpha = @\fbox{\phantom{abc}}@°; beta = @\fbox{\phantom{abc}}@°; gamma = @\fbox{\phantom{vstup}}@°
Výšky: v(a) = @\fbox{\phantom{abc}}@; v(b) = @\fbox{\phantom{abc}}@; v(c) = @\fbox{\phantom{abc}}@
O = @\fbox{\phantom{abc}}@
S = @\fbox{\phantom{abc}}@
Trojuholník je: @\fbox{\phantom{abc}}@, @\fbox{\phantom{abc}}@
\end{code}
\end{tabular}
\vspace{-2em}

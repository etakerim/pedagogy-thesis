\chapter{Cieľ a metodika práce}
Cieľom záverečnej práce je zostaviť rozšíriteľnú \textbf{zbierku úloh z programovania} so vzorovými riešeniami vo všeobecnovzdelávacom predmete informatika stredných škôl. Primárny zámer súčasne vyžaduje uspokojenie nasledujúcich požiadaviek pre pedagogickú prax na navrhnutý systém úloh:

\begin{itemize}[noitemsep]
\item pokrytie základného učiva v súlade so ŠVP a cieľovými požiadavkami 
\item poskytnutie metódy na spôsob zaradenia ďalších úloh do systému úloh
\item skóre čitateľnosti zadaní textu primerané žiakom vo veku 15 - 18 rokov
\item príbehovosť opisu problémovej situácie v znení zadania
\item postupne odkrývajúce sa vzorové riešenia v programovacom jazyku Python
\item zobrazenie aj na webovej stránke s efektívnym využitím hypertextu
\end{itemize}

Splnenie vytýčených cieľov sa opieralo o rešerš informačných zdrojov vyhľadaných prostredníctvom plnotextových vyhľadávačov webu a knižníc, na základe rád odborníka a osobného povedomia o oblasti. Knihy a články boli podrobené procese analýzy, kedy sa vyselektovali relevantné definície, charakteristiky, kategorizácie a postupy. Metóde porovnania sa podrobili existujúce učebné texty úvodov do programovania. 

Vlastná tvorba zbierky úloh prebiehala syntézou psychologických východísk porozumenia textu, známych typových úloh a známej metodiky na preformulovanie úloh. Hodnotenie kvality textu úloh v zbierke sa uskutočnilo kvantitatívne i kvalitatívne. Úroveň čitateľnosti sa vyčíslila pomocou Flesh skóre (Vzorec~\ref{equ:flesh-score}) a Fog indexu (Vzorec~\ref{equ:fog-index}). Kvalitatívne sa pozorovali obtiaže žiakov pri riešenia vybranej úlohy na vyučovacej hodine plynúcej z neporozumenia formulácie zadania. Metóda klasifikácie úlohy v systéme vychádza z kritérii témy, podtémy, elementu, didaktickej funkcie a kognitívnej úrovne podľa Minďákovej (Sekcia~\ref{sec:klasifikacia-ulohy}).

\chapter{Cieľ a metodika práce}
Cieľom záverečnej práce je zostaviť rozšíriteľnú \textbf{zbierku úloh z programovania} so vzorovými riešeniami v predmete informatika pre stredné školy. Primárny zámer súčasne vyžaduje uspokojenie nasledujúcich požiadaviek na navrhnutý systém úloh pre pedagogickú prax:

\begin{itemize}[noitemsep]
\item pokrytie základného učiva v súlade so ŠVP a cieľovými požiadavkami
\item poskytnutie metodiky na spôsob zaradenia ďalších úloh do nášho systému úloh
\item čitateľnosť textu zadaní primeranej žiakom vo veku 15 - 18 rokov
\item príbehovosť opisu problémovej situácie v znení zadania
\item vzorové riešenia v programovacom jazyku Python
\item zobrazenie na webovej stránke s efektívnym využitím hypertextu
\end{itemize}

Splnenie vytýčených cieľov sa opiera o rešerš informačných zdrojov vyhľadaných prostredníctvom plnotextových vyhľadávačov webu a knižníc, na základe rád školiteľky a osobného povedomia o oblasti. Knihy a články boli podrobené procesu analýzy, kedy sa vyselektovali relevantné definície, charakteristiky, kategorizácie a postupy. Metóde porovnania sa podrobili existujúce učebné texty základov programovania.

Vlastná tvorba zbierky úloh prebiehala syntézou psychologických východísk porozumenia textu, známych typových úloh a známej metodiky na preformulovanie úloh. Hodnotenie kvality textu úloh v zbierke sa uskutočnilo kvantitatívne i kvalitatívne. Úroveň čitateľnosti sa indikatívne vyčíslila pomocou Gunning fog indexu (Vzorec~\ref{equ:fog-index}). Kvalitatívne sa pozorovali problémy žiakov pri riešenia úloh v zbierke na vyučovacej hodine plynúcej z neporozumenia formulácie zadania. Metodika klasifikácie úlohy v systéme vychádza z kritérií témy, podtémy, elementu, didaktickej funkcie a kognitívnej úrovne podľa Minďákovej (Sekcia~\ref{sec:klasifikacia-ulohy}).

\chapter{Záver}
V záverečnej práci sme poukázali na problém s nedostatkom kvalitných učebníc v stedoškolskej informatike a problémových cvičení, ktoré by podnecovali objavnú tvorivú činnosť žiakov a umožňovali samostatnú domácu prípravu. Ako reakciu na tieto výzvy sme vytvorili súbor 54 slovných úloh na programovanie so vzorovými riešenami v jazyku Python. 

Učebnou náplňou sa v zbierke dohromady presahujú požiadavky na základné učivo vymedzené obsahovým a výkonovým štandardom štátneho vzdelávacieho programu. Rozsahu vzdelávacích štandardov pre algoritmizáciu a programovanie sa venujú kapitoly zo našej zbierky, hlavne: premenné, podmienky, cykly. Pre školy s rozšíreným ŠkVP infomatiky oproti ŠVP a pre maturantov z informatiky sú určené kapitoly: náhodné čísla, reťazce a zoznamy, súbory, funkcie. 

V druhej časti čerpajúcej z cieľových požiadaviek na maturitu z informatiky si väčšmi uvedomujeme nepostačujúci počet zadaní s propedeutickou funckiou a tých na osvojovanie novo zavedenej syntaxe jazyka. V časti funkcie sme identifikovali veľa divergentných úloh na úkor jednoduchších úloh na systemizáciu, ktoré by boli prístupné aj slabším žiakom. Na zaplátanie spomenutých nedostatkov v rozmanitosti do budúcna, sme predstavili metódu na zaraďovanie ďalších úloh. Metóda spočíva v zhodnotení náročnosti úlohy podľa jej dominantných didaktických funkcií, požadovanej poznávacej úrovne a prvkov jazyka vyskytujúcich sa vo vzorovom riešení programu. 

Na zjednotenie predstáv potenciálnych prispievajúcich spoluautorov sme vypracovali interpretačný popis pre didaktické funkcie s modelovými prípadmi ako podobné skupiny navzájom odlíšiť. Podľa stúpajúcej náročnosti sme dospeli k nasledovnému poradiu poradiu funkcií: prípravná, propedeutiká, osvojovacia, upevňovacia, opakovanie a systemizácia. Motivačné úlohy a úlohy na aktualizáciu predošlého učiva majú byť zaraďované priebežne. Navyše sme dosiahli zahrnutie medzipredmetových vzťahov v $56 \%$ úloh z celkového počtu. Aplikáciou sú zasadené do kontextu vyučovacích predmetov: matematika, fyzika, slovenský jazyk a litertúra, geografia, chémia, dejepis. 

Od úloh požadujeme šablónovitosť príbehového textu znenia zadania, ktoré sa člení do troch častí: scenéria, problém a pokyn. Podľa nami vytvorených úloh vyplýva, že úloha sú krátke texty priemerne $49 \pm 18$ slov, $4 \pm 1$ viet, s $10 \pm 2$ slovami na vetu a $33.9\% \pm 8\%$ slovami nad slabiky. Skóre čitateľnosti Gunning fog index je $18 \pm 3$ a ideálne sa má udržiavať pod hodnotou 20. Zbierka je náročnosťou textu prístupná žiakom stredných škôl s menšími výhradami. 

Grafickej úprave zbierky tiež predpisujeme jednotnú podobu s číslovaným názvom úlohy, textom znenia problémovej situácie, ukážkou súboru, vstupu a výstupu programu. V online podobe učebného textu sa pod špecifikáciu úlohy umiestňujú ešte navigačné hypertextové prepojenia. Učebný text je prístupný na webovej stránke a cez portál Github podporuje kolaboratívnu modifikáciu. Rozvrhnutie materiálu v online prostredí sa podriaďuje osvedčeným metódam na znižovanie vonkajšej kognitívnej záťaže ako je zjednodušovanie štruktúry obsahu a jeho segmentácia kapitoly.

Súčasná práca poskytuje možnosti na viaceré rozšírenia. Vzorové riešenia úloh by sa po vzore programového vyučovania mohli transformovať na postupnosť nápoveďových otázok a odpovedí v podobe kusov programového kódu. Ďalšie problémy sa naskýtajú v súvislosti s detailnejším overením intergácie novovytvorených úloh do zbierky podľa vypracovanej metódy. Porovnanie úspešnosti žiakov pri práci s papierovou verzus elektronickou učebnicou využívajúcou hypertext nám zas umožní lepšie porovnať prínosy týchto prezentačných médií.
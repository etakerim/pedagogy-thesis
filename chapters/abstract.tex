\thispagestyle{empty}
\section*{Abstrakt}
Záverečná práca sa venuje tvorbe učebníc so zameraním na organizáciu a na obsahovú štruktúru systému úloh podľa náročnosti, tak aby boli všestranne podporné didaktické funkcie učebného textu. Za týmto účelom sa predstavuje inovatívna zbierka úloh z programovania so vzorovými riešeniami v jazyku Python pre stredné školy vo všeobecno-vzdelávacom vyučovacom predmete informatika. Problémové cvičenia sú po vzore súťažných úloh zasadené do kontextu krátkych príbehov, ktoré motivujú žiaka k užitočnosti konkrétnych algoritmov a programov. Aplikačné oblasti sú volené s prihliadaním na rozvoj medzipredmetových vzťahov. Široká dostupnosť zbierky je podporená prezentáciou v hypertextovom priestore webu. Kolaboratívna rozšíriteľnosť sa umožňuje vypracovaním metódy stanovujúcej pevnú predlohu úlohy, požiadaviek na kvantitatívne vlastnosti jej znenia a zaradenie do zbierky podľa témy, funkcie a poznávacej úrovne. Výsledkom je sada problémových úloh v súlade so štátnym vzdelávacím programom a cieľovými požiadavkami na maturitnú skúšku, ktorá má aktivizovať žiakov v problémovom vyučovaní a dopomáha k samostatnej domácej príprave. \\

\textbf{Kľúčové slová:} problémové vyučovanie, učebnice, zbierka úloh, informatika, základy programovania

\emptypage

\thispagestyle{empty}
\section*{Abstract}
The final thesis is aimed at the textbook design with a focus on the organization and content structure of the system of problems according to difficulty so that the didactic functions of the teaching text are supported comprehensively. For this purpose, a collection of programming problems with sample solutions in the Python language is presented for secondary schools in the general education subject of computer science. Following the model of competition problems, problem exercises are set in the context of short stories, which motivate the student to the usefulness of specific algorithms and programs. Application areas are chosen to develop interdisciplinary relationships. The wide accessibility of the textbook is supported by presenting them in the hypertext environment of the web. Collaborative extensibility is made possible by developing a method establishing a fixed template of the problem, requirements for the quantitative properties of its wording, and inclusion in the textbook according to topic, function, and cognitive level.
The result is a set of problems in the scope given by the state educational program and the target requirements for the matura exam, which is supposed to activate students in problem-based teaching and help with individual home preparation. \\

\textbf{Keywords:} problem-based learning, textbooks, collection of problems, Computer science, Programming basics
\emptypage 
\chapter{Tvorba edukačných materiálov} 
Súčasný stav problematiky tvorby edukačných materiálov obsahuje zákonitosti stavby a funkcií učebnice, odporúčania pri písaní zrozumiteľných učebných textov a rozvrhovaní didakticky správneho systému úloh v zbierke. Tiež porovnáme doterajšie učebnice programovania pre stredné školy navzájom a vzhľadom na štátny vzdelávací štandard. 

\section{Učebnica}
Základným prameňom poznatkov a nositeľom obsahu vzdelávania je učebnica, ktorá patrí medzi čelných predstaviteľov pedagogických textov. Predstavuje jadro zoskupujúce okolo seba ostatné učebné prostriedky (\cite{zujev_ako_1986}). V medziach učebných osnov vymedzuje obsah základného učiva s doplnením o rozširujúce učivo, pričom rozsah osnovy učebnice nemusí byť totožný s učebnými osnovami (\cite{mlady_tvorba_1988}). 

Učebnica pomáha žiakom s osvojením si obsahu učiva, čím podporuje všetky súvisiace čiastkové činnosti: precvičovania, opakovania, systematizácie a integrácie. V edukačnom procese učebnica pôsobí aj výchovne, čím vplýva na formovanie postojov, motívov a záujmov. Odlišuje sa v tom od iných kníh a texov, pretože má priamu spätosť so získavaním a spracovaním faktov, pojmov a vzťahov žiakmi. Efektívne tak smeruje dosiahnutie výchovno-vzdelávacích cieľov vyučovacieho predmetu (\cite{gavora_ziak_1992}). Učebný program žiaka a vyučovací program učiteľa je ideálne v učebnici pochytený a odráža sa do scenáru učebného a vyučovacieho procesu. Hlvaná časť učebnice prestavuje súbor úloh určených na aktívne riešenie (\cite{pavlovkin_ziak_1989}).

Podľa školského zákona sa učebnica spolu s učebným textom a pracovným zošitom zaraďuje medzi edukačné publikácie. Na vzdelávanie sa používajú edukačné publikačné schválené ministerstvom školstva alebo zodpovedajúce princípom a cieľmi výchovy a vzdelávania (\cite{skolsky_zakon}). Princípy súvisiace s vlastnými vzdelávacími materiálmi sú v duchu rovnoprávnosti, rovnocennosti, zodpovednosti, tolerancie a vyváženého rozvoja osobnosti a zdokonaľovania vzdelávania podľa výsledkov výskumu a vývoja.

\subsection{Didaktické funkcie učebnice}
V učebnici pôsobí viacero naviazných a prelínajúcich sa vlastností vystupujúcich vo výchovno-vzdelávacom procese (\cite{zujev_ako_1986}), ktoré sú popísané bez stanoveného poradia dôležitosti:
\begin{itemize}
\itemsep0pt
\item \textbf{Informačná funkcia}: sa sústreďuje na stanovenie povinného rozsahu informácií pri štúdiu nevyhnutných na zapamätanie.
\item \textbf{Transformačná funkcia}: spočíva v didaktickom transfere poznatkov vedného odboru na obsah učiva v zrozumiteľnej a pútavej podobe, a zaroveň sa berie do úvahy na vekové a kultúrne osobitosti žiakov. Nabáda na výber vzdelávacích metód a uľahčuje aktivizáciu žiaka pri cvičeniach a úlohách prieskumného charakteru. Pri transformácii znalostí sa hľadí sa aj na potreby profesijného života a spoločenského očakávania od absolventov.
\item \textbf{Systematizačná funkcia}: pri objasňovaní učiva zabezpečuje následnosť poznatkov, postupný nárast náročnosti a vedie k metódam vedeckej systematizácie.
\item \textbf{Usmerňujúca funkcia}: slúži k upevňovaniu vedomostí napomáha v orientácii sa v nich a zapojením ich do praktických druhov činností. Vyžaduje sprevádazanie navrhnutými aktivitami pod vedením učiteľa.
\item \textbf{Motivačná funkcia}: pobáda túžbu a schopnosti žiakov na samostatné získavanie vedomostí.
\item \textbf{Integračná funkcia}: ucelene spája poznatky žiakov nadobudnuté z ich rozličných činností.
\item \textbf{Koordinačná funkcia}: zapája ku vzťahu k študovanému predmetu informácie z masovo-komunikačných prostriedkov. 
\item \textbf{Výchovná funkcia}: súčasne tiež rozvíjajúca funkcia, ktoré spočívajú v zladenom formovaní čŕt osobnosti žiaka.
\end{itemize}
Uvedené didaktické funkcie zabezpečujú komplexné pôsobenie učebnice na rozvoj kognitívnych a afektívnych schopností žiaka. Pri ich prepojení v učebnom texte dochádza nielen k nadobudnutiu nevyhnutných vedomostí a zručností na zvládnutie vyučovacieho predmetu, ale aj rozvoj kľúčových kompetencií a medzi nimi pripravenosti k všestrannejšiemu učeniu sa.


\subsection{Prvky učebnice}
Aby učebnica plnohodnotne napĺňala svoje mnohé poslania je poskladaná z \textbf{prvkov textového i mimotextového charakteru}, ktoré podnecujú aktívne kognitívne procesy a umožňujú zapojenie zvoleného učebného štýlu čitateľa. Zaradenie súčastí učebnice do členenia jej podsystémov nie je striktné, ale riadi sa dominantnou funkciou danej časti (\cite{zujev_ako_1986}).

Text v učebnici je súhrn viacerých viet, ktoré sú prostriedkom na odovzdanie informácií žiakom podľa komunikačného zámeru autora. Z povahy súvislého písaného jazykového prejavu sa vyznačuje kohéziou a koherenciou. Kohézia je súdržnosť textu na úrovni vzájomnej nadväznosti medzi vetami za použitia gramatických, lexikálnych a grafických jazykových prostriedkov. Najčastejšie sa uplatňujú gramatická zhoda medzi nadradeným a podradeným slovným druhom alebo vetným členom, opakovanie výrazu ďalej v texte, synonymá, a interpunkčné znamienka. Koherencia je zase tématická spojitosť textu, keď sa z hlavnej rozvíjajú vedľajšie myšlienky   (\cite{gavora_ziak_1992}). 

Podľa úlohy, ktorú zohráva text v predstavovaní učebnej látky sa rozlišuje \textbf{základný, doplňujúci a vysvetľujúci text}. Základný text je určený ako povinný na osvojenie pre úspešné zvládnutie problematiky. Podľa typu činností motivovanej základným textom sú povšimnuté teoretické poznávacie texty považované tiež za výkladovú zložku zastávajúcu informačnú funkciu vysvetľovania a komentovania nového učiva. Kdežto u inštrumentálno-praktických textov, alebo aj nevýkladovej zložky prevláda transformačná funkcia premietajúca sa do otázok, úloh a cvičení. Doplňujúci text prehlbuje rozsah učebných osnov dodatočňou argumentáciu vplývajúcej na rozumovú a emočnú stránku. Regulovanie poznávacej činnosti má na starosti vysvetľujúci text, ktorý dáva základný text do súvislostí (\cite{zujev_ako_1986}).

Mimotextové zložky nachádzajúce sa v učebnici sú zatrieďované na \textbf{aparát organizácie osvojovania, ilustračný materiál, a orientačný aparát}. Na organizáciu osvojovania slúžia prehľadové tabuľky, otázky a úlohy spolu s odpoveďami, ktoré v závislosti od kontextu sú spadajú do základného textu. Ilustrácie prevažne graficky dotvárajú textovú zložku, s ktorou sú vo vzťahovej rovine buď nadradenosti ako vedúce ilustrácie, rovnocennosti, alebo podradenosti ako doplnkové ilustrácie. Súvislosť medzi textom a ilustráciou sa spozná, podľa toho či text opisuje ilustráciu, vtedy je text podradený, alebo ilustrácia slúži na dokreslenie sitúcie. Typickými príkladmi ilustrácií sú obrázky, schémy, plány, diagramy, grafiky, mapy. Orientačný aparát slúži na zdôraznenie slovných spojení alebo myšlienok cez tlačové zvýraznenia a symbolické značenie, alebo opakuje prvky z hlavnej časti v zhutnenej podobe na navigáciu v knihe, v čom spočíva úloha napríklad predhovoru, obsahu, a registrov (\cite{zujev_ako_1986}).

Tradičná redakčná výroba učebnice dbá na ustálené metódy a organizovanú spoluprácu pre kontrolu správnosti obsahu po odbornej stránke a dohliada na gramatickú, pravopisnú a štylistickú úpravu (\cite{mlady_tvorba_1988}). Zasadzuje sa o koordináciu činnosti autorských kolektívov, tak aby umožnila zosúladiť všetky dôležité prvky učebnice najmä však súhru textovej a grafickej časti. Osvedčené pracovné postupy redakcie zabezpečujúce zahrnutie podstatných zložiek učebnice začínajú vypracovaním jej osnovy, príprave materiálov a podkladov, následne príprave rukopisu a obrazových predlôh v čase vymedzenom stanoveným harmonogramom, ktoré prechádzajú korektúrou, a celkové snaženie je zavŕšené vydaním učebnice a získaním doložky ministerstvom školstva podľa osobitého predpisu. Tvorba vzdelávacích materiálov priamo učiteľmi nie je až tak rigidná, ale navádza na zmysluplnú organizáciu práce.

\subsection{Multimediálne prostredie}
Postavenie výuky informatiky okolo počítačov a súvisiaceho prídavného vybavenia vedie k prirodzenej snahe uspôsobovať edukačné materiály naskýtajúcim sa podmienkam. Tým môžu učebné texty zúžikovať príležitosti pre obohatenie ich obsahu multimédiami a hypertextovými prepojeniami. Princípy uplatňujúce sa pri tvorbe klasických učebníc sa nevyhnutne prenášajú aj na tvorbe multimediálnych učebníc, pretože rovnako zostávajú publikáciami uspôsobených ku didaktickej komunikácii (\cite{krotky_nove_2015}).

Množstvo existujúcich učebníc prechádza do online prostredia zo svojej pôvodne knižnej úpravy na zvýšenie ich atraktívnosti a pohodlia pri prístupe k nim. Ani vznik učebnice určenej primárne pre elektronické médium však ešte nezaručuje využitie ponúkaného potenciálu na skĺbenie inovatívnych vyučovacích metód a ponúknutých technických vymožeností. Preto sa odlišujú učebnice podľa náročnosti prítomných konštrukcií na \textbf{jednoduché, komplexné a pokročilé učebnice} (\cite{krotky_nove_2015}).

Jednoduché učebnice sú elektronické obrazom svojich papierových vzorov bez uplatnenia akýkoľvek nových rozširujúcich možností. Komplexné učebnice získame zakomponovaním multimediálnych prvkov, zastúpených prevažne zvukmi, obrázkami, animáciami, videom, a vložením  hypertextových odkazov smerujúcich dovnútra vlastného obsahu a na externé webové portály a ďalší multimediálny obsah. Pokročilé učebnice navyše pozostvávajú s interaktívnych prvkov aktívne prispôsobujúcich tok informácii a manipulácie s nimi cez tlačidlá, posuvníky kontextové nápovedy, a príbuzné ovládanie. Nadstavbou pokročilej učebnice je edukačný softvér.

\textbf{Interaktívne personalizované úlohy} slúžiace na obohatenia osvojenia vedomostí z multimediálnej učebnice sa snažia o zníženie kognitívnej záťaže pri návrhu používateľských rozhraní, aby sa žiak mohol sústrediť výhradne na osvojované učivo skôr než na obtiaže s komplikovanými krokmi na dosiahnutie vytýčených zámerov vo virtuálnom priestore. Aplikované teórie súvisia s obmedzeniami kapacity krátkodobej pamäte. 

\textbf{Teória kogitivnej záťaže} odporúča elimovanie vonkajšej (\emph{extraneous}) kognitívnej záťaže cez zjednodušenie kompozície zobrazovaných prvkov. Vonkajšia kognitívna záťaž zaberá miesto vnútornej (\emph{intrisic}) a konceptuálnej (\emph{germane}) záťaže, ktoré sú potrebné na riešenie samotnej úlohy (\cite{uhercik_vyznam_2012}). \textbf{Teória duálneho kódovania} hovorí, že verbálne a neverbálne podnety sú v pamäti kódované zvlášť, čím sa zvýši počet položiek v krátkodobej pamäti pokiaľ pochádzajú z odlišného zdroja (\cite{mishra_interactive_2005}). 

\textbf{Pokyny pre multimediálne úlohy} nasledujú kognitívne princípy, ktoré sa odvíjajú od schopnosti človeka vstrebávať nové podnety. Multimediálny princíp tvrdí, že ku optimálnejšiemu učeniu dochádza keď pokyny obsahujú obrázky a texty spoločne ako samostatne. Princíp modality uprednostňuje zvukovú nahrávku a animáciu pred animáciou s textom. Príníp redundancie vylučuje nadbytočné elementy, ktorým je vložený text v prípade zvuku a animácie. Princíp súdržnosti je za vynechanie nadbytočných slov, obrázkov a zvukov. Signalizačný princíp vnáša do prostredia nápovedy na usmernenie pozornosti a orientácie sa. Spojitosť v čase a priestore navádza na radšej súčasné ako postupné zobrazenie súviaceho textu a obrázkov. Princíp segmentácie odporúča rozčleniť na animáciu so sprievodným slovom na kratšie učiacim sa kontrolované časti (\cite{mishra_interactive_2005}). 

\subsection{Skvalitňovanie učebného textu}
porozumenie textu je psycholingvistická činnosť. Taxomonómia porúch pri porozumení textu (\cite{gavora_ziak_1992})

Makroštruktúra  - tématické okruhy, kapitoly, texty, ich hlavné myšlienky (Štrukúra textu - liner, hierarch), Mikroštruktúra (\cite{pavlovkin_ziak_1989})

Skvalitnenie textov na úrovni mikroštruktúry, Kvantitativne charakteristiky textu (\cite{pavlovkin_ziak_1989})

metódy hodnotenia kvality učebníc. Odporúčania pri písaní textu. Hodnotenie zrozumiteľnosti textu učebnice (\cite{drahosova_hodnotenie_2014})

\section{Systém úloh v zbierke}

\subsection{Psychologické vychodiská}
 - Kedy žiak chce úlohy v učebnici riešiť?- Ako difrencovať úlohy vzhľadom na individuálne osobitosti žiakov? - Ako zoradiť úlohy v učebnici? (\cite{pavlovkin_ziak_1989})

\subsection{Klasifikácia vlastností úlohy}
Téma, podtéma, element (\cite{mindakova_tvorba_2008})
% Didaktické funkcie úloh}  
úlohy na motiváciu učebnopoznávacej činnosti žiakov, úlohy na aktualizáciu učiva, prípravné úlohy, úlohy na osvojenie definície pojmu, formulácie vety a postupu riešenia, úlohy na upevňovanie učiva, úlohy na aplikáciu učiva mimo informatiky, úlohy na aplikáciu učiva vo vnútri informatiky, úlohy propedeutického charakteru, úlohy na opakovanie a systemizáciu učiva (\cite{mindakova_tvorba_2008})

% Úroveň kogitívnych procesov}
1) vnímanie, pamäť, nižšie konvergentné procesy, vyššie konvergentné procesy, hodnotiace myslenie, tvorivé, divergentné myslenie \cite{mindakova_tvorba_2008}

\subsection{Preformulovávanie úloh}
Preformulovávanie úloh, 1. Zmena podmienky v zadaní úlohy, 2. Tvorba otočenej úlohy, 3. Zmena fabuly úlohy (\cite{mindakova_tvorba_2008})


\section{Vzdelávací štandard}
Učebné ciele \cite{noauthor_statny_2023} \cite{noauthor_statny_2023-1}

\subsection{Algoritmické myslenie}

\subsection{Existujúce učebnice}
Zbierka úloh pre 3. ročník gymnázia \cite{busek_zbierka_1987}
Matematika pre 3. ročník gymnázia \cite{sedivy_matematika_1986}

{abcPython} - {Pracovné} listy pre {Python} \cite{blaho_abcpython_2019}
Metodiky k pracovným listom pre {Python} \cite{blaho_metodiky_2019}
Programujeme v {Pythone}: učebnica informatiky pre stredné školy \cite{kucera_programujeme_2016}
Python a korytnačia grafika: \cite{meszarosova_python_2017}

Python tutoriál \cite{korman_python_2020}
Kuchárka {KSP} - {Korešpondenčný} seminár z programovania \cite{noauthor_kucharka_2022}

The {Nature} of {Code}: {Simulating} {Natural} {Systems} with {Processing} \cite{shiffman_nature_2012}
Idiots {Guide}: {Beginning} {Programming} \cite{talles_idiots_2014}

Skúsiš programovať?: {Basic} a strojový kód \cite{wattsova_skusis_1991}
Skúsiš to s mikropočítačom?: {Poznávame} a programujeme \cite{tatchellova_skusis_1990}

Algoritmy a programovanie v {Pascale}: nielen pre maturantov z predmetu informatika \cite{hedvigova_algoritmy_2020}



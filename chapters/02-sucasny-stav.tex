\chapter{Tvorba edukačných materiálov} 
Súčasný stav problematiky tvorby edukačných materiálov obsahuje zákonitosti stavby a funkcií učebnice, odporúčania pri písaní zrozumiteľných učebných textov a rozvrhovaní didakticky správneho systému úloh v zbierke. Tiež porovnáme doterajšie učebnice programovania pre stredné školy navzájom a vzhľadom na štátny vzdelávací štandard. 

\section{Učebnica}
Základným prameňom poznatkov a nositeľom obsahu vzdelávania je učebnica. Patrí medzi predstaviteľov pedagogických textov, ktoré podľa osnov vymedzuje rozsah základného učiva s doplnením o rozširujúce učivo. Podľa Zujeva je ,,učebnica jadrom okolo ktorého sa zoskupujú všetky ostatné učebné prostriedky'' (\cite{zujev_ako_1986}). Žiakom pomáha s osvojením si obsahu učiva, čím podporuje všetky súvisiace čiastkové činnosti: precvičovania, opakovania, systematizácie a integrácie. V edukačnom procese učebnica pôsobí aj výchovne, čím vplýva na formovanie postojov, motívov a záujmov. 

Učebnica sa odlišuje sa od iných kníh a texov priamou spätosťou so získavaním a spracovaním faktov, pojmov a vzťahov žiakmi. Slúži na efektívne dosiahnutie výchovno-vzdelávacích cieľov vyučovacieho predmetu (\cite{gavora_ziak_1992}). Učebný program žiaka a vyučovací program učiteľa je ideálne v učebnici pochytený a odráža sa do scenáru učebného a vyučovacieho procesu. Jadro učebnice prestavuje súbor úloh určených na aktívne riešenie (\cite{pavlovkin_ziak_1989}).

Podľa školského zákona sa učebnica spolu s učebným textom a pracovným zošitom zaraďuje medzi edukačné publikácie. Na vzdelávanie sa používajú edukačné publikačné schválené ministerstvom školstva alebo zodpovedajúce princípom a cieľmi výchovy a vzdelávania (\cite{skolsky_zakon}). Princípy súvisiace s vlastnými vzdelávacími materiálmi sú v duchu rovnoprávnosti, rovnocennosti, zodpovednosti, tolerancie a vyváženého rozvoja osobnosti a zdokonaľovania vzdelávania podľa výsledkov výskumu a vývoja.


\subsection{Didaktické funkcie učebnice}
V učebnici pôsobí viacero naviazných a prelínajúcich sa vlastností vystupujúcich vo výchovno-vzdelávacom procese (\cite{zujev_ako_1986}), ktoré sú popísané bez stanoveného poradia dôležitosti:
\begin{itemize}
\itemsep0pt
\item \textbf{Informačná funkcia}: sa sústreďuje na stanovenie povinného rozsahu informácií pri štúdiu nevyhnutných na zapamätanie.
\item \textbf{Transformačná funkcia}: spočíva v didaktickom transfere poznatkov vedného odboru na obsah učiva v zrozumiteľnej a pútavej podobe, pričom sa berie do úvahy na vekové a kultúrne osobitosti žiakov. Nabáda na výber vzdelávacích metód a uľahčuje aktivizáciu žiaka pri cvičeniach a úlohách prieskumného charakteru. Pri transformácii znalostí sa hľadí sa aj na potreby profesijného života a spoločenského očakávania od absolventov.
\item \textbf{Systematizačná funkcia}: pri objasňovaní učiva zabezpečuje následnosť poznatkov, postupný nárast náročnosti a vedie k metódam vedeckej systematizácie.
\item \textbf{Usmerňujúca funkcia}: slúži k upevňovaniu vedomostí napomáha v orientácii sa v nich a zapojením ich do praktických druhov činností. Vyžaduje sprevádazanie navrhnutými aktivitami pod vedením učiteľa.
\item \textbf{Motivačná funkcia}: pobáda túžbu a schopnosti žiakov na samostatné získavanie vedomostí.
\item \textbf{Integračná funkcia}: ucelene spája poznatky žiakov nadobudnuté z ich rozličných činností.
\item \textbf{Koordinačná funkcia}: zapája ku vzťahu k študovanému predmetu informácie z masovo-komunikačných prostriedkov. 
\item \textbf{Výchovná funkcia}: súčasne tiež rozvíjajúca funkcia, ktoré spočívajú v zladenom formovaní čŕt osobnosti žiaka.
\end{itemize}
Uvedené didaktické funkcie zabezpečujú komplexné pôsobenie učebnice na rozvoj kognitívnych a afektívnych schopností žiaka. Pri ich prepojení v učebnom texte dochádza nielen k nadobudnutiu nevyhnutných vedomostí a zručností na zvládnutie vyučovacieho predmetu, ale aj rozvoj kľúčových kompetencií a medzi nimi pripravenosti k všestrannejšiemu učeniu sa.


\subsection{Prvky učebnice}
Členenie systému učebnice na podsystémy - texty, mimotextové zložky (\cite{zujev_ako_1986})

Makroštruktúra \cite{pavlovkin_ziak_1989} - tématické okruhy, kapitoly, texty, ich hlavné myšlienky (Štrukúra textu - liner, hierarch), Mikroštruktúra

Časti učebnice - výkladová zložka, nevykladova \cite{gavora_ziak_1992}

(redakčný systém) zásady pri grafickej úprave, Vypracovanie osnovy učebnice \cite{mlady_tvorba_1988}

\subsection{Multimediálne prostredie}
\cite{krotky_nove_2015}

\subsection{Skvalitňovanie učebného textu}
porozumenie textu je psycholingvistická činnosť. Taxomonómia porúch pri porozumení textu \cite{gavora_ziak_1992}

Skvalitnenie textov na úrovni mikroštruktúry, Kvantitativne charakteristiky textu \cite{pavlovkin_ziak_1989}

metódy hodnotenia kvality učebníc. Odporúčania pri písaní textu. Hodnotenie zrozumiteľnosti textu učebnice \cite{drahosova_hodnotenie_2014}


\section{Systém úloh v zbierke}
Téma, podtéma, element \cite{mindakova_tvorba_2008}

\subsection{Psychologické vychodiská}
 - Kedy žiak chce úlohy v učebnici riešiť?- Ako difrencovať úlohy vzhľadom na individuálne osobitosti žiakov? - Ako zoradiť úlohy v učebnici?\cite{pavlovkin_ziak_1989}

\subsection{Didaktické funkcie úloh}
úlohy na motiváciu učebnopoznávacej činnosti žiakov, úlohy na aktualizáciu učiva, prípravné úlohy, úlohy na osvojenie definície pojmu, formulácie vety a postupu riešenia, úlohy na upevňovanie učiva, úlohy na aplikáciu učiva mimo informatiky, úlohy na aplikáciu učiva vo vnútri informatiky, úlohy propedeutického charakteru, úlohy na opakovanie a systemizáciu učiva \cite{mindakova_tvorba_2008}

\subsection{Úroveň kogitívnych procesov}
1) vnímanie, pamäť, nižšie konvergentné procesy, vyššie konvergentné procesy, hodnotiace myslenie, tvorivé, divergentné myslenie \cite{mindakova_tvorba_2008}

\subsection{Preformulovávanie úloh}
Preformulovávanie úloh, 1. Zmena podmienky v zadaní úlohy, 2. Tvorba otočenej úlohy, 3. Zmena fabuly úlohy \cite{mindakova_tvorba_2008}


\section{Vzdelávací štandard}
Učebné ciele \cite{noauthor_statny_2023} \cite{noauthor_statny_2023-1}

\subsection{Algoritmické myslenie}

\subsection{Existujúce učebnice}
Zbierka úloh pre 3. ročník gymnázia \cite{busek_zbierka_1987}
Matematika pre 3. ročník gymnázia \cite{sedivy_matematika_1986}

{abcPython} - {Pracovné} listy pre {Python} \cite{blaho_abcpython_2019}
Metodiky k pracovným listom pre {Python} \cite{blaho_metodiky_2019}
Programovanie v {Pythone} \cite{blaho_programovanie_2016}
Programujeme v {Pythone}: učebnica informatiky pre stredné školy \cite{kucera_programujeme_2016}
Python a korytnačia grafika: \cite{meszarosova_python_2017}

Python tutoriál \cite{korman_python_2020}
Kuchárka {KSP} - {Korešpondenčný} seminár z programovania \cite{noauthor_kucharka_2022}

The {Nature} of {Code}: {Simulating} {Natural} {Systems} with {Processing} \cite{shiffman_nature_2012}
Idiots {Guide}: {Beginning} {Programming} \cite{talles_idiots_2014}

Skúsiš programovať?: {Basic} a strojový kód \cite{wattsova_skusis_1991}
Skúsiš to s mikropočítačom?: {Poznávame} a programujeme \cite{tatchellova_skusis_1990}

Algoritmy a programovanie v {Pascale}: nielen pre maturantov z predmetu informatika \cite{hedvigova_algoritmy_2020}



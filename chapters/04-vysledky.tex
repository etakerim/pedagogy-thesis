\chapter{Výsledky práce}
%TODO


Výsledky (vlastné postoje alebo vlastné riešenie vecných problémov), ku ktorým autor dospel, sa musia logicky usporiadať a pri popisovaní sa musia dostatočne zhodnotiť. Zároveň sa komentujú všetky skutočnosti a poznatky v konfrontácii s výsledkami iných autorov. 

\section{Zbierka úloh}
medzipredmetové vzťahy: matematika, fyzika, finančná gramotnosť, chémia, dejepis, geografia

zoradené podľa obtiažnosti: 

1. Premenné - 11 úloh
2. Podmienky - 9 úloh
3. Cykly - 9 úloh
4. Náhodné čísla - 3 úlohy
5. Zoznamy - 9 úloh
6. Súbory - 6 úloh
7. Funkcie - 11 úloh

\subsection{Premenné}
\textbf{Premenná} je ako krabička slúžiaca na odkladanie informácií, ktoré si potrebujeme pre vykonanie danej činnosti zapamätať. Podľa účelu sa líšia svojim *dátovým typom*, ktorý sa vytvorí, keď do premennej niečo vložíme (*priradenie*) a určuje to, čo sa vo vnútri nachádza.

Základné stavebné kamene, z ktorých vyskladáme opis zložitejších javov sú:

\begin{itemize}
\itemsep0pt
\item \textbf{Logická hodnota} (\textit{bool}) - Boolean môže mať len dve hodnoty - pravda (\textit{True}) alebo nepravda (\textit{False})
\item \textbf{Celé číslo} (\textit{int}) - Do integer-u ukladáme ľubovolné kladné a záporné celé čísla (napr. \textit{97})
\item \textbf{Desatinné číslo} (\textit{float}) - Líšia sa od celých čísel spôsobom uloženia (napr. \textit{3.14159})
\item \textbf{Reťazec} (\textit{str}) - Označujeme ich úvodzovkami alebo apostrofmi a väčšinou predstavujú text napísaný na klávesnici alebo zobrazený na obrazovke. (napr. \textit{"Učím sa programovať!"})
\end{itemize}

\subsubsection*{1. Pozdrav}
Vytvor program, ktorý ťa po vložení mena pozdraví. Zameň pozdrav a zároveň nechaj program sa rozlučiť.

\begin{code}
Ako sa voláš?: ______
Ahoj ______
\end{code}

\subsubsection*{2. Básnik}
Vytváraš básničky na počkanie. Dnes sa ti ťažko premýšľa nad kreatívnymi textami, tak si chceš ušetriť námahu tým, že budeš meniť len rým.

\begin{code}
Napíš slovo, ktoré sa rýmuje so slovom strach: _____
Tu je báseň:
Z počítačov mával som vždy strach
teraz som však šťastný ako _____.
\end{code}

\subsubsection*{3. Pozvánka}
Každému kamarátovi chceš poslať pozvánku na svoju narodeninovú oslavu. Okrem mena v správe potrebuješ meniť aj čas konania oslavy (nie všetci chodia načas), vec, ktorú priniesie a jedlo, ktoré bude mať prichystané.

\begin{code}
Meno kamaráta: _____
Čas oslavy: _____
Prines: _____
Jedlo: ______

Ahoj _____,
pozývam ťa na moju narodeninovú oslavu, ktorá sa bude konať 12.4. o _____. Nezabudni priniesť _____ a pekný darček. Na večeru ťa čaká _____ a samozrejme lahodná torta. Teším sa na teba! :)
\end{code}

\subsubsection*{4. Prevod jednotiek teploty}
Prišiel si na návštevu v Amerike a keď ideš von nevieš ako sa máš obliecť, lebo na teplomere vidíš len stupne Fahrenheita ($F$). Premeň ich na stupne Celzia ($C$).

$$ C = (F - 32) \cdot (5 / 9) $$

\begin{code}
Vonku je °F: _____
Doma by to bolo _____°C.
\end{code}

\subsubsection*{5. Hlboká roklina}
Stojíš na útese nad hlbokým údolím a rozmýšlaš ako odmerať jej hĺbku (*h*). Vtom ťa osvietia tvoje dávne vedomosti z fyziky. Zoberieš do ruky kameň a pustíš ho z ruky do rokliny.
Zároveň spustíš stopky a zmeriaš čas dopadu. Rýchlosť zvuku rachotu pri náraze na zem môžeme zanedbať. Na kameň sa ohybuje sa nadol rovnomerným spomaleným pohybom. Pôsobí naň tiažové zrýchlenie: $g = 9.81$.
$$ h = (gt^2)\;/\; 2 $$

\begin{code}
Čas dopadu kameňa (s): ________
Hĺbka rokliny je potom _______ metrov.
\end{code}

\subsubsection*{6. Vedro s vodou}
Do nádrže z dažďovou vodou napršalo cez noc veľa vody. Jediný spôsob ako zúžitkovať zachytenú vodu je preniesť ju vo vedre valcového tvaru. Naberieme vždy len toľko vody koľko budeme potrebovať, preto je dobré poznať objem vedra. Rozmery vedra dokážeme odmerať pravítkom. Objem valcového vedra $V$ s výškou $v$ a priemerom podstavy $d$ sa vypočíta ako:
$$ V = \pi \cdot (d\;/\;2)^2 \cdot v $$

\begin{code}
Výška vedra (cm): ________
Priemer dna (cm): ________

Do vedra sa zmestí _______ litrov vody.
\end{code}

\subsubsection*{7. Cesta autom}
Plánuješ trasu na výlet autom a chceš zistiť akou rýchlosťou musíte priemerne ísť, aby ste stihli navštíviť všetky miesta a prišli večer včas do hotela.

\begin{code}
Dĺžka cesty (km): ____
Odchod z domu (hodina): ____
Príchod do hotela (hodina): ____

Pôjdete priemerne ____ km/h.
\end{code}

\subsubsection*{8. Kúpalisko}
Začína sa letná sezóna a prevádzka kúpaliska musí pred otvorením plne napustiť bazény v areáli. Všetky sú kvádrového tvaru a poznáme ich rozmery. Zaujíma nás spotrebovaná voda na konkrétny bazén a cena, ktorú za ňu zaplatíme.

\begin{code}
Dĺžka bazéna (m): ____
Šírka bazéna (m): ____
Hĺbka bazéna (m): ____
Hĺbka hladiny od okraja (cm): ____
Cena za m^3 vody v eurách: _____

Na bazén sa minie ____ litrov vody a bude to stáť ____ eur.
\end{code}

\subsubsection*{9. Maľovanie}
Sťahuješ sa s rodičmi do nového bytu a dali ti za úlohu vymalovať si izbu. Myslíš si, že nástroj na rýchle počítanie množstva farby by sa hodil aj profesionálnym maliarom, preto vytvoríš program na vypočítanie plochy stien a stropu bez okna a podlahy.

\begin{code}
Rozmery miestnosti
Dĺžka (cm): ____
Šírka (cm): ____
Výška (cm): ____
Rozmery okna
Šírka (cm): ____
Výška (cm): ____
Výdatnosť farby (m^2/kg): ____

Maľovať budeš plochu ____ m^2. Kúp ____ kg farby.
\end{code}

\subsubsection*{11. Pokladnička}
Do banky vložíme peniaze (vklad)  a každý rok sa nám na nich pripočítava úrok. V banke necháme peniaze určitý počet rokov. Vypočítajte ako sumu dostaneme pri výbere.

Napíš program, ktorý nám umožní na vstupe napísať rôzne sumy peňazí, úrokové miery a obdobie sporenia a na výstupe vypíše nasporenú sumu. Vyberte si, či použijete jednoduché alebo zložené úročenie - vzorec nájdite na internete alebo v zošite matematiky. Spustenie programu môže vyzerať nasledovne (pri jednoduchom úročení):

\begin{code}
Vklad (euro): ____
Úrok (%): ___
Dĺžka sporenia (rok): ___

Na konci sporenia si v banke vyberieš _____ eur.
\end{code}

\subsubsection*{12. Chemikálie}
Napíš program na výpočet zmiešavania dvoch roztokov. Každý roztok je opísaný svojou hmotnosťou ($m$) v gramoch a hmotnostným zlomkom rozpustenej látky v rozpúštadle ($w$) v percentách.
 
Roztoky vo vzorci sú rozlíšené dolným indexom, napr. . Na vstupe sú zadané všetky premenné s indexami 1 a 2. Premenné s indexom 3 je potrebné vypočítať v programe.  Percentá je potrebné prerátať na pomer z celku vydelením 
100-mi.

Na výpočet v programe využiješ rovnice:
$$m_3 = m_1 + m_2$$
$$m_3 \cdot w_3 = m_1 \cdot w_1 +  m_2 \cdot w_2$$

\begin{code}
m1 (hmotnosť roztoku č.1)? ___
w1 (hmotnostný zlomok roztoku č.1)? ___
m2 (hmotnosť roztoku č.2)? ___
w2 (hmotnostný zlomok roztoku č.2)? ___

Výsledný roztok má hmotnosť ___ g.
Hmotnostný zlomok rozpustenej látky je ___ %.
\end{code}


\subsubsection*{13. Brzdenie}
V poslednej dobe je na trati viacej nebezpečných zrážok. Rušňovodiči ťa požiadali, aby si zistil ako rýchlo pred prekážkou dokáže vlaková súprava zastaviť pri danej rýchlosti.

\begin{itemize}
\itemsep0pt
\item Kinetická energia pohybujúceho sa vlaku (práca potrebná na zabrzdenie): $ W = E_k = \frac{1}{2} \cdot m \cdot v^2 $
\item Brzdná dráha pri brzdnej sile $F_b$: $ s = \frac{W}{F_b \cdot m} $
\item Čas potrebný na zastavenie vlaku pri rovnomernom spmalenom pohybe: $ t = \sqrt{\frac{2 \cdot s}{F / m}} $
\end{itemize}

\begin{code}
Vlaková súprava
- Rýchlosť (km/h): ____
- Hmotnosť lokomotívy (t): ____
- Hmotnosť vagóna (t): ____
- Počet vagónov: ____
- Počet miest na vagón: ____
- Zaplnenosť vlaku (%): ____
- Brzdná sila (N/t): ____

V rýchlosti ____ km/h zabrzdí súprava s hmotnosťou ____ t na vzdialnosť _____ m a bude to trvať ____ s.
\end{code} 

\subsection{Podmienky}
\textbf{Podmienky} sú ako križovatky na ceste. Podľa toho kam chceme ísť, sa rozhodneme, ktorou cestou pôjdeme ďalej. Aby sme sa uistili, že máme ten správny smer (\emph{vetva podmienky}) pýtame sa vždy logickú otázku. Otázka používa údaje uložené v premenných.

\subsubsection*{1. Heslo}
Tvoj dom na strome už vykradlo pár nezvaných návštevníkov. Vymyslel si preto spôsob ako dovoliť návštevu len overeným osobám. Tie musia poznať tajné heslo. Napíš program, ktorý slovne privíta členov a odoženie zlodejov.

\begin{tabular}{@{}p{0.15\linewidth}p{0.75\linewidth}}
\textbf{\small Vstup:} &
\vspace{-3em}
\begin{code}
Stoj! Povedz Heslo!
? @\fbox{\phantom{vstup}}@
\end{code}
\end{tabular}

\vspace{-2em}
\begin{tabular}{@{}p{0.15\linewidth}p{0.75\linewidth}}
\textbf{\small Výstup:} &
\vspace{-3em}
\begin{code}
Vstúp, priateľ 
(alebo Zmizni kade ľahšie)
\end{code}
\end{tabular}
\vspace{-2em}

\subsubsection*{2. Najväčšie číslo}
Na lúke sa hrajú šípky. Hrači si zapisujú dosiahnuté skóre na tabuľu. Dnes proti sebe hrali v partii traja protihráči. Napíš program, ktorý označí hráča s najväčším získaným počtom bodov.

\begin{tabular}{@{}p{0.15\linewidth}p{0.75\linewidth}}
\textbf{\small Vstup:} &
\vspace{-3em}
\begin{code}
1.skóre: @\fbox{\phantom{vstup}}@
2.skóre: @\fbox{\phantom{vstup}}@
3.skóre: @\fbox{\phantom{vstup}}@
\end{code}
\end{tabular}

\vspace{-2em}
\begin{tabular}{@{}p{0.15\linewidth}p{0.75\linewidth}}
\textbf{\small Výstup:} &
\vspace{-3em}
\begin{code}
Najväčie skóre @\fbox{\phantom{vstup}}@ bodov má @\fbox{\phantom{vstup}}@ hráč
\end{code}
\end{tabular}
\vspace{-2em}


\subsubsection*{3. Vhodné oblečenie}
Módni poradcovia vyšli z módy a ich prácu prebrali počítače. Na základe počasia a príležitosti odporúčajú vhodný outfit. Vymysli pár tipov pre rôzne situácie a začni radiť.

\begin{tabular}{@{}p{0.15\linewidth}p{0.75\linewidth}}
\textbf{\small Vstup:} &
\vspace{-3em}
\begin{code}
Ako je vonku?: @\fbox{\phantom{vstup}}@
Kam ideš?: @\fbox{\phantom{vstup}}@
\end{code}
\end{tabular}

\vspace{-2em}
\begin{tabular}{@{}p{0.15\linewidth}p{0.75\linewidth}}
\textbf{\small Výstup:} &
\vspace{-3em}
\begin{code}
Určite si nezabudni @\fbox{\phantom{vstup}}@ a tiež si vezmi @\fbox{\phantom{vstup}}@.
\end{code}
\end{tabular}
\vspace{-2em}

\subsubsection*{4. Morský vánok}
Kapitán plachetnice na otvorenom oceáne musí mať vždy prehľad odkiaľ fúka vietor, aby dokormidloval do vytúženého cieľa. Príliš silné závany vetra môžu byť nebezpečné pre posádku. Rozthať polámať lodné sťažne, potrhať plachty, či zaplaviť palubu. 

Cez rádio dostáva plavidlo každý deň správy o predpovedi sily vetra v Beafortovej stupnici. Sila vetra je ňou vyjadrená do dvanástich stupňov od bezvetria až po orkán. Napíšte program, ktorý kapitánu vysvetlí stupeň vetra. Podľa stupnice určíme jeho pomenovanie, rýchlosti v námorných uzloch a očakávateľnej výšky vĺn.

\begin{tabular}{@{}p{0.15\linewidth}p{0.75\linewidth}}
\textbf{\small Vstup:} &
\vspace{-3em}
\begin{code}
Sila vetra na Beaufortovej stupnici: @\fbox{\phantom{12}}@
\end{code}
\end{tabular}

\vspace{-2em}
\begin{tabular}{@{}p{0.15\linewidth}p{0.75\linewidth}}
\textbf{\small Výstup:} &
\vspace{-3em}
\begin{code}
Vietor sa nazýva @\fbox{\phantom{vstup}}@.
Vietor má rýchlosť @\fbox{\phantom{vstup}}@ kt.
Očakávaná výška vĺn je @\fbox{\phantom{vstup}}@ m.
\end{code}
\end{tabular}
\vspace{-2em}


\subsubsection*{5. Pokazený rozpis}
Továreň na železnú rudu dostala nový časový rozpis vylepšeného technologického procesu. Spracovanie zvyčajne trvá dlhšie ako hodinu. Nehodí sa im teda mať časy napísané iba v minútach. Rozpíš programom minúty na dni, hodiny, minúty pre jednoduchšie čítanie rozpisu. Vynechaj nepotrebné časové údaje.

\begin{tabular}{@{}p{0.15\linewidth}p{0.75\linewidth}}
\textbf{\small Vstup:} &
\vspace{-3em}
\begin{code}
Trvanie (min.): @\fbox{\phantom{vstup}}@
\end{code}
\end{tabular}

\vspace{-2em}
\begin{tabular}{@{}p{0.15\linewidth}p{0.75\linewidth}}
\textbf{\small Výstup:} &
\vspace{-3em}
\begin{code}
= @\fbox{\phantom{vstup}}@ d. @\fbox{\phantom{vstup}}@ hod. @\fbox{\phantom{vstup}}@ min.
\end{code}
\end{tabular}
\vspace{-2em}


\subsubsection*{6. Hovoriaca kalkulačka}
Výpočty neboli nikdy väčšia zábava. Teda aspoň s kalkulačkou, ktorá namiesto čudných matematických čmáraníc hovorí ľudskou rečou. Vytvor program pre kalkulačku, ktorá si vypýta dve čísla. Tie bude ich vedieť sčítať alebo odčítať podľa slovného pokynu.

\begin{tabular}{@{}p{0.15\linewidth}p{0.75\linewidth}}
\textbf{\small Vstup:} &
\vspace{-3em}
\begin{code}
Som hovorica kalkulačka a rada počítam!
Povedz mi prvé číslo: @\fbox{\phantom{vstup}}@
Potrebujem ďašie číslo: @\fbox{\phantom{vstup}}@
Chceš ich sčítať alebo odčítať: @\fbox{\phantom{vstup}}@ (sčítať alebo odčítať)
\end{code}
\end{tabular}

\vspace{-2em}
\begin{tabular}{@{}p{0.15\linewidth}p{0.75\linewidth}}
\textbf{\small Výstup:} &
\vspace{-3em}
\begin{code}
Výsledok tvojho príkladu: @\fbox{\phantom{vstup}}@ (plus alebo mínus) @\fbox{\phantom{vstup}}@ je @\fbox{\phantom{vstup}}@.
\end{code}
\end{tabular}
\vspace{-2em}

\subsubsection*{7. Chaos v lístkoch}
Vyznať sa v linkách mestskej hromadnej dopravy si vyžaduje dlhoročné skúsenosti. Treba oplývať aj riadnou dávkou trpezlivosti. Ľahko sa nám stane, že omylom nasadneme do autobusu a hneď sa vydáme na okružnú jazdu po siedmich divoch sídliska. Horší zážitok je stretnutie revízora po zistení, že máme nesprávny lístok alebo že nemáme žiaden ...

Postávaš pri automate na lístky a nevieš sa vysomáriť z množstva časov a zón v ponuke. Napíš program, ktorý podľa počtu zónu a trvania ceny vypíše cenu zľavneného lístka. Nájdi na internete aktuálnu tarifu MHD v tvojom meste.

\begin{tabular}{@{}p{0.15\linewidth}p{0.75\linewidth}}
\textbf{\small Vstup:} &
\vspace{-3em}
\begin{code}
Popíš mi svoju cestu s MHD
Koľko zón prejdeš?: @\fbox{\phantom{vstup}}@
Koľko minút má trvať cesta?: @\fbox{\phantom{vstup}}@
\end{code}
\end{tabular}

\vspace{-2em}
\begin{tabular}{@{}p{0.15\linewidth}p{0.75\linewidth}}
\textbf{\small Výstup:} &
\vspace{-3em}
\begin{code}
Zlavnený lístok stojí @\fbox{\phantom{vstup}}@ eur.
\end{code}
\end{tabular}
\vspace{-2em}


\subsubsection*{8. Kvadratická rovnica}
Matematika v škole dokáže byť poriadna otrava. Hlavne, keď od rána do večera nič iné nerobíš ako počítaš príklady na kvadratické rovnice. ,,Načo mám ten počítač'', pomyslíš si večer vo svetle stolenj lampy. Pre zadané koeficienty $a$, $b$, $c$ predpisu $ax^2 + bx + c = 0$ napíš program, ktorý vypočíta jej korene a vrchol paraboly.

\begin{tabular}{@{}p{0.15\linewidth}p{0.75\linewidth}}
\textbf{\small Vstup:} &
\vspace{-3em}
\begin{code}
Koeficienty kvadratickej rovnice:
a = @\fbox{\phantom{vstup}}@
b = @\fbox{\phantom{vstup}}@
c = @\fbox{\phantom{vstup}}@
\end{code}
\end{tabular}

\vspace{-2em}
\begin{tabular}{@{}p{0.15\linewidth}p{0.75\linewidth}}
\textbf{\small Výstup:} &
\vspace{-3em}
\begin{code}
@\fbox{\phantom{a}}@x^2 + @\fbox{\phantom{b}}@x + @\fbox{\phantom{c}}@ = 0
x1 = @\fbox{\phantom{abc}}@
x2 = @\fbox{\phantom{abc}}@
V[@\fbox{\phantom{abc}}@; @\fbox{\phantom{abc}}@]
\end{code}
\end{tabular}
\vspace{-2em}


\subsubsection*{9. Trojuholníky}
Trojuholník je mýtická bytosť, o ktorej je vždy treba zistiť. Nesmieme použiť pravítko, lebo to by nás čakala príliš jednoduchá výzva. Veď bez rysovania zístíme o tejto trojcípej paráde všeličo. Hoci aj keď jej chýbajú niektoré rozmery.

\begin{enumerate}[label=\alph*)]
\item Ak je to možné, doplň chýbajúce informácie pre ľubovoľný trojuholník (zadaný ako SSS) ako sú dĺžky strán a výšok, veľkosti uhlov, obsah a obvod. Využi trojuholníkovú nerovnosť, sínusovú vetu, kosínusovú vetu a vzorec na výpočet obsahu trojuholníkov.
\item Rozšír program aj pre ostatné vety o trojuholníkoch: SUS, USU, UUS
\end{enumerate}

\begin{tabular}{@{}p{0.15\linewidth}p{0.75\linewidth}}
\textbf{\small Vstup:} &
\vspace{-3em}
\begin{code}
Zadajte strany ľubovolného trojuholníka:
a = @\fbox{\phantom{vstup}}@
b = @\fbox{\phantom{vstup}}@
c = @\fbox{\phantom{vstup}}@
\end{code}
\end{tabular}

\vspace{-2em}
\begin{tabular}{@{}p{0.15\linewidth}p{0.75\linewidth}}
\textbf{\small Výstup:} &
\vspace{-3em}
\begin{code}
Strany: a = @\fbox{\phantom{abc}}@; b = @\fbox{\phantom{abc}}@; c = ___
Uhly: alpha = @\fbox{\phantom{abc}}@°; beta = @\fbox{\phantom{abc}}@°; gamma = @\fbox{\phantom{vstup}}@°
Výšky: v(a) = @\fbox{\phantom{abc}}@; v(b) = @\fbox{\phantom{abc}}@; v(c) = @\fbox{\phantom{abc}}@
O = @\fbox{\phantom{abc}}@
S = @\fbox{\phantom{abc}}@
Trojuholník je: @\fbox{\phantom{abc}}@, @\fbox{\phantom{abc}}@
\end{code}
\end{tabular}
\vspace{-2em}

\subsection{Cykly}
Obrovský potenciál počítačov tkvie v bezchybnom neúnavnom vykonávaní presne zadaných inštrukcií. \underline{\textbf{Cykly}} umožňujú opakovať rovnaký postup ľubovoľný počet krát a tým efektívne odstraňovať rutinnú prácu.


\subsubsection*{1. 100-krát napíš}
Za vyrušovanie na hodinách sa stalo populárnym trestom ručné prepisovanie mravoučnej vety stokrát. Stalo sa to tak neznesiteľné, že si zhotovil robota na pomoc záškodníkom. Chýbajú mu len príkazy, čo má vlastne robiť.

\begin{tabular}{@{}p{0.15\linewidth}p{0.75\linewidth}}
\textbf{\small Vstup:} &
\vspace{-3em}
\begin{code}
Musím napísať: @\fbox{\phantom{vstup}}@
Toľkoto krát: @\fbox{\phantom{vstup}}@
\end{code}
\end{tabular}

\vspace{-2em}
\begin{tabular}{@{}p{0.15\linewidth}p{0.75\linewidth}}
\textbf{\small Výstup:} &
\vspace{-3em}
\begin{code}
@\fbox{\phantom{vstup}}@
@\fbox{\phantom{vstup}}@
@\fbox{\phantom{vstup}}@
...
\end{code}
\end{tabular}
\vspace{-2em}


\subsubsection*{2. Hodnotenie}
Filmoví a gastonomickí kritici zavŕšia namáhavý deň udelením číselného skóre k ich recenziam. Pre lepší efekt v časopise potrebujú nakresliť hviezdničky namiesto čísla. Pomôž im programom.

\begin{tabular}{@{}p{0.15\linewidth}p{0.75\linewidth}}
\textbf{\small Vstup:} &
\vspace{-3em}
\begin{code}
Skóre: @\fbox{5}@
\end{code}
\end{tabular}

\vspace{-2em}
\begin{tabular}{@{}p{0.15\linewidth}p{0.75\linewidth}}
\textbf{\small Výstup:} &
\vspace{-3em}
\begin{code}
@\fbox{\textit{*****}}@
\end{code}
\end{tabular}
\vspace{-2em}


\subsubsection*{3. Pyramída}
Hviezdičky zoskup do tvaru pyramídy zadanej výšky.

\begin{tabular}{@{}p{0.15\linewidth}p{0.75\linewidth}}
\textbf{\small Vstup:} &
\vspace{-3em}
\begin{code}
Výška pyramídy: @\fbox{4}@
\end{code}
\end{tabular}

\vspace{-2em}
\begin{tabular}{@{}p{0.15\linewidth}p{0.75\linewidth}}
\textbf{\small Výstup:} &
\vspace{-3em}
\begin{code}
   *
  ***
 *****
*******
\end{code}
\end{tabular}
\vspace{-2em}

\subsubsection*{4. Smaragd}
Na pyramídu pripoj zo spodu ďaľšiu obrátene, aby vznikol smaragd z hviezdičiek.

\begin{tabular}{@{}p{0.15\linewidth}p{0.75\linewidth}}
\textbf{\small Vstup:} &
\vspace{-3em}
\begin{code}
Veľkosť smaragdu: @\fbox{5}@
\end{code}
\end{tabular}

\vspace{-2em}
\begin{tabular}{@{}p{0.15\linewidth}p{0.75\linewidth}}
\textbf{\small Výstup:} &
\vspace{-3em}
\begin{code}
  *
 ***
*****
 ***
  *
\end{code}
\end{tabular}
\vspace{-2em}

\subsubsection*{5. Duté vnútro}
Nakresli duté pyramídu a smaragd podľa prechádzajúcich úloh.

\begin{tabular}{@{}p{0.15\linewidth}p{0.75\linewidth}}
\textbf{\small Vstup:} &
\vspace{-3em}
\begin{code}
Výška pyramídy: @\fbox{4}@
\end{code}
\end{tabular}

\vspace{-2em}
\begin{tabular}{@{}p{0.15\linewidth}p{0.75\linewidth}}
\textbf{\small Výstup:} &
\vspace{-3em}
\begin{code}
    *
   * *
  *   *
 *******
\end{code}
\end{tabular}
\vspace{-2em}


\subsubsection*{6. Mriežka slov}
Načítaj veľkosť tabuľky a slovo, ktoré sa v nej bude na každom riadku v stĺpci opakovať.

\begin{tabular}{@{}p{0.15\linewidth}p{0.75\linewidth}}
\textbf{\small Vstup:} &
\vspace{-3em}
\begin{code}
Počet riakov a stĺpcov: @\fbox{4}@
Opakovať slovo: @\fbox{ano}@
\end{code}
\end{tabular}

\vspace{-2em}
\begin{tabular}{@{}p{0.15\linewidth}p{0.75\linewidth}}
\textbf{\small Výstup:} &
\vspace{-3em}
\begin{code}
ano ano ano ano
ano ano ano ano
ano ano ano ano
ano ano ano ano
\end{code}
\end{tabular}
\vspace{-2em}


\subsubsection*{7. Rám}
Prvý a posledný riadok a stĺpec bude tvoriť rám pre mriežku slov.

\begin{tabular}{@{}p{0.15\linewidth}p{0.75\linewidth}}
\textbf{\small Vstup:} &
\vspace{-3em}
\begin{code}
Počet riakov a stĺpcov: @\fbox{4}@
Opakovať slovo: @\fbox{ano}@
\end{code}
\end{tabular}

\vspace{-2em}
\begin{tabular}{@{}p{0.15\linewidth}p{0.75\linewidth}}
\textbf{\small Výstup:} &
\vspace{-3em}
\begin{code}
### ### ### ###
### ano ano ###
### ano ano ###
### ### ### ###
\end{code}
\end{tabular}
\vspace{-2em}


\subsubsection*{8. Malá násobilka}
K výbave každého žiaka základnej školy patrí tabuľky malej násobilky. Vytvor takúto tabuľku obsahujúcu každý násobok od 1x1 po 10x10, aby si pomohol všetkým malým matematikom.


\vspace{-2em}
\begin{tabular}{@{}p{0.15\linewidth}p{0.75\linewidth}}
\textbf{\small Výstup:} &
\vspace{-3em}
\begin{code}
   1   2   3   4   5   6   7   8   9  10
   2   4   6   8  10  12  14  16  18  20
   3   6   9  12  15  18  21  24  27  30
   4   8  12  16  20  24  28  32  36  40
   5  10  15  20  25  30  35  40  45  50
   6  12  18  24  30  36  42  48  54  60
   7  14  21  28  35  42  49  56  63  70
   8  16  24  32  40  48  56  64  72  80
   9  18  27  36  45  54  63  72  81  90
  10  20  30  40  50  60  70  80  90 100
\end{code}
\end{tabular}
\vspace{-2em}


\subsubsection*{9. Sporenie}
Na letnej brigáde si zarobil peniaze, ktoré chceš usporiť. Porovnáš ponuky bánk a hľadáš najvýhodnejší plán. Vytvor si sporiacu kalkulačku, ktorá na základe nemenného počiatčného vkladu, ročnej úrokovej sadzby, typu úročenia a žiadanej konečnej sumy, vypíše vývoj tvojich finančných prostriedkov do budúcnosti.

\begin{tabular}{@{}p{0.15\linewidth}p{0.75\linewidth}}
\textbf{\small Vstup:} &
\vspace{-3em}
\begin{code}
Počiatočný vklad v eurách: @\fbox{\phantom{vstup}}@
Úroková sadzba p.a. v %: @\fbox{\phantom{vstup}}@
Typ úročenia (jednoduché / zložené): @\fbox{\phantom{vstup}}@
Žiadaná suma v eurách: @\fbox{\phantom{vstup}}@
\end{code}
\end{tabular}

\vspace{-2em}
\begin{tabular}{@{}p{0.15\linewidth}p{0.75\linewidth}}
\textbf{\small Výstup:} &
\vspace{-3em}
\begin{code}
Rok      Suma						Úrok
  1.		@\fbox{\phantom{vstup}}@ Eur	@\fbox{\phantom{vstup}}@ Eur
  2.		@\fbox{\phantom{vstup}}@ Eur	@\fbox{\phantom{vstup}}@ Eur
\end{code}
\end{tabular}
\vspace{-2em}

%
\subsection{Náhodné čísla}
Pri tvorbe simulácií sú náhodné čísla nepostrádateľné. Umožňujú vniesť variabilitu a rôznorodosť do inak statických scén. Nesmierne poslúžia v hrách, kde dovoľujú modelovať napríklad pravdepodobnosť výskytu monštier, či pokladov.


\subsubsection*{1. Hádzanie kockou}
Vytvorte simuláciu hodu kockou. Po stlačení klávesy Enter sa nakreslí kocka s padnutým číslom.

\begin{code}
HOĎ<ENTER>
+-------+
| #   # |
|   #   |
| #   # |
+-------+
\end{code}

\subsubsection*{2. Hádaj číslo}
Náhodne vyber číslo s rozsahu medzi 0 a 100 a nechaj hráča hádať dokým neuhádne. Pri tom mu poskytni nápovedy, či je jeho tip priveľa alebo primalo. Zakomponuj rôzne obtiažnosti s možnosťou nastavenia rozsahu alebo maximálnym počtom tipov.

\begin{code}
Hádaj číslo: 8
Málo
Hádaj číslo: 18
Veľa
Hádaj číslo: 13
Výborne. Uhádol si!
\end{code}

\subsubsection*{3. Opakovanie násobilky}
Vďaka tvojej tabuľke malej násobilky sa malý školáci mohli naučiť násobiť. Ako dobre to vedia, musíš teraz odtestovať. Vygeneruj dve čísla od 1 do 10 do príkladu na násobenie. Over správnosť žiačikovej odpovede.

\begin{code}
Koľko je ____ x _____?
= _____
Správne - len tak ďalej / Nesprávne - hádaj znovu
Chceš ďalší príklad?
\end{code}
 

%\subsection{Reťazce a zoznamy}
\underline{\textbf{Zoznam}} (tiež aj \underline{\textbf{Pole}}) je množina údajov zaznamenaných spolu pod jedným menom. Každý údaj poľa sa nazýva \underline{\textbf{prvok}} a poradie jeho pozície sa nazýva \underline{\textbf{index}}. \underline{\textbf{Reťazce}} sa správajú podobne ako zoznamy, ale ich prvkami sú jednotlivé \underline{\textbf{znaky}}.

\subsubsection*{1. Vymeň písmeno}
Niekto ti posiela správy s diakritikou, ale po ceste sa vždy prekrúti jedno písmeno. Texty obsahujú aj pekné básne, ktoré si chceš vytlačiť a pripnúť na nástenku. Pokazený znak však kazí celkový dojem z diela. Zameň zadané chybné písmeno v celom reťazci.

\begin{tabular}{@{}p{0.15\linewidth}p{0.75\linewidth}}
\textbf{\small Vstup:} &
\vspace{-3em}
\begin{code}
Správa: @\fbox{\phantom{dlhý text}}@
Za chybné písmeno: @\fbox{\phantom{a}}@
Vymeň: @\fbox{\phantom{b}}@
\end{code}
\end{tabular}

\vspace{-2em}
\begin{tabular}{@{}p{0.15\linewidth}p{0.75\linewidth}}
\textbf{\small Výstup:} &
\vspace{-3em}
\begin{code}
Opravené!
@\fbox{\phantom{dlhý text}}@
\end{code}
\end{tabular}
\vspace{-2em}


\subsubsection*{2. Cenzúra}
Prišla tvrdá cenzúra s nariadením, že nikto už nesmie vidieť žiadnu samohlásku. Nahraď každý priestupok vo vstupnom texte iným špeciálnym znakom.

\begin{tabular}{@{}p{0.15\linewidth}p{0.75\linewidth}}
\textbf{\small Vstup:} &
\vspace{-3em}
\begin{code}
Správa: @\fbox{Ja som tvoj kamarat}@
Samohlásku nahraď: @\fbox{*}@
\end{code}
\end{tabular}

\vspace{-2em}
\begin{tabular}{@{}p{0.15\linewidth}p{0.75\linewidth}}
\textbf{\small Výstup:} &
\vspace{-3em}
\begin{code}
Cenzurované: @\fbox{J* s*m tv*j k*m*r*t}@
\end{code}
\end{tabular}
\vspace{-2em}


\subsubsection*{3. Počítanie slov}
Do redakcie miestnych novín chodia dennodenne články, vtipy, poviedky a príbehy zo života od verných čitateľov. Aby mohli byť uverejnené potrebujú sa zmestiť do vyhradeného priestoru. Vypíš počet znakov, slov, viet a normostrán (=\emph{1800 znakov}), aby sa príhody rýchlejšie rozšírili medzi ľudí.

\begin{tabular}{@{}p{0.15\linewidth}p{0.75\linewidth}}
\textbf{\small Vstup:} &
\vspace{-3em}
\begin{code}
Článok: @\fbox{\phantom{Dlhý text článku s veľa slovami}}@
\end{code}
\end{tabular}

\vspace{-2em}
\begin{tabular}{@{}p{0.15\linewidth}p{0.75\linewidth}}
\textbf{\small Výstup:} &
\vspace{-3em}
\begin{code}
Znaky: @\fbox{\phantom{123}}@
Slová: @\fbox{\phantom{123}}@
Vety: @\fbox{\phantom{123}}@
Normostrany: @\fbox{\phantom{123}}@
\end{code}
\end{tabular}
\vspace{-2em}


\subsubsection*{4. Najdlhšie slovo}
Debatný spolok usporiadal súťaž o nájdenie najdlhšieho slova, ktoré sa kedy vyskytlo v historických prejavoch. Zaujali ťa odmeny, ale nechce sa ti prehrabávať knižnicou starých záznamníkov. Prácu si preto uľahčíš. Nájdi najdlhšie slovo v ľubovoľnom reťazci.

\begin{tabular}{@{}p{0.15\linewidth}p{0.75\linewidth}}
\textbf{\small Vstup:} &
\vspace{-3em}
\begin{code}
Rečnícky prejav: @\fbox{\phantom{Dlhý text článku s veľa slovami}}@
\end{code}
\end{tabular}

\vspace{-2em}
\begin{tabular}{@{}p{0.15\linewidth}p{0.75\linewidth}}
\textbf{\small Výstup:} &
\vspace{-3em}
\begin{code}
Najdlhšie slovo v ňom: @\fbox{\phantom{slovo}}@
\end{code}
\end{tabular}
\vspace{-2em}

\subsubsection*{5. Výskyt písmen}
Dlho do noci čítaš časopisy o umelej inteligencii a fascinuje ťa jej schopnosť rozprávať sa s človekom. Na vytvorenie viet na danú tému potrebuje mať prehľad o percentuálnom výskyte hlások v texte. Spočítaj a vypíš zoznam početnosti písmen v reťazci.

\begin{tabular}{@{}p{0.15\linewidth}p{0.75\linewidth}}
\textbf{\small Vstup:} &
\vspace{-3em}
\begin{code}
Článok: @\fbox{\phantom{Dlhý text článku s veľa slovami}}@
\end{code}
\end{tabular}

\vspace{-2em}
\begin{tabular}{@{}p{0.15\linewidth}p{0.75\linewidth}}
\textbf{\small Výstup:} &
\vspace{-3em}
\begin{code}
A: @\fbox{23.2}@ %
B: @\fbox{11.5}@ %
C: @\fbox{8.9}@ %
...
Z: @\fbox{0.3}@ %
\end{code}
\end{tabular}
\vspace{-2em}


\subsubsection*{6. Histogram}
Počas predošlého pokusu s početnosťou písmen si všimneš, že každé ďalšie písmeno v zozname sa objavuje oveľa menej než očakávaš. Vykresli hviezdičky namiesto počtu percent. Over si tak svoje pozorovanie graficky.

\begin{tabular}{@{}p{0.15\linewidth}p{0.75\linewidth}}
\textbf{\small Vstup:} &
\vspace{-3em}
\begin{code}
Článok: @\fbox{\phantom{Dlhý text článku s veľa slovami}}@
\end{code}
\end{tabular}

\vspace{-2em}
\begin{tabular}{@{}p{0.15\linewidth}p{0.75\linewidth}}
\textbf{\small Výstup:} &
\vspace{-3em}
\begin{code}
A: @\fbox{****}@
E: @\fbox{*******}@
I: @\fbox{****}@
...
X: @\fbox{*}@
\end{code}
\end{tabular}
\vspace{-2em}


\subsubsection*{7. Nákupný košík}
Na veľkých nákupoch sa často zíde prehľadný zoznam s tým, čo doma treba. Pýtaj si položky s ich cenami až kým sa nerozhodneš, že máš spísané všetko. Zobraz prehľadnú orámovanú tabuľku s údajmi, podobne ako na pokladničnom bločku. To sú názov tovaru, DPH tovaru, cena tovaru s DPH a cena spolu za nákup.

\begin{tabular}{@{}p{0.15\linewidth}p{0.75\linewidth}}
\textbf{\small Vstup:} &
\vspace{-3em}
\begin{code}
Čo kúpiť?: @\fbox{\phantom{vstup}}@
Cena @\fbox{\phantom{vstup}}@?: @\fbox{\phantom{vstup}}@
\end{code}
\end{tabular}

\vspace{-2em}
\begin{tabular}{@{}p{0.15\linewidth}p{0.75\linewidth}}
\textbf{\small Výstup:} &
\vspace{-3em}
\begin{code}
+----------+--------+--------------+
| Tovar    |  DPH   |  Cena s DPH  |
+----------+--------+--------------+
| Chlieb   |  0,20  |      0,98    |
+----------+--------+--------------+
|    ...   |  ...   |     ...      |
+----------+--------+--------------+
| CELKOM   |  0,20  |      0,98    |
+----------+--------+--------------+
\end{code}
\end{tabular}
\vspace{-2em}

\subsubsection*{8. Akronym}
SMS-ky rapídne zdraželi. Napadlo ti, že bude lepšie posielať slovné spojenia ako skratky. Zo zadaných slov vytvor akronym, ktorý vznikne ponechaním len začiatočných písmen každého slova.

\begin{tabular}{@{}p{0.15\linewidth}p{0.75\linewidth}}
\textbf{\small Vstup:} &
\vspace{-3em}
\begin{code}
Slovné spojenie: @\fbox{Slovenské národné divadlo}@
\end{code}
\end{tabular}

\vspace{-2em}
\begin{tabular}{@{}p{0.15\linewidth}p{0.75\linewidth}}
\textbf{\small Výstup:} &
\vspace{-3em}
\begin{code}
Skratka: @\fbox{SND}@
\end{code}
\end{tabular}
\vspace{-2em}


\subsubsection*{9. Veľa opakovania}
Roboti rozvážajú pizzu po meste. Popri tom si zapisujú zmenu smeru pre postupné vylepšovanie trás k častým zákazníkom. Keďže sa firme darí, nachodili roboti toho už riadne veľa. Všetky záznamy o ich cestách sa im ani nezmestia do pamäti. Všimneš si, že si značia každý jeden krok, čiže sa často opakujú. Nahraď postupnosť za sebou idúceho písmena, písmenom a počtom jeho výskytov.

\begin{tabular}{@{}p{0.15\linewidth}p{0.75\linewidth}}
\textbf{\small Vstup:} &
\vspace{-3em}
\begin{code}
Cesta robota: @\fbox{NNNNNNSSSSSSSSSSSWWWWNNN}@
\end{code}
\end{tabular}

\vspace{-2em}
\begin{tabular}{@{}p{0.15\linewidth}p{0.75\linewidth}}
\textbf{\small Výstup:} &
\vspace{-3em}
\begin{code}
Skomprimované: @\fbox{6N11S4W3N}@
\end{code}
\end{tabular}
\vspace{-2em}

%\subsection{Súbory}
\underline{\textbf{Súbor}} je zoskupením súvisiacich údajov, ktoré sú uložené na disku počítača. Oproti načítavaniu vstupu z klávesnice majú výhodu hlavne pri spracovaní a uchovaní veľkého množstva dát. Súbory sa dajú: \underline{vytvoriť} alebo \underline{vymazať}, \underline{otvoriť} alebo \underline{zatvoriť}, \underline{čítať} alebo \underline{zapisovať}.

Podľa typu uchovávaných údajov (označované \underline{\textbf{príponou}}), súbory rozdeľujeme na:
\begin{itemize}[noitemsep]
\item \textbf{Textové súbory} - .txt, .csv, .html, .py
\item \textbf{Obrazové súbory} - .bmp, .png, .jpg, .gif, .svg
\item \textbf{Zvukové súbory} - .wav, .mp3, .midi
\item \textbf{Video súbory} - .avi, .mp4, .mkv
\item \textbf{Spustiteľné súbory} - .exe
\end{itemize}
V tejto kapitole budeme pre jednoduchosť pracovať s textovými súbormi.

\subsubsection*{1. Prepisovanie}
Príde ti zbytočne prepisovať dlhé články na vstup programu a vždy sa pomýliš. Načítaj články pre každú úlohu z predošlej kapitoly zo súboru. Uprav programy tak, aby si najprv vypýtali názov súboru. V úlohe ,,veľa opakovania'' ulož záznam o ceste robota do nového súboru.


\subsubsection*{2. Turistika}
Na víkend sa črtajú ideálne podmienky na horskú turistiku. Nenecháš nič na náhodu a pripravíš si detailný plán s výškovým profilom trasy. Na každých desať metrov trasy si do súboru poznačíš nadmorskú výšku z mapy. Zisti celkové stúpanie a klesanie počas celého výletu spolu s najvyššou a najnižšou nadmorskou výškou. Vypíš aj celkovú dĺžku túry v kilometroch a trvanie prechodu horami v hodinách.

\begin{tabular}{@{}p{0.2\linewidth}p{0.7\linewidth}}
\textbf{\small Obsah súboru:} &
\vspace{-3em}
\begin{code}
348
351
379
384
395
401
396
\end{code}
\end{tabular}

\vspace{-2em}
\begin{tabular}{@{}p{0.2\linewidth}p{0.7\linewidth}}
\textbf{\small Vstup:} &
\vspace{-3em}
\begin{code}
Trasa je v súbore s názvom: @\fbox{\phantom{vstup}}@
\end{code}
\end{tabular}

\vspace{-2em}
\begin{tabular}{@{}p{0.2\linewidth}p{0.7\linewidth}}
\textbf{\small Výstup:} &
\vspace{-3em}
\begin{code}
Trasa: @\fbox{0.140 km}@ - @\fbox{0}@ h @\fbox{21}@ min
Stúpanie: @\fbox{53}@ m
Klesanie: @\fbox{40}@ m
Najnižšie miesto trasy: @\fbox{361}@ m
Najvyššie miesto trasy: @\fbox{401}@ m
\end{code}
\end{tabular}
\vspace{-2em}


\subsubsection*{3. Vedomostný kvíz}
Bifľovanie ti vôbec nepríde prínosné. Keby existoval spôsob, akým si opakovanie učiva spríjemniť. Včera si zo smútku nad vidinou takto premárneného času, pri jedení čokolády a čipsov, pozeral kvízovú reláciu. Prišlo ti to neuveriteľne poučné. Polož náhodnú otázku s možnostami zo súboru kvízových otázok a bodovo ohodnoť správnu odpoveď. Všetky kvízové otázky s možnosťami sa však nezmestia do pamäti programu. Náhodnu otázku vyber priamo zo súboru.

\begin{tabular}{@{}p{0.2\linewidth}p{0.7\linewidth}}
\textbf{\small Obsah súboru:} &
\vspace{-3em}
\begin{code}
Otázka: V ktorom roku začala Francúzska revolúcia?
  A: 1763
  B: 1813
  C: 1789
  D: 1654
Odpoveď: C
Otázka: Al2O3 je?
  A: hydroxid vápenatý
  B: oxid hlinitý
  C: hydroxid sodný
Odpoveď: B
\end{code}
\end{tabular}

\vspace{-2em}
\begin{tabular}{@{}p{0.2\linewidth}p{0.7\linewidth}}
\textbf{\small Kvíz:} &
\vspace{-3em}
\begin{code}
Súbor s kvízovými otázkami: @\fbox{kviz.txt}@
Kvízové otázky pripravené. Ideme na to!
V ktorom roku sa začala Francúzska revolúcia?
A: 1763
B: 1813
C: 1789
D: 1654
Aká je správna odpoveď?: @\fbox{C}@
Správne! Máš 1 bodov. 
(alebo) Nabudúce si to lepšie premysli.
\end{code}
\end{tabular}
\vspace{-2em}


\subsubsection*{4. Narodeniny}
Darčeky k narodeninám zvykneš kupovať na poslednú chvílu. Potrebuješ mať prehľad aspoň na mesiac dopredu, kto bude mať narodeniny, aby si stihol vybrať niečo výnimočné. Zo súboru načítaj ľudí, ktorí majú sviatok v požadovaný mesiac v roku.

\begin{tabular}{@{}p{0.2\linewidth}p{0.7\linewidth}}
\textbf{\small Obsah súboru:} &
\vspace{-3em}
\begin{code}
Jožko Mrkvička, 15.3.2002
Katka Krátka, 2.7.1993
Martinko Klingáč, 12.11.1995
Iveta Novotná, 27.2.2001
\end{code}
\end{tabular}

\vspace{-2em}
\begin{tabular}{@{}p{0.2\linewidth}p{0.7\linewidth}}
\textbf{\small Vstup:} &
\vspace{-3em}
\begin{code}
Zobraz narodeniny pre mesiac v roku: @\fbox{3.2019}@
\end{code}
\end{tabular}

\vspace{-2em}
\begin{tabular}{@{}p{0.2\linewidth}p{0.7\linewidth}}
\textbf{\small Výstup:} &
\vspace{-3em}
\begin{code}
Narodeniny: @\fbox{Marec 2019}@
@\fbox{15.3. - Jožko Mrkvička - 17 rokov}@
\end{code}
\end{tabular}
\vspace{-2em}


\subsubsection*{5. Cestovné poriadky}
Z celoštátneho rýchlika prestupujú cestujúci v okresných mestách na miestne autobusy. Podľa času odchodu a trvania cesty zisti, ktorý autobus stihnú. Vypíš najbližší spoj s najmenším čakaním medzi vlakom a autobusom. Daj pozor! Prvý časový údaj v riadku s odchodom autobusu je trvanie cesty vlakom  do stanice, odkiaľ odchádza ten autobus.

\begin{tabular}{@{}p{0.2\linewidth}p{0.7\linewidth}}
\textbf{\small Obsah súboru:} &
\vspace{-3em}
\begin{code}
vlak,9:15,10:45,12:15,14:30,16:15,18:20
bus,1:00,11:00,13:00,15:00,17:00
bus,1:45,9:30,12:08,16:33
\end{code}
\end{tabular}

\vspace{-2em}
\begin{tabular}{@{}p{0.2\linewidth}p{0.7\linewidth}}
\textbf{\small Vstup:} &
\vspace{-3em}
\begin{code}
Čas: @\fbox{10:00}@
Trvanie cesty vlakom: @\fbox{1:00}@
\end{code}
\end{tabular}

\vspace{-2em}
\begin{tabular}{@{}p{0.2\linewidth}p{0.7\linewidth}}
\textbf{\small Výstup:} &
\vspace{-3em}
\begin{code}
Najbližší spoje (vlak, autobus):
@\fbox{12:15 - 13:15, 15:00 -}@
\end{code}
\end{tabular}
\vspace{-2em}

%
\subsection{Funkcie}
\textbf{Funkcia} je pomenovaná časť programu, ktorá vykonáva špecifickú činnosť. Hovorí sa im preto tiež \textit{procedúry} alebo \textit{podprogramy}. Predstavuje súvislú časť kód, obsahujúcu sled na seba nadväzujúcich príkazov, tvoriacich jeden logický celok.  Takto umožňuje zložitejší program rozdeliť na viacero samostatných častí.


\subsubsection*{1. Vraky}
V šírich vodách Atlantiku sa stále ukrýka nepreberné bohatstvo vo vrakoch potopených lodí. V tejto minhre bude tvojou úlohou odkryť tajomstvo skrývajúce sa pod hladinou, nájdením parníku vytvoreného na náhodnej pozícii. Do programu napíš funkciu \verb|vzdialenost(x, y)|, ktorá na základe zadaných súradníc vypočíta ako ďaleko si od vraku.

\begin{code}
Sonar hlási potopený parník na dohľad!
Tvoje súradnice?: ___,___
Od vraku si _____ námorných míľ.
...
Našiel si vrak. Dobrá práca!
\end{code}


\subsubsection*{2. Cézarová šifra}
Pri tvojich cestách po lodných pokladoch ťa odpočúvajú piráti, ktorí ťa chcú predbehnúť a obohatiť sa. Na utajenie svojej polohy a správ s pevninou musíš svoje informácie šifrovať. 

Funkcia \verb|sifruj(sprava, kluc)| zašifruje text správy tak, že posunie každé písmeno abecedy podľa písmena \verb|kluc|, čiže napríklad správa \emph{"ABC"} sa kľúčom \emph{"B"} zmení na \emph{"BCD"}. 

Funkcia \verb|desifruj(sifra, kluc)| bude fungovať spätne.  Pre lepšiu bezpečnosť podporuj aj dlhšie kľúče. 

Každé písmeno bude vyjadrovať posun od začiatku abecedy písmena, s ktorým sa stretne. Potom správa \emph{"AVE CEZAR"} s kľúčom \emph{"BCD"} bude \emph{"BXH DGCBT"}.


\subsubsection*{3. Pascalov trojuholník}
Vytvor funkciu \verb|pascalov_trojuholnik(n)|, ktorá vypíšte súčtovú pyramídu s $n$ riadkami, ktorá má po okrajoch jednotky a nasledujúce riadky sa tvoria ako súčet dvoch čísel v predchádzajúcom riadku.

\begin{code}
Počet riadkov: 5

    1
   1 1
  1 2 1
 1 3 3 1
1 4 6 4 1
\end{code}

\subsubsection*{4. Štatistika}
Pre investora je dôležité poznať podmienky trhu a potenciálnu konkurenciu predtým, než si naplánuje stratégiu investovania. Rozbiehaš realitnú kanceláriu a skôr než nastaviš ceny pre konkrétne byty, zisti v akom vzťahu je výmera bytu k jeho cene v lokalite. Pre každú štatistickú funkciu si napíš zodpovedajúcu procedúru. Údaje o bytoch načítaj zo súboru.

\begin{code}
Súbor s bytmi v lokalite: ______

                    :   Cena (E)  :   Výmera(m^2) :
Priemer             :             :               :
Medián              :             :               :
Modus               :             :               :
Smerodajná odchýlka :             :               :
\end{code}


\subsubsection*{5. Lietadlo}
Pilotov v kokpite lietadlo by počas letu zaujímalo, ako ďaleko sú ešte od prístatia. Zo zemepisných súradníc aktuálnej polohy a súradníc cieľa vypočíataj vo funkcii `letime(x, y)` najkratšiu vzdialenosť medzi týmito bodmi na sférickom povrchu zemegule (\textit{Ortodróma}).
$$ \theta = \arccos(\sin(90° - \alpha_1) \cdot \sin(90° - \alpha_2) + \cos(90° - \alpha_1) \cdot \cos(90° - \alpha_2) \cdot \cos(|\beta_1 - \beta_2|)) $$
$$ d = (r \cdot \theta \cdot \pi)\;/\;180 $$

\begin{code}
Pozícia: 42.990967 -71.463767
Cieľ: 48.53682 -13.855231

Vzdialenosť: 4416.21 km
\end{code}


\subsubsection*{5. Bublikové triedenie}
Pre prehľadnosť údajov je užitočné vedieť ich utriediť podľa rôznych kritérií. Napíš program, ktorý vypíše študentov zo súboru zoradených podľa zadaného názvu stĺpčeka vzostupne.  Na začiatok použi algoritmus bublinkového triedenia, neskôr proces zefektívni využitím algoritmom triedenia zlučovaním alebo rýchlym triedením.

\paragraph{Obsah súboru (ziaci.csv):}

\begin{code}
meno, priezvisko, vek, datum narodenia, bydlisko, priemer, trieda
Milan, Peterka, 15, 2004-09-18, Bratislava, 1.6, I.B.
...
\end{code}


\subsubsection*{7. Rímske čísla}
Od archeológov si dostal dlhý zoznam rímskych čísel, ktoré boli nájdené v novobjavených podzemených historických pamiatkach. Tažko sa v nich dá vyznať a je na tebe, aby si ich premenil na "normálne" arabské čísla. Pre zhrnutie ti poslali aj zoznam pravidiel prevodu týchto číselných systémov. Napíš funkciu \verb|rimske_na_arabske(rimske)|, ktorá premení rímske na arabské číslo.

\begin{code}
I = 1
V = 5
X = 10
L = 50
C = 100
D = 500
M = 1000
\end{code}


\subsubsection*{8. Základný tvar zlomku}
Zlomky sú vhodné na presné výpočty s častami z celku. Vytvor jednoduchú kalkulačku, ktorá umožňuje dva zlomky sčítať, odčítať, násobiť a deliť. Výsledok vždy zjednoduš na základný tvar (\emph{Euklidov algoritmus pre NSD a NSN}).

\begin{code}
Kalkulačka zlomkov
a = 3/4
b = 1/2
Vypočítaj (+, -, *, /): +

Výsledok:
3/4 + 1/2 = 5/4
\end{code}


\subsubsection*{9. Hra Poklad}
Povráva sa, že na strašidelnom hrade v Karpatoch je bludisko so siedmimi tajomnými komnatami. Každá má meno a je v nej truhlica s pokladom. Mapa bludiska je náhodne poskladaná, uložená v pamäti počítača, ale nie je nakreslená na obrazovke. Hráč musí zistiť, ako sú komnaty navzájom pospájané. Na začiatku hry sa ocitne v náhodne vybranej komnate. Jeho úlohou je zhromaždiť všetky truhlice v jednej komnate, pričom môže vykonať iba ohraničený počet krokov.

\paragraph{Komnaty v mriežke s uloženým pokladom:}
\begin{enumerate}
\itemsep0pt
\item Purpurová a pekelná - Drahokamy
\item Červená a čudná - Žuvačky
\item Sivá a studená - Nanuky
\item Žltá a žeravá - Zlatky
\item Čierna a čarodejná - Smeti
\item Hnedá a hrozivá - Kalkulačky
\item Zelená a záhadná - Medeňáky
\end{enumerate}
   
\paragraph{Vzorová časť hrania hry:}

\begin{code}
Počítač rozumie týmto príkazom
S, V, J, Z   : Pohyb na sever, východ, juh, západ
ZDVIHNI		 : Zdvihne truhlicu
POLOZ		 : Položí truhlicu
KDE			 : Informuje o polohe truhlíc
SOS			 : Vypíše pravidlá hry

Si v 4.komnate
Je žltá a žeravá
Sú v nej: ZLATKY
Čo chceš robiť?
? ZDVIHNI
Zdvihol si truhlicu, v ktorej sú zlatky.

Ešte stále si 4.komnate
Čo chceš robiť?
? Z
...
\end{code}

\subsubsection*{10. Databáza} - Na školu za siedmimi horami a dolinami si objednali počítač na uloženie a prehliadanie záznamov o študentoch. Keďže rok, čo rok odchádzajú maturanti a prichádzajú prváci, potrebujú tabuľky i upravovať. Napíš databázový systém, ktorý bude umožňovať vytvárať a mazať tabuľky, kde každá bude vo vlastnom csv súbore. Budú sa dať vkladať a mazať aj riadky, či upravovať jednotlivé políčka. Ulož do databázy napríklad aj informácie o knihách zo školskej knižnice.

Pre nápady na rozšírenie pozri: \textbf{Postavme si databázu[EN]}: \url{https://cstack.github.io/db_tutorial/}

\paragraph{Ukážka možností systému:}

\begin{code}
DATABÁZA> NOVÁ TABUĽKA žiaci: meno, priezvisko, dátum narodenia
DATABÁZA> TABUĽKY
žiaci
DATABÁZA> OTVOR TABUĽKU žiaci
ŽIACI> VLOŽ Ružena, Kvetinková, 1998-11-15
ŽIACI> ZOBRAZ
   +----+---------+-------------+-----------------+
   | id |  meno   |  priezvisko | dátum narodenia |
   +----+---------+-------------+-----------------+
   | 1  |  Ružena | Kvetinková  |  1998-11-15     |
   +----+---------+-------------+-----------------+
ŽIACI> UPRAV 1 NASTAV priezvisko: Sedmokrásková
ŽIACI> ZOBRAZ: ZORAĎ PODĽA priezvisko
...
ŽIACI> ZOBRAZ: HĽADAJ PODĽA priezvisko: Sedmokrásková
...
ŽIACI> ZMAŽ 1
ŽIACI> ZMAŽ TABUĽKU žiaci
DATABÁZA> SKONČI
\end{code}


\subsubsection*{11. Kalkulačka}
Moderné vedecké kalkulačky sú takmer zázrakom. Buď tým, že sa mimo akademickej pôdy skoro vôbec nepoužívajú, alebo samotnou zložitosťou ich fungovania. Dokážu rozlíšiť, či má prednosť násobenie alebo sčítanie, zatiaľ čo vezmú do úvahy zátvorky. Nemôže byť pre nich nič jednoduchšie ako prijsť na to, čo je číslo a čo operátor v dlhom posuvnom texte displeja. Vytvor program kalkuačky, ktorá sa bude správať ako vrecková vedecká kalkulačka (s infixovým zápisom)(\textit{Algoritmus posunovacej stanice (Shunting yard algorithm)}).

\begin{code}
> 5 * (1589 - 2 * 74) / 2 + (33 * 8)
> 3866.5
> ...
\end{code} 



\section{Vzorové riešenia}

\section{Premenné}
\subsection{Pozdrav}
\begin{solution}
meno = input("Ako sa voláš?: ")
print("Ahoj ", meno)
print("Dovidenia ", meno)
\end{solution}

\subsection{Básnik}
\begin{solution}
slovo = input("Napíš slovo, ktoré sa rýmuje so slovom strach: ")
print("Tu je báseň:")
print("Z počítačov mával som vždy strach")
print("teraz som však šťastný ako", slovo, ".")
\end{solution}

\subsection{Pozvánka}
\begin{solution}
meno = input("Meno kamaráta: ")
cas = input("Čas oslavy: ")
vec = input("Prinesie okrem darčeku: ")

print(f"Ahoj {meno},")
print(f"pozývam ťa na moju narodeninovú párty.")
print(f"Bude sa konať 12.4. o {cas}.")
print("Nezabudni priniesť {vec} a pekný darček.")
print("Teším sa na teba! :)")
\end{solution}

\subsection{Teplota vo Farenheitoch}
\begin{solution}
f = input("Vonku je °F: ")
f = float(f)
c = (5 / 9) * (f - 32)
print(f"Doma by bolo na teplomeri {c:.2f}°C.")
\end{solution}

\subsection{Hlboká roklina}
\begin{solution}
g = 9.81
t = input("Čas do dopadu kameňa: ")
t = int(t)
h = (g * (t ** 2)) / 2
print("Hĺbka rokliny je", h, "metrov")
\end{solution}

\subsection{Vedro s vodou}
\begin{solution}
pi = 3.14159
v = input("Výška vedra (cm): ")
d = input("Priemer dna (cm): ")
v = int(v)
d = int(d)
V = pi * ((d / 2) ** 2)
V = V / 1000
print("Do vedra sa zmestí", V, "litrov vody.")
\end{solution}

\subsection{Cesta autom}
\begin{solution}
km = input("Dĺžka cesty (km): ")
odchod = input("Odchod z domu (hodina): ")
prichod = input("Príchod do hotela (hodina): ")

km = float(km)
odchod = int(odchod)
prichod = int(prichod)
hod = prichod - odchod

print(f"Auto pôjde priemernou rýchlosťou {km / hod:.2f} km/h.")
\end{solution}

\subsection{Kúpalisko}
\begin{solution}
dlzka = input("Dĺžka bazéna (m): ")
sirka = input("Šírka bazéna (m): ")
hlbka = input("Hĺbka bazéna (m): ")
okraj = input("Hĺbka hladiny od okraja (cm): ")
cena = input("Cena za m^3 vody v eurách: ")

dlzka = float(dlzka)
sirka = float(sirka)
hlbka = float(hlbka)
okraj = int(okraj)
cena = float(cena)
V = dlzka * sirka * (hlbka - (okraj / 100))
V *= 1000
cena = cena * V

print(f"Na bazén sa minie {V} litrov vody")
print(f"Voda bude to stáť {cena} eur.")
\end{solution}

\subsection{Maľovanie}
\begin{solution}
# Získaj z klávesnice rozmery miestnosti
print("Rozmery miestnosti")
dlzka = input("Dĺžka (cm): ")
sirka = input("Širka (cm): ")
vyska = input("Výška (cm): ")

# Premeň z písmen na čísla
dlzka = int(dlzka)
sirka = int(sirka)
vyska = int(vyska)

# Získaj z klávesnice rozmery okna a výdatnosť farby
print("Rozmery okna")
sirkaOkna = input("Širka (cm): ")
vyskaOkna = input("Výška (cm): ")
vydatnost = input("Výdatnosť farby (m^2/kg): ")

# Premeň z písmen na čísla
sirkaOkna = int(sirkaOkna)
vyskaOkna = int(sirkaOkna)
vydatnost = float(vydatnost)

# Spočítaj plochy stien, stropu a odpočítaj plochu okna
PlochaMiestnost = (dlzka * sirka) + 2 * (vyska * sirka) + 2 * (vyska * dlzka)
PlochaOkno = sirkaOkna * vyskaOkna
S = (PlochaMiestnost - PlochaOkno) / 10000
farbaKg = S / vydatnost

print(f"Maľovať budeš plochu {S:.2f} m2.")
print(f"Kúp {farbaKg:.2f} kg farby.")
\end{solution}

\subsection{Chemikálie}
\begin{solution}
m1 = int(input("Hmotnosť roztoku č.1 (m1)?"))
w1 = int(input("Hmotnostný zlomok roztoku č.1 (w1)?"))
m2 = int(input("Hmotnosť roztoku č.2 (m2)?"))
w2 = int(input("Hmotnostný zlomok roztoku č.2 (w2)?"))

m3 = m1 + m2
w3 = (m1 * w1 + m2 * w_2) / m3

print("Výsledný roztok má hmotnosť", m3, "g")
print("Hmotnostný zlomok rozpustenej látky je", w3 * 100, "%")
\end{solution}

\subsection{Brzdenie}
\begin{solution}
import math
print("Vlaková súprava")
v = int(input("- Rýchlosť (km/h): "))
lokomotiva = float(input("- Hmotnosť lokomotívy (t): "))
vagon = float(input("- Hmotnosť vagóna (t): "))
pocet_vagonov = int(input("- Počet vagónov: "))
F_b = int(input("- Brzdná sila (N/t): "))

# Premeň jednotky na základné SI
v /= 3.6
lokomotiva *= 1000
vagon *= 1000
F_b /= 1000

# Hmotnosť súpravy je hmotnosť lokomotívy a všetkých vagónov
m = lokomotiva + (pocet_vagonov * vagon)
# Vypočítaj celkovú kinetickú energiu, tá je rovnaká ako práca
# ktorú musia brzdy vykonať na zabrzdenie.
W = 0.5 * m * (v ** 2)
# Celková sila pôsobiaca proti pohybu vlaku
F = F_b * m
# Z definície práce W = F * s, vypočítaj dráhu potrebnú na zastavenie
s = W / F
# Vypočítaj čas potrebný na zastavenie pre rovnomerný spomalený pohyb
a = F / m
t = math.sqrt(2 * s / a)

print(f"Vlaková súprava má hmotnosť {int(m / 1000)} ton.")
print(f"V rýchlosti {int(v * 3.6)} km/h zabrzdí na vzdialenosť {int(s)} metrov.")
print(f"Brzdenie bude trvať {int(t)} sekúnd.")
\end{solution}
\section{Podmienky}

\subsection{Heslo}
\begin{solution}
print("Stoj! Povedz Heslo!")
pokus = input("? ")
if pokus == "tajne heslo":
    print("Vstúp, priateľ")
else:
    print("Zmizni kade ľahšie")
\end{solution}

\subsection{Najväčšie číslo}
\begin{solution}
x = input("1.skóre: ")
y = input("2.skóre: ")
z = input("3.skóre: ")
x = int(x)
y = int(y)
z = int(z)
najviac = x
poradie = 1
if y > najviac:
    najviac = y
    poradie = 2
if z > najviac:
    najviac = z
    poradie = 3

print(f"Najväčie skóre {najviac} bodov má {poradie} hráč.")
\end{solution}

\subsection{Vhodné oblečenie}

\begin{solution}
pocasie = input("Ako je vonku?: ")
miesto = input("Kam ideš?: ")

if pocasie == "slnečno"
	povinne = "šiltovka"
if pocasie == "zamračené":
	povinne = "mikina"
if počasie == "dážď":
	povinne = "vetrovka"
	
if miesto == "ihrisko":
	odporucanie = "tepláky"
if miesto == "škola"
	odporucanie = "košela"

print("Určite si nezabudni", povinne, "a tiež si vezmi", odporucanie, ".")
\end{solution}

\subsection{Morský vánok}
\begin{solution}
stupen = input("Sila vetra na Beaufortovej stupnici: ")
stupen = int(stupen)

if stupen == 0:
	nazov = "bezvetrie"
	rychlost = 0
	vlny = 0
elif stupen == 1:
	nazov = "vánok"
	rychlost = 2
	vlny = 0.1
elif stupen == 2:
	nazov = "slabý vietor"
	rychlost = 5
	vlny = 0.2
# Doplň ostatné stupne podľa Beafortovej stupnice

print(f"Vietor sa nazýva {nazov}.")
print(f"Vietor má rýchlosť {rychlost} kt.")
print(f"Očakávaná výška vĺn je {vlny} m.")
\end{solution}

\subsection{Pokazený rozpis}
\begin{solution}
min = input("Trvanie (min.): ")
min = int(min)
hod = min // 60
dni = hod // 24
hod -= dni * 24
min -= (hod * 60) + (dni * 24 * 60)

print("=", end=" ")
if dni > 0:
    print(f"{dni} d.", end=" ")
if hod > 0:
    print(f"{hod} hod.", end=" ")

print(f"{min} min.")
\end{solution}

\subsection{Hovoriaca kalkulačka}
\begin{solution}
print("Som hovorica kalkulačka a rada počítam!")
a = int(input("Povedz mi prvé číslo: "))
b = int(input("Potrebujem ďašie číslo: "))
cinnost = input("Chceš ich sčítať alebo odčítať: ")

if cinnost == "sčítať":
    print(f"Výsledok tvojho príkladu: {a} plus {b} je {a + b}")
elif cinnost == "odčítať":
    print(f"Výsledok tvojho príkladu: {a} mínus {b} je {a - b}")
else:
    print(f"Neviem čo znamená '{cinnost}'")
\end{solution}

\subsection{Chaos v lístkoch}
\begin{solution}
print("Popíš mi svoju cestu s MHD")
zony = int(input("Koľko zón prejdeš?:"))
minuty = int(input("Koľko minút má trvať cesta?:"))

if zony == 2 and minuty <= 30:
	cena = 0.55
elif zony == 3 and minuty <= 60:
	cena = 0.80
elif zony == 4 and minuty <= 60:
	cena = 1.00
elif zony == 5 and minuty <= 90:
	cena = 1.25
elif zony == 6 and minuty <= 90:
	cena = 1.50
elif zony == 7 and minuty <= 120:
	cena = 1.65

print("Zlavnený lístok stojí {cena:.2f} eur.")
\end{solution} 

\subsection{Kvadratická rovnica}
\begin{solution}
import math
print("Koeficienty kvadratickej rovnice:")
a = float(input("a = "))
b = float(input("b = "))
c = float(input("c = "))

if a == 0:
	print("Ide o lineárnu rovnicu")
else:
	print(f"{a:g}x^2 + {b:g}x + {c:g} = 0")
	D = b ** 2 - 4 * a * c
	if D < 0:
		print("Kvadratická rovnica nemá riešenie v R")
	elif D > 0:
		x1 = (-b - math.sqrt(D)) / (2 * a)
		x2 = (-b + math.sqrt(D)) / (2 * a)
		print(f"x1 = {x1}")
		print(f"x2 = {x2}")
	elif D == 0:
		x = -b / (2 * a)
		print(f"x = {x}")
		Vx = -b / (2 * a)
		Vy = c - ((b ** 2) / (4 * a))
		print(f"V[{Vx}; {Vy}]")
\end{solution}

\subsection{Trojuholníky}
\begin{solution}
import math

print("Zadajte strany ľubovolného trojuholníka:")
a = input("a = ")
b = input("b = ")
c = input("c = ")

a = float(a)
b = float(b)
c = float(c)

if a + b <= c:
	print("Pre trojuholník neplatí trojuholníková nerovnosť")
	print("a + b <= c")
	print(f"{a} + {b} <= {c}")
elif a + c <= b:
	print("Pre trojuholník neplatí trojuholníková nerovnosť")
	print("a + c <= b")
	print(f"{a} + {c} <= {b}")
elif b + c <= a:
	print("Pre trojuholník neplatí trojuholníková nerovnosť")
	print("b + c <= a")
	print(f"{b} + {c} <= {a}")
else:
	alpha = math.acos((a**2 - b**2 - c**2) / (-2*b*c))
	beta = math.acos((b**2 - a**2 - c**2) / (-2*a*c))
	gamma = math.acos((c**2 - a**2 - b**2) / (-2*a*b))

	va = c * math.sin(beta)
	vb = a * math.sin(gamma)
	vc = b * math.sin(alpha)

	alpha = math.degrees(alpha)
	beta = math.degrees(beta)
	gamma = math.degrees(gamma)

    print(f"\nStrany: a = {a}; b = {b}; c = {c}")
    print(f"Uhly: alpha = {alpha}°; beta = {beta}°; gamma = {gamma}°")
    print(f"Výšky: v(a) = {va}; v(b) = {vb}; v(c) = {vc}")
    print(f"O = {a + b + c}")
    print(f"S = {a * va * 0.5}")

        print("Trojuholník je:", end=" ")
        if a == b == c:
            print("Rovnostranný", end=", ")
        elif a == b or b == c or c == a:
            print("Rovnoramenný", end=", ")
        else:
            print("Rôznostranný", end=", ")

	if alpha < 90 and beta < 90 and gamma > 90:
		print("Ostrouhlý")
	elif alpha > 90 or beta > 90 or gamma > 90:
		print("Tupouhlý")
	else:
        print("Pravouhlý")
\end{solution}

\subsection{Cykly}

\subsubsection*{1. 100-krát napíš}
\begin{solution}
veta = input("Musím napísať: ")
pocet = int(input("Toľkoto krát: "))
for i in range(pocet):
    print(veta)
\end{solution}

\subsubsection*{2. Hodnotenie}
\begin{solution}
skore = int(input("Skóre: "))
for i in range(skore):
    print("*", end="")
print()
\end{solution}


\subsubsection*{3. Pyramída}
\begin{solution}
vyska = int(input("Výška pyramídy: "))
for riadok in range(vyska):
    medzery = vyska - riadok - 1
    hviezdy = 2 * riadok + 1
    print(" " * medzery + "*" * hviezdy)
\end{solution}

\subsubsection*{4. Smaragd}
\begin{solution}
vyska = int(input("Veľkosť: "))

if vyska < 3 or vyska % 2 != 1:
    print("Neviem vytvoriť taký smaragd")
else:
    vyska = (vyska // 2) + 1

    # Horná časť
    for riadok in range(vyska):
        medzery = vyska - riadok - 1
        hviezdy = 2 * riadok + 1
        print(" " * medzery + "*" * hviezdy)

    # Dolná časť
    for riadok in range(1, vyska):
        medzery = riadok
        hviezdy = 2 * (vyska - riadok) - 1
        print(" " * medzery + "*" * hviezdy)
\end{solution}


\subsubsection*{5. Duté vnútro}

\begin{solution}
vyska = int(input("Výška pyramídy: "))
for riadok in range(vyska):
    medzery = vyska - riadok - 1
    dute = 2 * riadok - 1

    print(" " * medzery, end="")
    if riadok == 0:
        print("*")
    elif riadok == vyska - 1:
        print("*" * (dute + 2))
    else:
        print("*" + " " * dute + "*")
\end{solution}

\subsubsection*{6. Mriežka slov}
\begin{solution}
n = int(input("Počet riadkov a stĺpcov: "))
slovo = input("Opakovať slovo: ")

for riadok in range(n):
    for stlpec in range(n):
        print(slovo, end=" ")
    print()
\end{solution}

\subsubsection*{7. Rám}
\begin{solution}
n = int(input("Počet riadkov a stĺpcov: "))
slovo = input("Opakovať slovo: ")
ram = len(slovo) * "#"

for riadok in range(n):
    for stlpec in range(n):
        if riadok == 0 or stlpec == 0 or riadok == n - 1 or stlpec == n - 1:
            print(ram, end=" ")
        else:
            print(slovo, end=" ")
    print()
\end{solution}

\subsubsection*{8. Malá násobilka}
\begin{solution}
for i in range(1, 11):
    for j in range(1, 11):
        print(f"{i * j:3d}", end=" ")
    print()
\end{solution}

\subsubsection*{9. Sporenie}
\begin{solution}
vklad = float(input("Vklad v Eur: "))
sadzba = float(input("Úroková sadzba p.a. v \%: "))
urocenie = input("Typ úročenia (jednoduché / zložené): ")
ciel = float(input("Žiadaná suma v Eur: "))

sadzba /= 100
rok = 0
suma = vklad

if urocenie == "jednoduché":
    urok = vklad * sadzba
if urocenie == "zložené":
    sadzba += 1
    povodna_sadzba = sadzba

print(f"{'Mesiac':10s} {'Suma':15s} {'Úrok':10s}")

while suma < ciel:
    if urocenie == "jednoduché":
        suma += urok
    elif urocenie == "zložené":
        urok = suma * (sadzba - 1)
        suma = vklad * sadzba
        sadzba *= povodna_sadzba
    else:
        break
    rok += 1
    print(f"{rok:10d} {suma:15.2f} {urok:10.2f}")
\end{solution}

%\subsection{Náhodné čísla}
\subsubsection*{1. Hádzanie kockou}

\begin{solution}
import random
input("HOĎ")
kocka = random.randint(1, 6)

if kocka == 1:
	print("+-------+")
	print("|       |")
    print("|   #   |")
	print("|       |")
	print("+-------+")
elif kocka == 2:
	print("+-------+")
    print("| #     |")
    print("|       |")
    print("|     # |")
    print("+-------+")
elif kocka == 3:
	print("+-------+"
	print("| #     |")
	print("|   #   |")
	print("|     # |")
	print("+-------+")
elif kocka == 4:
	print("+-------+")
	print("| #   # |")
	print("|       |")
	print("| #   # |")
	print("+-------+")
elif kocka == 5:
	print("+-------+")
	print("| #   # |")
	print("|   #   |")
	print("| #   # |")
	print("+-------+")
elif kocka == 6:
	print("+-------+")
	print("| #   # |")
	print("| #   # |")
	print("| #   # |")
	print("+-------+")
\end{solution}

\subsubsection*{2. Hádaj číslo}
\begin{solution}
import random
hadaj = random.randint(1, 100)
while True:
	tip = int(input("Hádaj číslo: "))
	if tip > hadaj:
		print("Veľa")
	elif tip < hadaj:
        print("Málo")
    else:
        print("Uhádol si")
        break
\end{solution}


\subsubsection*{3. Opakovanie násobilky}
\begin{solution}
import random
while True:
    x = random.randint(1, 10)
    y = random.randint(1, 10)
    print(f"Koľko je {x} x {y}?")
    vysledok = int(input("= "))

    while vysledok != x * y:
        print("Nesprávne - hádaj znovu")
        vysledok = int(input("= "))

    print("Správne - len tak ďalej")
    pokracuj = input("Chceš ďaľší príklad? (a / n): ")
    if pokracuj == 'n':
        break
\end{solution}

%\section{Reťazce a zoznamy}

\subsection{Vymeň písmeno}
\begin{solution}
text = input("Správa: ")
chyba = input("Za chybné písmeno: ")
nahrada = input("Vymeň: ")
upravene = ""
for pismeno in text:
    if pismeno == chyba:
        upravene += nahrada
    else:
        upravene += pismeno
print("\nOpravené!")
print(upravene)
\end{solution}


\subsection{Cenzúra}
\begin{solution}
vstup = input("Správa: ")
prepis = input("Samohlásku nahraď: ")
vystup = ""
samohlasky = "aeiouyáéíóúý"
najdene = False

for i in range(len(vstup)):
    for j in range(len(samohlasky)):
        if vstup[i] == samohlasky[j]:
            vystup += prepis
            najdene = True
            break
    if not najdene:
        vystup += vstup[i]
    najdene = False
 
print("Cenzurované", vystup)
\end{solution}

\subsection{Počítanie slov}
\begin{solution}
clanok = input("Článok: ")
pocet_znakov = 0
pocet_slov = 0
pocet_viet = 0
je_medzera = True

for znak in clanok:
    pocet_znakov += 1
    if znak == ".":
        pocet_viet += 1
    if znak.isspace():
        je_medzera = True
    elif je_medzera and not znak.isspace():
        pocet_slov += 1
        je_medzera = False

print(f"Znaky: {pocet_znakov}")
print(f"Slová: {pocet_slov}")
print(f"Vety: {pocet_viet}")
print(f"Normostany: {int(pocet_znakov / 1800)}")
\end{solution}


\subsection{Najdlhšie slovo}
\begin{solution}
prejav = input("Rečnícky prejav: ")
slovo = ""
najdlhsie = ""

for znak in prejav:
    if znak.isalpha():
        slovo += znak
    else:
        if len(slovo) > len(najdlhsie):
            najdlhsie = slovo
        slovo = ""

print(f"Najdlhšie slovo v ňom: {najdlhsie}")
\end{solution}

\subsection{Frekvencia písmen}
\begin{solution}
clanok = input("Článok: ")
abeceda = [0] * 26
pismena = 0

for pismeno in clanok:
    if pismeno.isalpha():
        pozicia = ord(pismeno.upper()) - ord("A")
        if pozicia >= 0 and pozicia <= 26:
            abeceda[pozicia] += 1
            pismena += 1

for i in range(len(abecedaReťazce a zoznamy - Riešenia)):
    pismeno = chr(ord("A") + i)
    vyskyt = 100 * (abeceda[i] / pismena)
    print(f"{pismeno}: {vyskyt:.2f}%")
\end{solution}

\subsection{Histogram}
\begin{solution}
clanok = input("Článok: ")

STO_PERCENT = 100
abeceda = [0] * 26
pismena = 0

for pismeno in clanok:
    if pismeno.isalpha():
        pozicia = ord(pismeno.upper()) - ord("A")
        if pozicia >= 0 and pozicia <= 26:
            abeceda[pozicia] += 1
            pismena += 1
for i in range(len(abeceda)):
	pismeno = chr(ord("A") + i)
	vyskyt = int(STO_PERCENT * (abeceda[i] / pismena))
	print(f"{pismeno}: {'*' * vyskyt}")
\end{solution}

\subsection{Nákupný košík}
\begin{solution}
nakup = []
while True:
    tovar = input("Čo kúpiť?: ")
    if tovar == "HOTOVO":
        break
    cena = float(input(f"Cena {tovar}?: "))
    nakup.append([tovar, cena])

riadok = "+" + 20 * "-" + "+" + 15 * "-" + "+" + 15 * "-" + "+"
print(riadok)
print(f"|{'Tovar':20s}|{'DPH':15s}|{'Cena s DPH':15s}|")

celkom = 0
for polozka in nakup:
    tovar = polozka[0]
    cena = polozka[1]
    celkom += cena
    print(riadok)
    print(f"|{tovar:20s}|{cena * 0.2:15.2f}|{cena:15.2f}|")

print(riadok)
print(f"|{'CELKOM':20s}|{celkom * 0.2:15.2f}|{celkom:15.2f}|")
print(riadok)
\end{solution}


\subsection{Akronym}
\begin{solution}
veta = input("Slovné spojenie: ")
skratka = ""
je_medzera = True
for znak in veta:
	if znak.isspace():
		je_medzera = True
	elif je_medzera and znak.isalpha():
		je_medzera = False
		skratka += znak.upper()
print(f"Skratka: {skratka}")
\end{solution}


\subsection{Veľa opakovania}
\begin{solution}
cesta = input("Cesta robota: ")
skratene = ""
smer = ""
n = 0
for krok in cesta:
	if krok.isalpha():
		if smer == "":
			smer = krok
			n = 1
		elif krok != smer:
			skratene += f"{n}{smer}"
			smer = krok
			n = 1
		else:
			n += 1
skratene += f"{n}{smer}"
print(f"Skomprimované: {skratene}")
\end{solution}

%\subsection{Súbory}

\subsubsection*{1. Prepisovanie}

\begin{solution}
nazov_suboru = input("Názov súboru")
subor = open(nazov_suboru, "r")

for riadok in subor:
    riadok = riadok.strip()

subor.close()
\end{solution}

\subsubsection*{2. Turistika}

\begin{solution}
nazov = input("Výškový profil trasy je v súbore: ")

ROVINA_KMH = 3.6
KROK_M = 10

predch_vyska = None
vzdialenost_m = 0
trvanie_min = 0

celkom_stupanie = 0
celkom_klesanie = 0

najvyssie = None
najnizsie = None

trasa = open(nazov, "r")

for miesto in trasa:
    nadmorska_vyska = int(miesto)
    vzdialenost_m += KROK_M

    # Ak neexistuje predošlá nadmorská výška, tak sme neprešli žiaden úsek
    if predch_vyska != None:
        stupanie = nadmorska_vyska - predch_vyska

        # Zisti, či sme dosiahli rekordnú nadmorskú výšku a zaznamenaj si ju.
        if najvyssie == None or nadmorska_vyska > najvyssie:
            najvyssie = nadmorska_vyska
        elif najnizsie == None or nadmorska_vyska < najnizsie:
            najnizsie = nadmorska_vyska

        # Zobrazenie vzdialenosti medzi dvomi miestami zo svahu do roviny.
        # Pri stúpaní prejdeme za rovnaký čas akoby kratšiu vzdialenosť, preto
        # sa prepona zobrazí do dolnej odvesy a pri klesaní naopak
        if stupanie > 0:
            rovina_vzd = KROK_M ** 2 - stupanie ** 2
            celkom_stupanie += stupanie

        elif stupanie < 0:
            rovina_vzd = KROK_M ** 2 + stupanie ** 2
            celkom_klesanie += abs(stupanie)

        else:
            rovina_vzd = KROK_M

        # Čas na prejdenie medzi miestami v minútach
        trvanie_min += ((rovina_vzd / 1000) / ROVINA_KMH) * 60

    predch_vyska = nadmorska_vyska

trasa.close()

print(f"Trasa: {vzdialenost_m / 1000:.3f} km - "
      f"{int(trvanie_min // 60)} h {int(trvanie_min % 60)} min")
print(f"Stúpanie: {celkom_stupanie} m")
print(f"Klesanie: {celkom_klesanie} m")
print(f"Najnižšie miesto trasy: {najnizsie} m")
print(f"Najvyššie miesto trasy: {najvyssie} m")
\end{solution}

\subsubsection*{3. Vedomostný kvíz}

\begin{solution}
import random

nazov = input("Súbor s kvízovými otázkami: ")
kviz = open(nazov, "r")

otazky = []
skore = 0

# Ulož si pozície otázok v súbore
while True:
    riadok = kviz.readline()
    if not riadok:
        break
    if riadok.startswith("Otázka: "):
        znacka = kviz.tell() - len(riadok)
        otazky.append(znacka)

print("Kvízové otázky pripravené.")
print("Ideme na to!", end="\n\n")

while True:
    # Náhodne vyber otázku
    i = random.randint(0, len(otazky) - 1)
    znacka = otazky[i]
    kviz.seek(znacka)

    # Spýtaj sa otázku a navrhni možnosti
    for riadok in kviz:
        riadok = riadok.rstrip()

        if riadok.startswith("Odpoveď: "):
            odpoved = riadok.lstrip("Odpoveď: ")
            break

        print(riadok.lstrip("Otázka: "))

    # Hráčov tip
    tip = input("Aká je správna odpoveď?: ")
    if tip == odpoved:
        skore += 1
        print(f"Správne! Máš {skore} bodov.\n")
    else:
        print("Nabudúce si to lepšie premysli. Skúsime niečo iné.\n")

kviz.close()
\end{solution}


\subsubsection*{4. Narodeniny}

\begin{solution}
datum = input("Zobraz narodeniny pre mesiac v roku: ")
datum = datum.split(".")

NAZVY_MESIACOV = ["Január", "Február", "Marec", "Apríl", "Máj", "Jún", "Júl",
                  "August", "September", "Október", "November", "December"]
mesiac = int(datum[0])
rok = int(datum[1])

narodeniny = open("narodeniny.csv", "r")
print(f"\nNarodeniny: {NAZVY_MESIACOV[mesiac - 1]} {rok}")

for osoba in narodeniny:
    osoba = osoba.split(",")
    meno = osoba[0]
    datum = osoba[1].split(".")

    narodenie_den = int(datum[0])
    narodenie_mesiac = int(datum[1])
    narodenie_rok = int(datum[2])

    if narodenie_mesiac == mesiac:
        print(f"{narodenie_den}.{narodenie_mesiac} - {meno} - "
              f"{rok - narodenie_rok} rokov")

narodeniny.close()
\end{solution}


\subsubsection*{5. Pripomienky v kalendári}

\begin{solution}
from datetime import datetime
from datetime import timedelta

csv_nazov = input("Prečítaj narodeniny zo súboru (.csv): ")
ics_nazov = input("Priprav kalendár s názvom (.ics): ")

narodeniny = open(csv_nazov, "r")
kalendar = open(ics_nazov, "w")
ciselnik = 0

print("BEGIN:VCALENDAR", file=kalendar)
print("PRODID:Programatorsky kruzok", file=kalendar)
print("VERSION:2.0", file=kalendar)

for osoba in narodeniny:
    osoba = osoba.split(",")
    meno = osoba[0].strip()
    narodenie = osoba[1].strip()

    print("BEGIN:VEVENT", file=kalendar)

    # Pre zjednodušenie prečítaj časovú značku na vstupe.
    casova_znacka = datetime.now().strftime("%Y%m%dT%H%M%SZ")
    print(f"DTSTAMP:{casova_znacka}", file=kalendar)

    # Podľa správnosti by si mal vygenerovať jedinečný kód, napríklad takto:
    # from uuid import uuid1
    # ciselnik = uuid1()
    ciselnik += 1
    print(f"UID:{ciselnik}", file=kalendar)

    # Krátky popis a kategória udalosti v kalendári. Ide o narodeniny.
    print(f"SUMMARY:{meno} narodeniny", file=kalendar)
    print("CATEGORIES:Narodeniny", file=kalendar)

    # Kalendár zobrazí udalosť každý rok
    print("RRULE:FREQ=YEARLY", file=kalendar)

    # Pre zjednodušenie zostav dátum narodenín spájaním reťazcov.
    datum = datetime.strptime(narodenie, "%d.%m.%Y")
    start = datum.strftime("%Y%m%d")
    print(f"DTSTART;VALUE=DATE:{start}", file=kalendar)

    koniec = datum + timedelta(days=1)
    koniec = koniec.strftime("%Y%m%d")
    print(f"DTEND;VALUE=DATE:{koniec}", file=kalendar)

    # Umožní vytvárať aj iné udalosti na deň narodenín
    print("TRANSP:TRANSPARENT", file=kalendar)

    # Pripomienka týždeň pred narodeninami
    print("BEGIN:VALARM", file=kalendar)
    print("DESCRIPTION:", file=kalendar)
    print("ACTION:DISPLAY", file=kalendar)
    print("TRIGGER:-P7D", file=kalendar)
    print("END:VALARM", file=kalendar)

    print("END:VEVENT", file=kalendar)

print("END:VCALENDAR", file=kalendar)
print("Hotovo.")

narodeniny.close()
kalendar.close()
\end{solution}


\subsubsection*{6. Cestovné poriadky}
\begin{solution}
odchod = input("Čas: ")
trvanie = input("Trvanie cesty vlakom: ")

odchod = odchod.split(":")
hod = int(odchod[0])
min = int(odchod[1])
odchod = [hod, min]

trvanie = trvanie.split(":")
hod = int(trvanie[0])
min = int(trvanie[1])
trvanie = [hod, min]

vlaky = []
autobusy = []
cp = open("cp.csv", "r")

for spoj in cp:
    spoj = spoj.split(",")
    doprava = spoj[0].strip()

    if doprava == "bus":
        autobusy.append([])

    for cas in spoj[1:]:
        cas = cas.split(":")
        hod = int(cas[0])
        min = int(cas[1])

        if doprava == "vlak":
            vlaky.append([hod, min])
        elif doprava == "bus":
            autobusy[-1].append([hod, min])

print("Najbližší spoj (vlak, autobus):")
nasiel = False

for vlak in vlaky:
    # Nájdi najbližší odchod vlaku
    if (vlak[0] * 60 + vlak[1]) >= (odchod[0] * 60 + odchod[1]):
        # Zisti, kedy prídeme odchod + trvanie = prichod
        min = (vlak[1] + trvanie[1]) % 60
        hod = ((vlak[0] + trvanie[0]) + ((vlak[1] + trvanie[1]) // 60)) % 24
        prichod = [hod, min]

        for linka in autobusy:
            stanica = linka[0]

            # K tomu pozri autobusovú linku, ktorá odchádza zo stanice, do ktorej vlak ide
            if (stanica[0] * 60 + stanica[1]) >= (trvanie[0] * 60 + trvanie[1]):

                for autobus in linka[1:]:
                    # Prestup: Nájdi autobus, ktorý odchádza najskôr po príchode vlaku
                    if (not nasiel and (autobus[0] * 60 + autobus[1]) > (prichod[0] * 60 + prichod[1])):
                        print(f"{vlak[0]:02d}:{vlak[1]:02d} - "
                              f"{prichod[0]:02d}:{prichod[1]:02d}, "
                              f"{autobus[0]:02d}:{autobus[1]:02d} - ")
                        nasiel = True
cp.close()
\end{solution}
%\subsection{Funkcie}

\subsubsection*{1. Vraky}
\begin{solution}
import random
import math
MIERKA = 15
VRAK_X = random.randint(0, MIERKA)
VRAK_Y = random.randint(0, MIERKA)

def vzdialenost(x, y):
    return math.hypot(x - VRAK_X, y - VRAK_Y)
  
def nasiel(x, y):
    return x == VRAK_X and y == VRAK_Y

print("Sonar hlási potopený parník na dohľad!")
while True:
    suradnice = input("Tvoje súradnice?: ")
    suradnice = suradnice.split(",")
    x = int(suradnice[0])
    y = int(suradnice[1])
    if nasiel(x, y):
        print("Našiel si vrak. Dobrá práca!")
        break

    print(f"Od vraku si {vzdialenost(x, y):.3f} námornych míľ")
\end{solution}


\subsubsection*{2. Lietadlo}
\begin{solution}
from math import sin, cos, acos, radians
def letime(x, y):
    POLOMER_ZEME = 6371.11
    uhol = acos(sin(x[0]) * sin(y[0]) +
                cos(x[0]) * cos(y[0]) * cos(abs(x[1] - y[1])))
    obluk = POLOMER_ZEME * uhol
    return obluk

start = input("Pozícia: ")
ciel = input("Cieľ: ")

start = start.split(" ")
start = [
    radians(float(start[0])),
    radians(float(start[1]))
]
ciel = ciel.split(" ")
ciel = [
    radians(float(ciel[0])),
    radians(float(ciel[1]))
]
vzdielenost = letime(start, ciel)
print(f"\nVzdialenosť: {vzdielenost:.2f} km")
\end{solution}


\subsubsection*{3. Cézarová šifra}

\begin{solution}
def sifruj(sprava, kluc):
    sifra = ""
    A = ord("A")
    Z = ord("Z")
    ABECEDA = Z - A + 1
    sprava = sprava.upper()

    for i in range(len(sprava)):
        pismeno = sprava[i]
        k = kluc[i % len(kluc)]

        if A <= ord(pismeno) <= Z:
            poradie = ord(pismeno) - A
            posun = ord(k) - A

            poradie = (poradie + posun) % ABECEDA
            sifra += chr(poradie + A)

    return sifra

def desifruj(sifra, kluc):
    sprava = ""
    A = ord("A")
    Z = ord("Z")
    ABECEDA = Z - A + 1
    sifra = sifra.upper()

    for i in range(len(sifra)):
        pismeno = sifra[i]
        k = kluc[i % len(kluc)]

        if A <= ord(pismeno) <= Z:
            poradie = ord(pismeno) - A
            posun = ord(k) - A

            poradie = (poradie - posun) % ABECEDA
            sprava += chr(poradie + A)

    return sprava

retazec = input("Zadaj správu: ")
kluc = input("Vlož tajný kľúč: ")
akcia = input("Čo spraviť (šifruj / dešifruj): ")
s = ""
if akcia == "šifruj":
    print("Zašifrovaná správa: ", end="")
    s = sifruj(retazec, kluc)
elif akcia == "dešifruj":
    print("Dešifrovaná správa: ", end="")
    s = desifruj(retazec, kluc)
print(s)
\end{solution}

\subsubsection*{4. Pascalov trojuholník}
\begin{solution}
def pascalov_trojuholnik(n):
    row = [1, 1]
    medzery = n
    pocet = 0

    for i in range(n):
        pocet += 1
        medzery -= 1

        print(" " * medzery, end="")
        for cislo in row[:pocet]:
            print(cislo, end=" ")
        print()

        for j in range(pocet - 1,  0, -1):
            row[j] = row[j] + row[j - 1]
        row.append(1)

vyska = int(input("Zadajte výšku Pascalovho trojuholníka: "))
pascalov_trojuholnik(vyska)
\end{solution}

\subsubsection*{5. Štatistika}
\begin{solution}
import math
def priemer(zoznam):
    sucet = 0
    for prvok in zoznam:
        sucet += prvok
    return sucet / len(zoznam)

def modus(zoznam):
    nazvy = []
    vyskyty = []
    # Zisti koľkokrát sa čo vyskytuje
    for prvok in zoznam:
        index = -1
        for i in range(len(nazvy)):
            if prvok == nazvy[i]:
                index = i

        if index != -1:
            vyskyty[index] += 1
        else:
            nazvy.append(prvok)
            vyskyty.append(0)

    # Pozri sa po najväčšom počte objavení sa a prehlás ho za modus
    najviac = None
    rekorder = -1

    for i in range(len(vyskyty)):
        if najviac == None or vyskyty[i] > najviac:
            najviac = vyskyty[i]
            rekorder = i

    return nazvy[rekorder]

def utried(zoznam):
    for i in range(len(zoznam) - 1):
        for j in range(len(zoznam) - i - 1):
            if zoznam[j] > zoznam[j + 1]:
                x = zoznam[j]
                zoznam[j] = zoznam[j + 1]
                zoznam[j + 1] = x

def median(zoznam):
    utried(zoznam)
    stred = (len(zoznam) + 1) // 2
    return zoznam[stred - 1]

def smerodajna_odchylka(zoznam):
    average = priemer(zoznam)

    sucet = 0
    for prvok in zoznam:
        sucet += (prvok - average) ** 2

    return math.sqrt(sucet / len(zoznam))

subor = input("Súbor s bytmi v lokalite: ")
ceny = []
vymery = []

byty = open(subor, "r")

for byt in byty:
    zaznam = byt.split(",")
    ceny.append(int(zaznam[0]))
    vymery.append(int(zaznam[1]))

byty.close()

# Pozri tiež modul "statistics" - https://docs.python.org/3/library/statistics.html
print(f"{'':25s}:{'Cena (eur)':15s}:{'Výmera(m^2)':15s}:")
print(f"{'Priemer':25s}:{priemer(ceny):15.2f}:{priemer(vymery):15.2f}:")
print(f"{'Medián':25s}:{median(ceny):15.2f}:{median(vymery):15.2f}:")
print(f"{'Modus':25s}:{modus(ceny):15.2f}:{modus(vymery):15.2f}:")
print(f"{'Smerodajná odchýlka':25s}:{smerodajna_odchylka(ceny):15.2f}:{smerodajna_odchylka(vymery):15.2f}:")
\end{solution}

\subsubsection*{6. Rímske čísla}
\begin{solution}
def rimske_na_arabske(rimske):
    TABULKA = {"I": 1, "V": 5, "X": 10, "L": 50, "C": 100, "D": 500, "M": 1000}
    arabske = []
    vysledok = 0
    for symbol in rimske:
        arabske.append(TABULKA[symbol])

    i = 0
    while i < len(arabske):
        if i + 1 != len(arabske) and arabske[i] < arabske[i + 1]:
            vysledok += arabske[i + 1] - arabske[i]
            i += 2
        else:
            vysledok += arabske[i]
            i += 1
    return vysledok

cislo = input("Zadaj rímske číslo: ")
print(rimske_na_arabske(cislo))
\end{solution}

\subsubsection*{7. Základný tvar zlomku}
\begin{solution}
def nsd(a, b):
    # Najväčší spoločný deliteľ
    # alebo: math.gcd(a, b)
    while b > 0:
        a, b = b, a \% b
    return a

def nsn(a, b):
    # Najmenší spoločný násobok
    return a * b // nsd(a, b)

def zakladny_tvar(zlomok):
    delitel = nsd(zlomok[0], zlomok[1])
    return [
        zlomok[0] // delitel,
        zlomok[1] // delitel
    ]

def vytvor_zlomok(retazec):
    # alebo: map(int, retazec.split("/"))
    zlomok = retazec.split("/")
    for i in range(len(zlomok)):
        zlomok[i] = int(zlomok[i])
    return zlomok

def nasobit(x, y):
    citatel = x[0] * y[0]
    menovatel = x[1] * y[1]
    zlomok = [citatel, menovatel]
    return zakladny_tvar(zlomok)

def delit(x, y):
    citatel = x[0] * y[1]
    menovatel = x[1] * y[0]
    zlomok = [citatel, menovatel]
    return zakladny_tvar(zlomok)

def scitat(x, y):
    menovatel = nsn(x[1], y[1])
    x_citatel = x[0] * (menovatel // x[1])
    y_citatel = y[0] * (menovatel // y[1])
    citatel = x_citatel + y_citatel
    zlomok = [citatel, menovatel]
    return zakladny_tvar(zlomok)

def odcitat(x, y):
    menovatel = nsn(x[1], y[1])
    x_citatel = x[0] * (menovatel // x[1])
    y_citatel = y[0] * (menovatel // y[1])
    citatel = x_citatel - y_citatel
    zlomok = [citatel, menovatel]
    return zakladny_tvar(zlomok)

def vypis(zlomok):
    return f"{zlomok[0]}/{zlomok[1]}"


print("Kalkulačka zlomkov")
a = input("a = ")
b = input("b = ")
akcia = input("Vypočítaj (+, -, *, /): ")

a = zakladny_tvar(vytvor_zlomok(a))
b = zakladny_tvar(vytvor_zlomok(b))

print("\nVýsledok:")
if akcia == '+':
    print(f"{vypis(a)} + {vypis(b)} = {vypis(scitat(a, b))}")
elif akcia == '-':
    print(f"{vypis(a)} - {vypis(b)} = {vypis(odcitat(a, b))}")
elif akcia == '*':
    print(f"{vypis(a)} * {vypis(b)} = {vypis(nasobit(a, b))}")
elif akcia == '/':
    print(f"{vypis(a)} / {vypis(b)} = {vypis(delit(a, b))}")
\end{solution}


\section{Systém úloh}

\begin{table}[h]
\centering
\begin{adjustbox}{width=1\textwidth}
\def\arraystretch{1.2}
\begin{tabular}{|l|c|c|c|c|c|c|c|c|c|c|c|}
\hline
\diagbox{kategória}{úloha}           & 1. & 2. & 3. & 4. & 5. & 6. & 7. & 8. & 9. & 10. & 11.\\ \Xhline{4\arrayrulewidth}
upevňovanie učiva       &    & $\bigtimes$ & $\bigtimes$  &    & $\bigtimes$ &  $\bigtimes$ & $\bigtimes$ &  &  &  &  \\ \hline
aplikácia mimo odbor    &    & $\bigtimes$ & $\bigtimes$  & $\bigtimes$ & $\bigtimes$ & $\bigtimes$ & $\bigtimes$ & $\bigtimes$ & $\bigtimes$ & $\bigtimes$ & $\bigtimes$   \\ \hline
aplikácia vo vnútri odboru    &    &    &    &    &    &    &     &  &    &    &  \\ \hline
opakovanie a systematizácia   &    &    & $\bigtimes$   &    &    & $\bigtimes$  & $\bigtimes$ & $\bigtimes$ & $\bigtimes$ & $\bigtimes$ &  \\ \hline
aktualizačné úlohy            &    &    &    & $\bigtimes$ &    &    &  & & & & \\ \hline
prípravné úlohy               & $\bigtimes$ &    &    &    &    &    &   & &  & &\\ \hline
osvojenie pojmu a postupu     & $\bigtimes$ & $\bigtimes$ &  & $\bigtimes$  &    &    & & & & &   \\ \hline
motivačné úlohy                    & $\bigtimes$ & $\bigtimes$ &    & $\bigtimes$ & $\bigtimes$ &    & & & & & $\bigtimes$   \\ \hline
propedeutické úlohy                & $\bigtimes$ &    &    & $\bigtimes$   &    &    & & &   & & \\ \Xhline{4\arrayrulewidth}
nižšie konvergentné procesy        & $\bigtimes$ &  $\bigtimes$  & $\bigtimes$ & $\bigtimes$ &  &  & $\bigtimes$ & $\bigtimes$ & $\bigtimes$ & $\bigtimes$ &  \\ \hline
vyššie konvergentné procesy        &    &    &    &    & $\bigtimes$ &  $\bigtimes$  &  & & & & $\bigtimes$ \\ \hline
hodnotiace myslenie                &    &    &    &    &    &    &  & & & & \\ \hline
divergentné myslenie               &    &    &    &    &    &    &  & & & & \\ \Xhline{4\arrayrulewidth}
fog index                          &  15,16  & 15,51  & 16,53 & 13,86 & 16,37 & 16,57  & 14,53 & 21,34 & 12,77 & 21,38 & 20,52 \\ \hline
\end{tabular}
\end{adjustbox}
\caption{Premenné}
\end{table} 


\begin{table}[h]
\centering
\begin{adjustbox}{width=1\textwidth}
\def\arraystretch{1.2}
\begin{tabular}{|l|c|c|c|c|c|c|c|c|c|}
\hline
\diagbox{kategória}{úloha}           & 1. & 2. & 3. & 4. & 5. & 6. & 7. & 8. & 9. \\ \Xhline{4\arrayrulewidth}
upevňovanie učiva       &  &  & $\bigtimes$ & $\bigtimes$  &  &  & & &  \\ \hline
aplikácia mimo odbor    &  & $\bigtimes$ & $\bigtimes$  & $\bigtimes$  & $\bigtimes$ &  & $\bigtimes$ & $\bigtimes$ &  $\bigtimes$ \\ \hline
aplikácia vo vnútri odboru    & $\bigtimes$ &  &  &   &   & $\bigtimes$ & & &  \\ \hline
opakovanie a systematizácia   &  &  &  & $\bigtimes$  & $\bigtimes$ & $\bigtimes$  & $\bigtimes$ & $\bigtimes$ & $\bigtimes$ \\ \hline
aktualizačné úlohy            & $\bigtimes$ &  $\bigtimes$ &  &   & $\bigtimes$  &  & & $\bigtimes$ & \\ \hline
prípravné úlohy              & $\bigtimes$ &  &  &   &  &  & & & \\ \hline
osvojenie pojmu a postupu     & $\bigtimes$ & $\bigtimes$  &  & $\bigtimes$   &  &  & & & \\ \hline
motivačné úlohy                    &  &  & $\bigtimes$ & $\bigtimes$  &  &  & $\bigtimes$ & $\bigtimes$ & $\bigtimes$ \\ \hline
propedeutické úlohy                & $\bigtimes$ &  &  &   &  &  & & & \\ \Xhline{4\arrayrulewidth}
nižšie konvergentné procesy        & $\bigtimes$ & $\bigtimes$  &  &   & $\bigtimes$ & $\bigtimes$  & & $\bigtimes$ & \\ \hline
vyššie konvergentné procesy        &  &  &  & $\bigtimes$  &  &  & $\bigtimes$ & & $\bigtimes$ \\ \hline
hodnotiace myslenie                & &  & $\bigtimes$  &  &   &  &  & & \\ \hline
divergentné myslenie               &  &  &  &   &  &  & & & \\ \Xhline{4\arrayrulewidth}
fog index                          & 15,42  & 11,00  & 16,93  &  18,83 & 26,22  & 18,04 &  16,46 & 17,99 & 18,97 \\ \hline
\end{tabular}
\end{adjustbox}
\caption{Podmienky}
\end{table} 

\begin{table}[h]
\centering
\begin{adjustbox}{width=1\textwidth}
\def\arraystretch{1.2}
\begin{tabular}{|l|c|c|c|c|c|c|c|c|c|}
\hline
\diagbox{kategória}{úloha}           & 1. & 2. & 3. & 4. & 5. & 6. & 7. & 8. & 9. \\ \Xhline{4\arrayrulewidth}
upevňovanie učiva       &  &  &  &   &  &  & & &  \\ \hline
aplikácia mimo odbor    &  &  &  &   &  &  & & & \\ \hline
aplikácia vo vnútri odboru    &  &  &  &   &  &  & &  & \\ \hline
opakovanie a systematizácia   &  &  &  &   &  &  & & & \\ \hline
aktualizačné úlohy            &  &  &  &   &  &  & & & \\ \hline
prípravné úlohy              &  &  &  &   &  &  & & & \\ \hline
osvojenie pojmu a postupu     &  &  &  &   &  &  & & & \\ \hline
motivačné úlohy                    &  &  &  &   &  &  & & &  \\ \hline
propedeutické úlohy                &  &  &  &   &  &  & & & \\ \Xhline{4\arrayrulewidth}
nižšie konvergentné procesy        &  &  &  &   &  &  & & & \\ \hline
vyššie konvergentné procesy        &  &  &  &   &  &  & & & \\ \hline
hodnotiace myslenie                & &  &  &  &   &  &  & & \\ \hline
divergentné myslenie               &  &  &  &   &  &  & & & \\ \Xhline{4\arrayrulewidth}
fog index                          &  &  &  &   &  &  & &  &\\ \hline
\end{tabular}
\end{adjustbox}
\caption{Cykly}
\end{table} 



%-----------------------------------------------

\begin{table}[h]
\centering
\begin{adjustbox}{width=1\textwidth}
\def\arraystretch{1.2}
\begin{tabular}{|l|c|c|c|c|c|c|c|c|}
\hline
\diagbox{kategória}{úloha}           & 1. & 2. & 3. & 4. & 5. & 6. & 7. & 8. \\ \Xhline{4\arrayrulewidth}
upevňovanie učiva       &  &  &  &   &  &  & &   \\ \hline
aplikácia mimo odbor    &  &  &  &   &  &  & &   \\ \hline
aplikácia vo vnútri odboru    &  &  &  &   &  &  & &  \\ \hline
opakovanie a systematizácia   &  &  &  &   &  &  & &  \\ \hline
aktualizačné úlohy            &  &  &  &   &  &  & & \\ \hline
prípravné úlohy              &  &  &  &   &  &  & & \\ \hline
osvojenie pojmu a postupu     &  &  &  &   &  &  & &  \\ \hline
motivačné úlohy                    &  &  &  &   &  &  & &   \\ \hline
propedeutické úlohy                &  &  &  &   &  &  & & \\ \Xhline{4\arrayrulewidth}
nižšie konvergentné procesy        &  &  &  &   &  &  & &  \\ \hline
vyššie konvergentné procesy        &  &  &  &   &  &  & & \\ \hline
hodnotiace myslenie                & &  &  &  &   &  &  & \\ \hline
divergentné myslenie               &  &  &  &   &  &  & & \\ \Xhline{4\arrayrulewidth}
fog index                          &  &  &  &   &  &  & & \\ \hline
\end{tabular}
\end{adjustbox}
\caption{Zoznamy}
\end{table} 

\begin{table}[h]
\centering
\begin{tabularx}{\textwidth}{|l|Y|Y|Y|Y|Y|Y|Y|Y|Y|}
\hline
\diagbox{kategória}{úloha}           & 1. & 2. & 3. & 4. & 5. & 6. & 7. & 8. & 9. \\ \Xhline{4\arrayrulewidth}
upevňovanie učiva       &  &  &  &   &  &  & & &  \\ \hline
aplikácia mimo odbor    &  &  &  &   &  &  & & &  \\ \hline
aplikácia vo vnútri odboru    &  &  &  &   &  &  & & &  \\ \hline
opakovanie a systematizácia   &  &  &  &   &  &  & & & \\ \hline
aktualizačné úlohy            &  &  &  &   &  &  & & & \\ \hline
prípravné úlohy              &  &  &  &   &  &  & & & \\ \hline
osvojenie pojmu a postupu     &  &  &  &   &  &  & & & \\ \hline
motivačné úlohy                    &  &  &  &   &  &  & & &  \\ \hline
propedeutické úlohy                &  &  &  &   &  &  & & & \\ \Xhline{4\arrayrulewidth}
nižšie konvergentné procesy        &  &  &  &   &  &  & & & \\ \hline
vyššie konvergentné procesy        &  &  &  &   &  &  & & & \\ \hline
hodnotiace myslenie                & &  &  &  &   &  &  & & \\ \hline
divergentné myslenie               &  &  &  &   &  &  & & & \\ \Xhline{4\arrayrulewidth}
fog index                          &  &  &  &   &  &  & & & \\ \hline
\end{tabularx}
\caption{Súbory}
\end{table} 




\section{Diskusia} 

 

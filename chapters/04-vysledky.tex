\chapter{Výsledky práce}
Medzi najdôležitejšie tu nachádzajúce sa dielo patria problémové úlohy z programovania pre začiatočníkov na stredných školách. V súvislosti s našou zbierkou úloh rozoberieme adoptovanú anatómiu úlohy a dôsledky vyplývajúce z terajšieho zoradenia cvičení. Vychádzame pritom z teoretických základov tvorby učebníc a požiadaviek na pestrosť aplikáčného prostredia pre pojmy za účelom pritiahnutia záujmu učiaceho sa subjektu.

\section{Zbierka úloh}
Zhromaždený súbor zadaní z programovania sa skladá počtom z 54 úloh s riešeniami, ktoré sú rozčlenené do tématických častí nerovnomerne. Oddiel premenné má 11 úloh, podmienky - 9 úloh, cykly - 9 úloh, náhodné čísla - 3 úlohy, reťazce a zoznamy - 9 úloh, súbory - 5 úloh, funkcie - 8 úloh. Usporiadanie celkov teda pochádza čiastočne zo vzdelávacích štandardov ale aj zvyklostí.

Nepravidelnosti v kvantite častí nastali kvôli organickému a postupnému pribúdaniu úloh, ktoré reagovalo na potreby žiakov vedúce k zvládnutiu predmetu. Na kostru úloh sa dodatočne nabalili aj niektoré testové problémy, cvičenia na doma, ale upravene i učivo z iných predmetov, ktoré trápilo žiakov pred písomkami. K zdanlivo relatívne menšiemu objemu časti súbory poznamenáme, že v skutočnosti prvá úloha spočíva v prispôsobeniu všetkých úloh z predošlej témy na súbory. Prísne vzaté na súbory je preto k dispozícii až 13 úloh.

Podľa didaktickej zásady spojenia teórie s praxov a na prispenie ku kompentencii celoživotného učenia sa pre život dbáme na interdisciplinaritu nastolených problémov zasadených do reálneho sveta. Medzipredmetové aplikácie čerpajú z matematiky (16 úloh) a finančnej gramotnosti (2 úlohy), fyziky (4 úlohy), slovenského jazyka a literatúry (3 úlohy), geografie (2 úlohy), dejepisu (2 úlohy) a chémie (1 úloha). 

Matematika má z hľadiska využitia jej aparátu v programátorských úlohách nespornú výhodu pre blízkosť k informatike ako takej. Existujú zároveň viaceré možnosti znovupoužitia príkladov z matematických učebníc. K medzipremetovým vzťahov s \textbf{matematikou} prispievajú témy:
\begin{itemize}[noitemsep,topsep=0pt]
\item objem a obsah telies \emph{(1.6, 1.8, 1.9)}
\item meranie času \emph{(2.5)}
\item kalkulačky \emph{(2.6, 7.10)}
\item kvadratické rovnice \emph{(2.8)}
\item trojuholníky \emph{(2.9.)}
\item malá násobilka \emph{(3.8, 4.3)}
\item zlomky a percentá \emph{(5.5, 5.6, 7.6)}
\item finančná gramotnosť \emph{(3.9, 5.7)}
\item súradnicová sústava \emph{(7.1)}
\item binomická veta \emph{(7.3)}
\item štatistika \emph{(7.4)}
\end{itemize}
Do \textbf{fyziky} môžeme zaradiť úlohy ohľadom: 
\begin{itemize}[noitemsep,topsep=0pt]
\item teploty \emph{(1.4)}
\item kinematiky \emph{(1.5, 1.7)}
\item dynamiky \emph{(1.9)}
\end{itemize}
\textbf{Slovenského jazyka} sa okrajovo týkajú úlohy na:
\begin{itemize}[noitemsep,topsep=0pt]
\item rým \emph{(1.2)}
\item hláskoslovie \emph{(5.2)}
\item tvaroslovie \emph{(5.8)}
\end{itemize}
Pod témy súvisace s \textbf{geografiou} spadajú meteorológia \emph{(2.4)} a kartografia \emph{(6.2)}, z \textbf{chémie} sme zahrnuli roztoky \emph{(1.10)} a z \textbf{dejepisu} sa zmieňujeme o Májskych pyramídach \emph{(3.3)} a rímskych číslach \emph{(7.6)}. 
V zbierke je tiež 9 úloh na rýdzo \textbf{informatické} záležitosti:
\begin{itemize}[noitemsep,topsep=0pt]
\item heslá \emph{(2.1)}
\item hry a simulácie \emph{(4.1, 4.2, 7.7)}
\item šifrovanie \emph{(7.2)}
\item kompresia \emph{(5.9)}
\item databázy \emph{(6.3, 6.4, 6.5)}
\end{itemize}

Výber úloh je aj na učiteľovi, aby pružne reagoval na časové súvislosti v učebných plánoch nadväzujúcic predmetov. Viac nežiadané je predbiehať predmet, z ktorého úloha čerpá, pretože to so sebou nesie znásobené časové nároky na objasnenie a hrozbu nadobudnutia povrchného obrazu o prebratej oblasti z iného vyučovacieho predmetu. Ďalej uvázame znenia zadaní úloh.

\subsection{Premenné}
\textbf{Premenná} je ako krabička slúžiaca na odkladanie informácií, ktoré si potrebujeme pre vykonanie danej činnosti zapamätať. Podľa účelu sa líšia svojim *dátovým typom*, ktorý sa vytvorí, keď do premennej niečo vložíme (*priradenie*) a určuje to, čo sa vo vnútri nachádza.

Základné stavebné kamene, z ktorých vyskladáme opis zložitejších javov sú:

\begin{itemize}
\itemsep0pt
\item \textbf{Logická hodnota} (\textit{bool}) - Boolean môže mať len dve hodnoty - pravda (\textit{True}) alebo nepravda (\textit{False})
\item \textbf{Celé číslo} (\textit{int}) - Do integer-u ukladáme ľubovolné kladné a záporné celé čísla (napr. \textit{97})
\item \textbf{Desatinné číslo} (\textit{float}) - Líšia sa od celých čísel spôsobom uloženia (napr. \textit{3.14159})
\item \textbf{Reťazec} (\textit{str}) - Označujeme ich úvodzovkami alebo apostrofmi a väčšinou predstavujú text napísaný na klávesnici alebo zobrazený na obrazovke. (napr. \textit{"Učím sa programovať!"})
\end{itemize}

\subsubsection*{1. Pozdrav}
Vytvor program, ktorý ťa po vložení mena pozdraví. Zameň pozdrav a zároveň nechaj program sa rozlučiť.

\begin{code}
Ako sa voláš?: ______
Ahoj ______
\end{code}

\subsubsection*{2. Básnik}
Vytváraš básničky na počkanie. Dnes sa ti ťažko premýšľa nad kreatívnymi textami, tak si chceš ušetriť námahu tým, že budeš meniť len rým.

\begin{code}
Napíš slovo, ktoré sa rýmuje so slovom strach: _____
Tu je báseň:
Z počítačov mával som vždy strach
teraz som však šťastný ako _____.
\end{code}

\subsubsection*{3. Pozvánka}
Každému kamarátovi chceš poslať pozvánku na svoju narodeninovú oslavu. Okrem mena v správe potrebuješ meniť aj čas konania oslavy (nie všetci chodia načas), vec, ktorú priniesie a jedlo, ktoré bude mať prichystané.

\begin{code}
Meno kamaráta: _____
Čas oslavy: _____
Prines: _____
Jedlo: ______

Ahoj _____,
pozývam ťa na moju narodeninovú oslavu, ktorá sa bude konať 12.4. o _____. Nezabudni priniesť _____ a pekný darček. Na večeru ťa čaká _____ a samozrejme lahodná torta. Teším sa na teba! :)
\end{code}

\subsubsection*{4. Prevod jednotiek teploty}
Prišiel si na návštevu v Amerike a keď ideš von nevieš ako sa máš obliecť, lebo na teplomere vidíš len stupne Fahrenheita ($F$). Premeň ich na stupne Celzia ($C$).

$$ C = (F - 32) \cdot (5 / 9) $$

\begin{code}
Vonku je °F: _____
Doma by to bolo _____°C.
\end{code}

\subsubsection*{5. Hlboká roklina}
Stojíš na útese nad hlbokým údolím a rozmýšlaš ako odmerať jej hĺbku (*h*). Vtom ťa osvietia tvoje dávne vedomosti z fyziky. Zoberieš do ruky kameň a pustíš ho z ruky do rokliny.
Zároveň spustíš stopky a zmeriaš čas dopadu. Rýchlosť zvuku rachotu pri náraze na zem môžeme zanedbať. Na kameň sa ohybuje sa nadol rovnomerným spomaleným pohybom. Pôsobí naň tiažové zrýchlenie: $g = 9.81$.
$$ h = (gt^2)\;/\; 2 $$

\begin{code}
Čas dopadu kameňa (s): ________
Hĺbka rokliny je potom _______ metrov.
\end{code}

\subsubsection*{6. Vedro s vodou}
Do nádrže z dažďovou vodou napršalo cez noc veľa vody. Jediný spôsob ako zúžitkovať zachytenú vodu je preniesť ju vo vedre valcového tvaru. Naberieme vždy len toľko vody koľko budeme potrebovať, preto je dobré poznať objem vedra. Rozmery vedra dokážeme odmerať pravítkom. Objem valcového vedra $V$ s výškou $v$ a priemerom podstavy $d$ sa vypočíta ako:
$$ V = \pi \cdot (d\;/\;2)^2 \cdot v $$

\begin{code}
Výška vedra (cm): ________
Priemer dna (cm): ________

Do vedra sa zmestí _______ litrov vody.
\end{code}

\subsubsection*{7. Cesta autom}
Plánuješ trasu na výlet autom a chceš zistiť akou rýchlosťou musíte priemerne ísť, aby ste stihli navštíviť všetky miesta a prišli večer včas do hotela.

\begin{code}
Dĺžka cesty (km): ____
Odchod z domu (hodina): ____
Príchod do hotela (hodina): ____

Pôjdete priemerne ____ km/h.
\end{code}

\subsubsection*{8. Kúpalisko}
Začína sa letná sezóna a prevádzka kúpaliska musí pred otvorením plne napustiť bazény v areáli. Všetky sú kvádrového tvaru a poznáme ich rozmery. Zaujíma nás spotrebovaná voda na konkrétny bazén a cena, ktorú za ňu zaplatíme.

\begin{code}
Dĺžka bazéna (m): ____
Šírka bazéna (m): ____
Hĺbka bazéna (m): ____
Hĺbka hladiny od okraja (cm): ____
Cena za m^3 vody v eurách: _____

Na bazén sa minie ____ litrov vody a bude to stáť ____ eur.
\end{code}

\subsubsection*{9. Maľovanie}
Sťahuješ sa s rodičmi do nového bytu a dali ti za úlohu vymalovať si izbu. Myslíš si, že nástroj na rýchle počítanie množstva farby by sa hodil aj profesionálnym maliarom, preto vytvoríš program na vypočítanie plochy stien a stropu bez okna a podlahy.

\begin{code}
Rozmery miestnosti
Dĺžka (cm): ____
Šírka (cm): ____
Výška (cm): ____
Rozmery okna
Šírka (cm): ____
Výška (cm): ____
Výdatnosť farby (m^2/kg): ____

Maľovať budeš plochu ____ m^2. Kúp ____ kg farby.
\end{code}

\subsubsection*{11. Pokladnička}
Do banky vložíme peniaze (vklad)  a každý rok sa nám na nich pripočítava úrok. V banke necháme peniaze určitý počet rokov. Vypočítajte ako sumu dostaneme pri výbere.

Napíš program, ktorý nám umožní na vstupe napísať rôzne sumy peňazí, úrokové miery a obdobie sporenia a na výstupe vypíše nasporenú sumu. Vyberte si, či použijete jednoduché alebo zložené úročenie - vzorec nájdite na internete alebo v zošite matematiky. Spustenie programu môže vyzerať nasledovne (pri jednoduchom úročení):

\begin{code}
Vklad (euro): ____
Úrok (%): ___
Dĺžka sporenia (rok): ___

Na konci sporenia si v banke vyberieš _____ eur.
\end{code}

\subsubsection*{12. Chemikálie}
Napíš program na výpočet zmiešavania dvoch roztokov. Každý roztok je opísaný svojou hmotnosťou ($m$) v gramoch a hmotnostným zlomkom rozpustenej látky v rozpúštadle ($w$) v percentách.
 
Roztoky vo vzorci sú rozlíšené dolným indexom, napr. . Na vstupe sú zadané všetky premenné s indexami 1 a 2. Premenné s indexom 3 je potrebné vypočítať v programe.  Percentá je potrebné prerátať na pomer z celku vydelením 
100-mi.

Na výpočet v programe využiješ rovnice:
$$m_3 = m_1 + m_2$$
$$m_3 \cdot w_3 = m_1 \cdot w_1 +  m_2 \cdot w_2$$

\begin{code}
m1 (hmotnosť roztoku č.1)? ___
w1 (hmotnostný zlomok roztoku č.1)? ___
m2 (hmotnosť roztoku č.2)? ___
w2 (hmotnostný zlomok roztoku č.2)? ___

Výsledný roztok má hmotnosť ___ g.
Hmotnostný zlomok rozpustenej látky je ___ %.
\end{code}


\subsubsection*{13. Brzdenie}
V poslednej dobe je na trati viacej nebezpečných zrážok. Rušňovodiči ťa požiadali, aby si zistil ako rýchlo pred prekážkou dokáže vlaková súprava zastaviť pri danej rýchlosti.

\begin{itemize}
\itemsep0pt
\item Kinetická energia pohybujúceho sa vlaku (práca potrebná na zabrzdenie): $ W = E_k = \frac{1}{2} \cdot m \cdot v^2 $
\item Brzdná dráha pri brzdnej sile $F_b$: $ s = \frac{W}{F_b \cdot m} $
\item Čas potrebný na zastavenie vlaku pri rovnomernom spmalenom pohybe: $ t = \sqrt{\frac{2 \cdot s}{F / m}} $
\end{itemize}

\begin{code}
Vlaková súprava
- Rýchlosť (km/h): ____
- Hmotnosť lokomotívy (t): ____
- Hmotnosť vagóna (t): ____
- Počet vagónov: ____
- Počet miest na vagón: ____
- Zaplnenosť vlaku (%): ____
- Brzdná sila (N/t): ____

V rýchlosti ____ km/h zabrzdí súprava s hmotnosťou ____ t na vzdialnosť _____ m a bude to trvať ____ s.
\end{code} 

\subsection{Podmienky}
\textbf{Podmienky} sú ako križovatky na ceste. Podľa toho kam chceme ísť, sa rozhodneme, ktorou cestou pôjdeme ďalej. Aby sme sa uistili, že máme ten správny smer (\emph{vetva podmienky}) pýtame sa vždy logickú otázku. Otázka používa údaje uložené v premenných.

\subsubsection*{1. Heslo}
Tvoj dom na strome už vykradlo pár nezvaných návštevníkov. Vymyslel si preto spôsob ako dovoliť návštevu len overeným osobám. Tie musia poznať tajné heslo. Napíš program, ktorý slovne privíta členov a odoženie zlodejov.

\begin{tabular}{@{}p{0.15\linewidth}p{0.75\linewidth}}
\textbf{\small Vstup:} &
\vspace{-3em}
\begin{code}
Stoj! Povedz Heslo!
? @\fbox{\phantom{vstup}}@
\end{code}
\end{tabular}

\vspace{-2em}
\begin{tabular}{@{}p{0.15\linewidth}p{0.75\linewidth}}
\textbf{\small Výstup:} &
\vspace{-3em}
\begin{code}
Vstúp, priateľ 
(alebo Zmizni kade ľahšie)
\end{code}
\end{tabular}
\vspace{-2em}

\subsubsection*{2. Najväčšie číslo}
Na lúke sa hrajú šípky. Hrači si zapisujú dosiahnuté skóre na tabuľu. Dnes proti sebe hrali v partii traja protihráči. Napíš program, ktorý označí hráča s najväčším získaným počtom bodov.

\begin{tabular}{@{}p{0.15\linewidth}p{0.75\linewidth}}
\textbf{\small Vstup:} &
\vspace{-3em}
\begin{code}
1.skóre: @\fbox{\phantom{vstup}}@
2.skóre: @\fbox{\phantom{vstup}}@
3.skóre: @\fbox{\phantom{vstup}}@
\end{code}
\end{tabular}

\vspace{-2em}
\begin{tabular}{@{}p{0.15\linewidth}p{0.75\linewidth}}
\textbf{\small Výstup:} &
\vspace{-3em}
\begin{code}
Najväčie skóre @\fbox{\phantom{vstup}}@ bodov má @\fbox{\phantom{vstup}}@ hráč
\end{code}
\end{tabular}
\vspace{-2em}


\subsubsection*{3. Vhodné oblečenie}
Módni poradcovia vyšli z módy a ich prácu prebrali počítače. Na základe počasia a príležitosti odporúčajú vhodný outfit. Vymysli pár tipov pre rôzne situácie a začni radiť.

\begin{tabular}{@{}p{0.15\linewidth}p{0.75\linewidth}}
\textbf{\small Vstup:} &
\vspace{-3em}
\begin{code}
Ako je vonku?: @\fbox{\phantom{vstup}}@
Kam ideš?: @\fbox{\phantom{vstup}}@
\end{code}
\end{tabular}

\vspace{-2em}
\begin{tabular}{@{}p{0.15\linewidth}p{0.75\linewidth}}
\textbf{\small Výstup:} &
\vspace{-3em}
\begin{code}
Určite si nezabudni @\fbox{\phantom{vstup}}@ a tiež si vezmi @\fbox{\phantom{vstup}}@.
\end{code}
\end{tabular}
\vspace{-2em}

\subsubsection*{4. Morský vánok}
Kapitán plachetnice na otvorenom oceáne musí mať vždy prehľad odkiaľ fúka vietor, aby dokormidloval do vytúženého cieľa. Príliš silné závany vetra môžu byť nebezpečné pre posádku. Rozthať polámať lodné sťažne, potrhať plachty, či zaplaviť palubu. 

Cez rádio dostáva plavidlo každý deň správy o predpovedi sily vetra v Beafortovej stupnici. Sila vetra je ňou vyjadrená do dvanástich stupňov od bezvetria až po orkán. Napíšte program, ktorý kapitánu vysvetlí stupeň vetra. Podľa stupnice určíme jeho pomenovanie, rýchlosti v námorných uzloch a očakávateľnej výšky vĺn.

\begin{tabular}{@{}p{0.15\linewidth}p{0.75\linewidth}}
\textbf{\small Vstup:} &
\vspace{-3em}
\begin{code}
Sila vetra na Beaufortovej stupnici: @\fbox{\phantom{12}}@
\end{code}
\end{tabular}

\vspace{-2em}
\begin{tabular}{@{}p{0.15\linewidth}p{0.75\linewidth}}
\textbf{\small Výstup:} &
\vspace{-3em}
\begin{code}
Vietor sa nazýva @\fbox{\phantom{vstup}}@.
Vietor má rýchlosť @\fbox{\phantom{vstup}}@ kt.
Očakávaná výška vĺn je @\fbox{\phantom{vstup}}@ m.
\end{code}
\end{tabular}
\vspace{-2em}


\subsubsection*{5. Pokazený rozpis}
Továreň na železnú rudu dostala nový časový rozpis vylepšeného technologického procesu. Spracovanie zvyčajne trvá dlhšie ako hodinu. Nehodí sa im teda mať časy napísané iba v minútach. Rozpíš programom minúty na dni, hodiny, minúty pre jednoduchšie čítanie rozpisu. Vynechaj nepotrebné časové údaje.

\begin{tabular}{@{}p{0.15\linewidth}p{0.75\linewidth}}
\textbf{\small Vstup:} &
\vspace{-3em}
\begin{code}
Trvanie (min.): @\fbox{\phantom{vstup}}@
\end{code}
\end{tabular}

\vspace{-2em}
\begin{tabular}{@{}p{0.15\linewidth}p{0.75\linewidth}}
\textbf{\small Výstup:} &
\vspace{-3em}
\begin{code}
= @\fbox{\phantom{vstup}}@ d. @\fbox{\phantom{vstup}}@ hod. @\fbox{\phantom{vstup}}@ min.
\end{code}
\end{tabular}
\vspace{-2em}


\subsubsection*{6. Hovoriaca kalkulačka}
Výpočty neboli nikdy väčšia zábava. Teda aspoň s kalkulačkou, ktorá namiesto čudných matematických čmáraníc hovorí ľudskou rečou. Vytvor program pre kalkulačku, ktorá si vypýta dve čísla. Tie bude ich vedieť sčítať alebo odčítať podľa slovného pokynu.

\begin{tabular}{@{}p{0.15\linewidth}p{0.75\linewidth}}
\textbf{\small Vstup:} &
\vspace{-3em}
\begin{code}
Som hovorica kalkulačka a rada počítam!
Povedz mi prvé číslo: @\fbox{\phantom{vstup}}@
Potrebujem ďašie číslo: @\fbox{\phantom{vstup}}@
Chceš ich sčítať alebo odčítať: @\fbox{\phantom{vstup}}@ (sčítať alebo odčítať)
\end{code}
\end{tabular}

\vspace{-2em}
\begin{tabular}{@{}p{0.15\linewidth}p{0.75\linewidth}}
\textbf{\small Výstup:} &
\vspace{-3em}
\begin{code}
Výsledok tvojho príkladu: @\fbox{\phantom{vstup}}@ (plus alebo mínus) @\fbox{\phantom{vstup}}@ je @\fbox{\phantom{vstup}}@.
\end{code}
\end{tabular}
\vspace{-2em}

\subsubsection*{7. Chaos v lístkoch}
Vyznať sa v linkách mestskej hromadnej dopravy si vyžaduje dlhoročné skúsenosti. Treba oplývať aj riadnou dávkou trpezlivosti. Ľahko sa nám stane, že omylom nasadneme do autobusu a hneď sa vydáme na okružnú jazdu po siedmich divoch sídliska. Horší zážitok je stretnutie revízora po zistení, že máme nesprávny lístok alebo že nemáme žiaden ...

Postávaš pri automate na lístky a nevieš sa vysomáriť z množstva časov a zón v ponuke. Napíš program, ktorý podľa počtu zónu a trvania ceny vypíše cenu zľavneného lístka. Nájdi na internete aktuálnu tarifu MHD v tvojom meste.

\begin{tabular}{@{}p{0.15\linewidth}p{0.75\linewidth}}
\textbf{\small Vstup:} &
\vspace{-3em}
\begin{code}
Popíš mi svoju cestu s MHD
Koľko zón prejdeš?: @\fbox{\phantom{vstup}}@
Koľko minút má trvať cesta?: @\fbox{\phantom{vstup}}@
\end{code}
\end{tabular}

\vspace{-2em}
\begin{tabular}{@{}p{0.15\linewidth}p{0.75\linewidth}}
\textbf{\small Výstup:} &
\vspace{-3em}
\begin{code}
Zlavnený lístok stojí @\fbox{\phantom{vstup}}@ eur.
\end{code}
\end{tabular}
\vspace{-2em}


\subsubsection*{8. Kvadratická rovnica}
Matematika v škole dokáže byť poriadna otrava. Hlavne, keď od rána do večera nič iné nerobíš ako počítaš príklady na kvadratické rovnice. ,,Načo mám ten počítač'', pomyslíš si večer vo svetle stolenj lampy. Pre zadané koeficienty $a$, $b$, $c$ predpisu $ax^2 + bx + c = 0$ napíš program, ktorý vypočíta jej korene a vrchol paraboly.

\begin{tabular}{@{}p{0.15\linewidth}p{0.75\linewidth}}
\textbf{\small Vstup:} &
\vspace{-3em}
\begin{code}
Koeficienty kvadratickej rovnice:
a = @\fbox{\phantom{vstup}}@
b = @\fbox{\phantom{vstup}}@
c = @\fbox{\phantom{vstup}}@
\end{code}
\end{tabular}

\vspace{-2em}
\begin{tabular}{@{}p{0.15\linewidth}p{0.75\linewidth}}
\textbf{\small Výstup:} &
\vspace{-3em}
\begin{code}
@\fbox{\phantom{a}}@x^2 + @\fbox{\phantom{b}}@x + @\fbox{\phantom{c}}@ = 0
x1 = @\fbox{\phantom{abc}}@
x2 = @\fbox{\phantom{abc}}@
V[@\fbox{\phantom{abc}}@; @\fbox{\phantom{abc}}@]
\end{code}
\end{tabular}
\vspace{-2em}


\subsubsection*{9. Trojuholníky}
Trojuholník je mýtická bytosť, o ktorej je vždy treba zistiť. Nesmieme použiť pravítko, lebo to by nás čakala príliš jednoduchá výzva. Veď bez rysovania zístíme o tejto trojcípej paráde všeličo. Hoci aj keď jej chýbajú niektoré rozmery.

\begin{enumerate}[label=\alph*)]
\item Ak je to možné, doplň chýbajúce informácie pre ľubovoľný trojuholník (zadaný ako SSS) ako sú dĺžky strán a výšok, veľkosti uhlov, obsah a obvod. Využi trojuholníkovú nerovnosť, sínusovú vetu, kosínusovú vetu a vzorec na výpočet obsahu trojuholníkov.
\item Rozšír program aj pre ostatné vety o trojuholníkoch: SUS, USU, UUS
\end{enumerate}

\begin{tabular}{@{}p{0.15\linewidth}p{0.75\linewidth}}
\textbf{\small Vstup:} &
\vspace{-3em}
\begin{code}
Zadajte strany ľubovolného trojuholníka:
a = @\fbox{\phantom{vstup}}@
b = @\fbox{\phantom{vstup}}@
c = @\fbox{\phantom{vstup}}@
\end{code}
\end{tabular}

\vspace{-2em}
\begin{tabular}{@{}p{0.15\linewidth}p{0.75\linewidth}}
\textbf{\small Výstup:} &
\vspace{-3em}
\begin{code}
Strany: a = @\fbox{\phantom{abc}}@; b = @\fbox{\phantom{abc}}@; c = ___
Uhly: alpha = @\fbox{\phantom{abc}}@°; beta = @\fbox{\phantom{abc}}@°; gamma = @\fbox{\phantom{vstup}}@°
Výšky: v(a) = @\fbox{\phantom{abc}}@; v(b) = @\fbox{\phantom{abc}}@; v(c) = @\fbox{\phantom{abc}}@
O = @\fbox{\phantom{abc}}@
S = @\fbox{\phantom{abc}}@
Trojuholník je: @\fbox{\phantom{abc}}@, @\fbox{\phantom{abc}}@
\end{code}
\end{tabular}
\vspace{-2em}

\subsection{Cykly}
Obrovský potenciál počítačov tkvie v bezchybnom neúnavnom vykonávaní presne zadaných inštrukcií. \underline{\textbf{Cykly}} umožňujú opakovať rovnaký postup ľubovoľný počet krát a tým efektívne odstraňovať rutinnú prácu.


\subsubsection*{1. 100-krát napíš}
Za vyrušovanie na hodinách sa stalo populárnym trestom ručné prepisovanie mravoučnej vety stokrát. Stalo sa to tak neznesiteľné, že si zhotovil robota na pomoc záškodníkom. Chýbajú mu len príkazy, čo má vlastne robiť.

\begin{tabular}{@{}p{0.15\linewidth}p{0.75\linewidth}}
\textbf{\small Vstup:} &
\vspace{-3em}
\begin{code}
Musím napísať: @\fbox{\phantom{vstup}}@
Toľkoto krát: @\fbox{\phantom{vstup}}@
\end{code}
\end{tabular}

\vspace{-2em}
\begin{tabular}{@{}p{0.15\linewidth}p{0.75\linewidth}}
\textbf{\small Výstup:} &
\vspace{-3em}
\begin{code}
@\fbox{\phantom{vstup}}@
@\fbox{\phantom{vstup}}@
@\fbox{\phantom{vstup}}@
...
\end{code}
\end{tabular}
\vspace{-2em}


\subsubsection*{2. Hodnotenie}
Filmoví a gastonomickí kritici zavŕšia namáhavý deň udelením číselného skóre k ich recenziam. Pre lepší efekt v časopise potrebujú nakresliť hviezdničky namiesto čísla. Pomôž im programom.

\begin{tabular}{@{}p{0.15\linewidth}p{0.75\linewidth}}
\textbf{\small Vstup:} &
\vspace{-3em}
\begin{code}
Skóre: @\fbox{5}@
\end{code}
\end{tabular}

\vspace{-2em}
\begin{tabular}{@{}p{0.15\linewidth}p{0.75\linewidth}}
\textbf{\small Výstup:} &
\vspace{-3em}
\begin{code}
@\fbox{\textit{*****}}@
\end{code}
\end{tabular}
\vspace{-2em}


\subsubsection*{3. Pyramída}
Hviezdičky zoskup do tvaru pyramídy zadanej výšky.

\begin{tabular}{@{}p{0.15\linewidth}p{0.75\linewidth}}
\textbf{\small Vstup:} &
\vspace{-3em}
\begin{code}
Výška pyramídy: @\fbox{4}@
\end{code}
\end{tabular}

\vspace{-2em}
\begin{tabular}{@{}p{0.15\linewidth}p{0.75\linewidth}}
\textbf{\small Výstup:} &
\vspace{-3em}
\begin{code}
   *
  ***
 *****
*******
\end{code}
\end{tabular}
\vspace{-2em}

\subsubsection*{4. Smaragd}
Na pyramídu pripoj zo spodu ďaľšiu obrátene, aby vznikol smaragd z hviezdičiek.

\begin{tabular}{@{}p{0.15\linewidth}p{0.75\linewidth}}
\textbf{\small Vstup:} &
\vspace{-3em}
\begin{code}
Veľkosť smaragdu: @\fbox{5}@
\end{code}
\end{tabular}

\vspace{-2em}
\begin{tabular}{@{}p{0.15\linewidth}p{0.75\linewidth}}
\textbf{\small Výstup:} &
\vspace{-3em}
\begin{code}
  *
 ***
*****
 ***
  *
\end{code}
\end{tabular}
\vspace{-2em}

\subsubsection*{5. Duté vnútro}
Nakresli duté pyramídu a smaragd podľa prechádzajúcich úloh.

\begin{tabular}{@{}p{0.15\linewidth}p{0.75\linewidth}}
\textbf{\small Vstup:} &
\vspace{-3em}
\begin{code}
Výška pyramídy: @\fbox{4}@
\end{code}
\end{tabular}

\vspace{-2em}
\begin{tabular}{@{}p{0.15\linewidth}p{0.75\linewidth}}
\textbf{\small Výstup:} &
\vspace{-3em}
\begin{code}
    *
   * *
  *   *
 *******
\end{code}
\end{tabular}
\vspace{-2em}


\subsubsection*{6. Mriežka slov}
Načítaj veľkosť tabuľky a slovo, ktoré sa v nej bude na každom riadku v stĺpci opakovať.

\begin{tabular}{@{}p{0.15\linewidth}p{0.75\linewidth}}
\textbf{\small Vstup:} &
\vspace{-3em}
\begin{code}
Počet riakov a stĺpcov: @\fbox{4}@
Opakovať slovo: @\fbox{ano}@
\end{code}
\end{tabular}

\vspace{-2em}
\begin{tabular}{@{}p{0.15\linewidth}p{0.75\linewidth}}
\textbf{\small Výstup:} &
\vspace{-3em}
\begin{code}
ano ano ano ano
ano ano ano ano
ano ano ano ano
ano ano ano ano
\end{code}
\end{tabular}
\vspace{-2em}


\subsubsection*{7. Rám}
Prvý a posledný riadok a stĺpec bude tvoriť rám pre mriežku slov.

\begin{tabular}{@{}p{0.15\linewidth}p{0.75\linewidth}}
\textbf{\small Vstup:} &
\vspace{-3em}
\begin{code}
Počet riakov a stĺpcov: @\fbox{4}@
Opakovať slovo: @\fbox{ano}@
\end{code}
\end{tabular}

\vspace{-2em}
\begin{tabular}{@{}p{0.15\linewidth}p{0.75\linewidth}}
\textbf{\small Výstup:} &
\vspace{-3em}
\begin{code}
### ### ### ###
### ano ano ###
### ano ano ###
### ### ### ###
\end{code}
\end{tabular}
\vspace{-2em}


\subsubsection*{8. Malá násobilka}
K výbave každého žiaka základnej školy patrí tabuľky malej násobilky. Vytvor takúto tabuľku obsahujúcu každý násobok od 1x1 po 10x10, aby si pomohol všetkým malým matematikom.


\vspace{-2em}
\begin{tabular}{@{}p{0.15\linewidth}p{0.75\linewidth}}
\textbf{\small Výstup:} &
\vspace{-3em}
\begin{code}
   1   2   3   4   5   6   7   8   9  10
   2   4   6   8  10  12  14  16  18  20
   3   6   9  12  15  18  21  24  27  30
   4   8  12  16  20  24  28  32  36  40
   5  10  15  20  25  30  35  40  45  50
   6  12  18  24  30  36  42  48  54  60
   7  14  21  28  35  42  49  56  63  70
   8  16  24  32  40  48  56  64  72  80
   9  18  27  36  45  54  63  72  81  90
  10  20  30  40  50  60  70  80  90 100
\end{code}
\end{tabular}
\vspace{-2em}


\subsubsection*{9. Sporenie}
Na letnej brigáde si zarobil peniaze, ktoré chceš usporiť. Porovnáš ponuky bánk a hľadáš najvýhodnejší plán. Vytvor si sporiacu kalkulačku, ktorá na základe nemenného počiatčného vkladu, ročnej úrokovej sadzby, typu úročenia a žiadanej konečnej sumy, vypíše vývoj tvojich finančných prostriedkov do budúcnosti.

\begin{tabular}{@{}p{0.15\linewidth}p{0.75\linewidth}}
\textbf{\small Vstup:} &
\vspace{-3em}
\begin{code}
Počiatočný vklad v eurách: @\fbox{\phantom{vstup}}@
Úroková sadzba p.a. v %: @\fbox{\phantom{vstup}}@
Typ úročenia (jednoduché / zložené): @\fbox{\phantom{vstup}}@
Žiadaná suma v eurách: @\fbox{\phantom{vstup}}@
\end{code}
\end{tabular}

\vspace{-2em}
\begin{tabular}{@{}p{0.15\linewidth}p{0.75\linewidth}}
\textbf{\small Výstup:} &
\vspace{-3em}
\begin{code}
Rok      Suma						Úrok
  1.		@\fbox{\phantom{vstup}}@ Eur	@\fbox{\phantom{vstup}}@ Eur
  2.		@\fbox{\phantom{vstup}}@ Eur	@\fbox{\phantom{vstup}}@ Eur
\end{code}
\end{tabular}
\vspace{-2em}


\subsection{Náhodné čísla}
Pri tvorbe simulácií sú náhodné čísla nepostrádateľné. Umožňujú vniesť variabilitu a rôznorodosť do inak statických scén. Nesmierne poslúžia v hrách, kde dovoľujú modelovať napríklad pravdepodobnosť výskytu monštier, či pokladov.


\subsubsection*{1. Hádzanie kockou}
Vytvorte simuláciu hodu kockou. Po stlačení klávesy Enter sa nakreslí kocka s padnutým číslom.

\begin{code}
HOĎ<ENTER>
+-------+
| #   # |
|   #   |
| #   # |
+-------+
\end{code}

\subsubsection*{2. Hádaj číslo}
Náhodne vyber číslo s rozsahu medzi 0 a 100 a nechaj hráča hádať dokým neuhádne. Pri tom mu poskytni nápovedy, či je jeho tip priveľa alebo primalo. Zakomponuj rôzne obtiažnosti s možnosťou nastavenia rozsahu alebo maximálnym počtom tipov.

\begin{code}
Hádaj číslo: 8
Málo
Hádaj číslo: 18
Veľa
Hádaj číslo: 13
Výborne. Uhádol si!
\end{code}

\subsubsection*{3. Opakovanie násobilky}
Vďaka tvojej tabuľke malej násobilky sa malý školáci mohli naučiť násobiť. Ako dobre to vedia, musíš teraz odtestovať. Vygeneruj dve čísla od 1 do 10 do príkladu na násobenie. Over správnosť žiačikovej odpovede.

\begin{code}
Koľko je ____ x _____?
= _____
Správne - len tak ďalej / Nesprávne - hádaj znovu
Chceš ďalší príklad?
\end{code}
 

\subsection{Reťazce a zoznamy}
\underline{\textbf{Zoznam}} (tiež aj \underline{\textbf{Pole}}) je množina údajov zaznamenaných spolu pod jedným menom. Každý údaj poľa sa nazýva \underline{\textbf{prvok}} a poradie jeho pozície sa nazýva \underline{\textbf{index}}. \underline{\textbf{Reťazce}} sa správajú podobne ako zoznamy, ale ich prvkami sú jednotlivé \underline{\textbf{znaky}}.

\subsubsection*{1. Vymeň písmeno}
Niekto ti posiela správy s diakritikou, ale po ceste sa vždy prekrúti jedno písmeno. Texty obsahujú aj pekné básne, ktoré si chceš vytlačiť a pripnúť na nástenku. Pokazený znak však kazí celkový dojem z diela. Zameň zadané chybné písmeno v celom reťazci.

\begin{tabular}{@{}p{0.15\linewidth}p{0.75\linewidth}}
\textbf{\small Vstup:} &
\vspace{-3em}
\begin{code}
Správa: @\fbox{\phantom{dlhý text}}@
Za chybné písmeno: @\fbox{\phantom{a}}@
Vymeň: @\fbox{\phantom{b}}@
\end{code}
\end{tabular}

\vspace{-2em}
\begin{tabular}{@{}p{0.15\linewidth}p{0.75\linewidth}}
\textbf{\small Výstup:} &
\vspace{-3em}
\begin{code}
Opravené!
@\fbox{\phantom{dlhý text}}@
\end{code}
\end{tabular}
\vspace{-2em}


\subsubsection*{2. Cenzúra}
Prišla tvrdá cenzúra s nariadením, že nikto už nesmie vidieť žiadnu samohlásku. Nahraď každý priestupok vo vstupnom texte iným špeciálnym znakom.

\begin{tabular}{@{}p{0.15\linewidth}p{0.75\linewidth}}
\textbf{\small Vstup:} &
\vspace{-3em}
\begin{code}
Správa: @\fbox{Ja som tvoj kamarat}@
Samohlásku nahraď: @\fbox{*}@
\end{code}
\end{tabular}

\vspace{-2em}
\begin{tabular}{@{}p{0.15\linewidth}p{0.75\linewidth}}
\textbf{\small Výstup:} &
\vspace{-3em}
\begin{code}
Cenzurované: @\fbox{J* s*m tv*j k*m*r*t}@
\end{code}
\end{tabular}
\vspace{-2em}


\subsubsection*{3. Počítanie slov}
Do redakcie miestnych novín chodia dennodenne články, vtipy, poviedky a príbehy zo života od verných čitateľov. Aby mohli byť uverejnené potrebujú sa zmestiť do vyhradeného priestoru. Vypíš počet znakov, slov, viet a normostrán (=\emph{1800 znakov}), aby sa príhody rýchlejšie rozšírili medzi ľudí.

\begin{tabular}{@{}p{0.15\linewidth}p{0.75\linewidth}}
\textbf{\small Vstup:} &
\vspace{-3em}
\begin{code}
Článok: @\fbox{\phantom{Dlhý text článku s veľa slovami}}@
\end{code}
\end{tabular}

\vspace{-2em}
\begin{tabular}{@{}p{0.15\linewidth}p{0.75\linewidth}}
\textbf{\small Výstup:} &
\vspace{-3em}
\begin{code}
Znaky: @\fbox{\phantom{123}}@
Slová: @\fbox{\phantom{123}}@
Vety: @\fbox{\phantom{123}}@
Normostrany: @\fbox{\phantom{123}}@
\end{code}
\end{tabular}
\vspace{-2em}


\subsubsection*{4. Najdlhšie slovo}
Debatný spolok usporiadal súťaž o nájdenie najdlhšieho slova, ktoré sa kedy vyskytlo v historických prejavoch. Zaujali ťa odmeny, ale nechce sa ti prehrabávať knižnicou starých záznamníkov. Prácu si preto uľahčíš. Nájdi najdlhšie slovo v ľubovoľnom reťazci.

\begin{tabular}{@{}p{0.15\linewidth}p{0.75\linewidth}}
\textbf{\small Vstup:} &
\vspace{-3em}
\begin{code}
Rečnícky prejav: @\fbox{\phantom{Dlhý text článku s veľa slovami}}@
\end{code}
\end{tabular}

\vspace{-2em}
\begin{tabular}{@{}p{0.15\linewidth}p{0.75\linewidth}}
\textbf{\small Výstup:} &
\vspace{-3em}
\begin{code}
Najdlhšie slovo v ňom: @\fbox{\phantom{slovo}}@
\end{code}
\end{tabular}
\vspace{-2em}

\subsubsection*{5. Výskyt písmen}
Dlho do noci čítaš časopisy o umelej inteligencii a fascinuje ťa jej schopnosť rozprávať sa s človekom. Na vytvorenie viet na danú tému potrebuje mať prehľad o percentuálnom výskyte hlások v texte. Spočítaj a vypíš zoznam početnosti písmen v reťazci.

\begin{tabular}{@{}p{0.15\linewidth}p{0.75\linewidth}}
\textbf{\small Vstup:} &
\vspace{-3em}
\begin{code}
Článok: @\fbox{\phantom{Dlhý text článku s veľa slovami}}@
\end{code}
\end{tabular}

\vspace{-2em}
\begin{tabular}{@{}p{0.15\linewidth}p{0.75\linewidth}}
\textbf{\small Výstup:} &
\vspace{-3em}
\begin{code}
A: @\fbox{23.2}@ %
B: @\fbox{11.5}@ %
C: @\fbox{8.9}@ %
...
Z: @\fbox{0.3}@ %
\end{code}
\end{tabular}
\vspace{-2em}


\subsubsection*{6. Histogram}
Počas predošlého pokusu s početnosťou písmen si všimneš, že každé ďalšie písmeno v zozname sa objavuje oveľa menej než očakávaš. Vykresli hviezdičky namiesto počtu percent. Over si tak svoje pozorovanie graficky.

\begin{tabular}{@{}p{0.15\linewidth}p{0.75\linewidth}}
\textbf{\small Vstup:} &
\vspace{-3em}
\begin{code}
Článok: @\fbox{\phantom{Dlhý text článku s veľa slovami}}@
\end{code}
\end{tabular}

\vspace{-2em}
\begin{tabular}{@{}p{0.15\linewidth}p{0.75\linewidth}}
\textbf{\small Výstup:} &
\vspace{-3em}
\begin{code}
A: @\fbox{****}@
E: @\fbox{*******}@
I: @\fbox{****}@
...
X: @\fbox{*}@
\end{code}
\end{tabular}
\vspace{-2em}


\subsubsection*{7. Nákupný košík}
Na veľkých nákupoch sa často zíde prehľadný zoznam s tým, čo doma treba. Pýtaj si položky s ich cenami až kým sa nerozhodneš, že máš spísané všetko. Zobraz prehľadnú orámovanú tabuľku s údajmi, podobne ako na pokladničnom bločku. To sú názov tovaru, DPH tovaru, cena tovaru s DPH a cena spolu za nákup.

\begin{tabular}{@{}p{0.15\linewidth}p{0.75\linewidth}}
\textbf{\small Vstup:} &
\vspace{-3em}
\begin{code}
Čo kúpiť?: @\fbox{\phantom{vstup}}@
Cena @\fbox{\phantom{vstup}}@?: @\fbox{\phantom{vstup}}@
\end{code}
\end{tabular}

\vspace{-2em}
\begin{tabular}{@{}p{0.15\linewidth}p{0.75\linewidth}}
\textbf{\small Výstup:} &
\vspace{-3em}
\begin{code}
+----------+--------+--------------+
| Tovar    |  DPH   |  Cena s DPH  |
+----------+--------+--------------+
| Chlieb   |  0,20  |      0,98    |
+----------+--------+--------------+
|    ...   |  ...   |     ...      |
+----------+--------+--------------+
| CELKOM   |  0,20  |      0,98    |
+----------+--------+--------------+
\end{code}
\end{tabular}
\vspace{-2em}

\subsubsection*{8. Akronym}
SMS-ky rapídne zdraželi. Napadlo ti, že bude lepšie posielať slovné spojenia ako skratky. Zo zadaných slov vytvor akronym, ktorý vznikne ponechaním len začiatočných písmen každého slova.

\begin{tabular}{@{}p{0.15\linewidth}p{0.75\linewidth}}
\textbf{\small Vstup:} &
\vspace{-3em}
\begin{code}
Slovné spojenie: @\fbox{Slovenské národné divadlo}@
\end{code}
\end{tabular}

\vspace{-2em}
\begin{tabular}{@{}p{0.15\linewidth}p{0.75\linewidth}}
\textbf{\small Výstup:} &
\vspace{-3em}
\begin{code}
Skratka: @\fbox{SND}@
\end{code}
\end{tabular}
\vspace{-2em}


\subsubsection*{9. Veľa opakovania}
Roboti rozvážajú pizzu po meste. Popri tom si zapisujú zmenu smeru pre postupné vylepšovanie trás k častým zákazníkom. Keďže sa firme darí, nachodili roboti toho už riadne veľa. Všetky záznamy o ich cestách sa im ani nezmestia do pamäti. Všimneš si, že si značia každý jeden krok, čiže sa často opakujú. Nahraď postupnosť za sebou idúceho písmena, písmenom a počtom jeho výskytov.

\begin{tabular}{@{}p{0.15\linewidth}p{0.75\linewidth}}
\textbf{\small Vstup:} &
\vspace{-3em}
\begin{code}
Cesta robota: @\fbox{NNNNNNSSSSSSSSSSSWWWWNNN}@
\end{code}
\end{tabular}

\vspace{-2em}
\begin{tabular}{@{}p{0.15\linewidth}p{0.75\linewidth}}
\textbf{\small Výstup:} &
\vspace{-3em}
\begin{code}
Skomprimované: @\fbox{6N11S4W3N}@
\end{code}
\end{tabular}
\vspace{-2em}

\subsection{Súbory}
\underline{\textbf{Súbor}} je zoskupením súvisiacich údajov, ktoré sú uložené na disku počítača. Oproti načítavaniu vstupu z klávesnice majú výhodu hlavne pri spracovaní a uchovaní veľkého množstva dát. Súbory sa dajú: \underline{vytvoriť} alebo \underline{vymazať}, \underline{otvoriť} alebo \underline{zatvoriť}, \underline{čítať} alebo \underline{zapisovať}.

Podľa typu uchovávaných údajov (označované \underline{\textbf{príponou}}), súbory rozdeľujeme na:
\begin{itemize}[noitemsep]
\item \textbf{Textové súbory} - .txt, .csv, .html, .py
\item \textbf{Obrazové súbory} - .bmp, .png, .jpg, .gif, .svg
\item \textbf{Zvukové súbory} - .wav, .mp3, .midi
\item \textbf{Video súbory} - .avi, .mp4, .mkv
\item \textbf{Spustiteľné súbory} - .exe
\end{itemize}
V tejto kapitole budeme pre jednoduchosť pracovať s textovými súbormi.

\subsubsection*{1. Prepisovanie}
Príde ti zbytočne prepisovať dlhé články na vstup programu a vždy sa pomýliš. Načítaj články pre každú úlohu z predošlej kapitoly zo súboru. Uprav programy tak, aby si najprv vypýtali názov súboru. V úlohe ,,veľa opakovania'' ulož záznam o ceste robota do nového súboru.


\subsubsection*{2. Turistika}
Na víkend sa črtajú ideálne podmienky na horskú turistiku. Nenecháš nič na náhodu a pripravíš si detailný plán s výškovým profilom trasy. Na každých desať metrov trasy si do súboru poznačíš nadmorskú výšku z mapy. Zisti celkové stúpanie a klesanie počas celého výletu spolu s najvyššou a najnižšou nadmorskou výškou. Vypíš aj celkovú dĺžku túry v kilometroch a trvanie prechodu horami v hodinách.

\begin{tabular}{@{}p{0.2\linewidth}p{0.7\linewidth}}
\textbf{\small Obsah súboru:} &
\vspace{-3em}
\begin{code}
348
351
379
384
395
401
396
\end{code}
\end{tabular}

\vspace{-2em}
\begin{tabular}{@{}p{0.2\linewidth}p{0.7\linewidth}}
\textbf{\small Vstup:} &
\vspace{-3em}
\begin{code}
Trasa je v súbore s názvom: @\fbox{\phantom{vstup}}@
\end{code}
\end{tabular}

\vspace{-2em}
\begin{tabular}{@{}p{0.2\linewidth}p{0.7\linewidth}}
\textbf{\small Výstup:} &
\vspace{-3em}
\begin{code}
Trasa: @\fbox{0.140 km}@ - @\fbox{0}@ h @\fbox{21}@ min
Stúpanie: @\fbox{53}@ m
Klesanie: @\fbox{40}@ m
Najnižšie miesto trasy: @\fbox{361}@ m
Najvyššie miesto trasy: @\fbox{401}@ m
\end{code}
\end{tabular}
\vspace{-2em}


\subsubsection*{3. Vedomostný kvíz}
Bifľovanie ti vôbec nepríde prínosné. Keby existoval spôsob, akým si opakovanie učiva spríjemniť. Včera si zo smútku nad vidinou takto premárneného času, pri jedení čokolády a čipsov, pozeral kvízovú reláciu. Prišlo ti to neuveriteľne poučné. Polož náhodnú otázku s možnostami zo súboru kvízových otázok a bodovo ohodnoť správnu odpoveď. Všetky kvízové otázky s možnosťami sa však nezmestia do pamäti programu. Náhodnu otázku vyber priamo zo súboru.

\begin{tabular}{@{}p{0.2\linewidth}p{0.7\linewidth}}
\textbf{\small Obsah súboru:} &
\vspace{-3em}
\begin{code}
Otázka: V ktorom roku začala Francúzska revolúcia?
  A: 1763
  B: 1813
  C: 1789
  D: 1654
Odpoveď: C
Otázka: Al2O3 je?
  A: hydroxid vápenatý
  B: oxid hlinitý
  C: hydroxid sodný
Odpoveď: B
\end{code}
\end{tabular}

\vspace{-2em}
\begin{tabular}{@{}p{0.2\linewidth}p{0.7\linewidth}}
\textbf{\small Kvíz:} &
\vspace{-3em}
\begin{code}
Súbor s kvízovými otázkami: @\fbox{kviz.txt}@
Kvízové otázky pripravené. Ideme na to!
V ktorom roku sa začala Francúzska revolúcia?
A: 1763
B: 1813
C: 1789
D: 1654
Aká je správna odpoveď?: @\fbox{C}@
Správne! Máš 1 bodov. 
(alebo) Nabudúce si to lepšie premysli.
\end{code}
\end{tabular}
\vspace{-2em}


\subsubsection*{4. Narodeniny}
Darčeky k narodeninám zvykneš kupovať na poslednú chvílu. Potrebuješ mať prehľad aspoň na mesiac dopredu, kto bude mať narodeniny, aby si stihol vybrať niečo výnimočné. Zo súboru načítaj ľudí, ktorí majú sviatok v požadovaný mesiac v roku.

\begin{tabular}{@{}p{0.2\linewidth}p{0.7\linewidth}}
\textbf{\small Obsah súboru:} &
\vspace{-3em}
\begin{code}
Jožko Mrkvička, 15.3.2002
Katka Krátka, 2.7.1993
Martinko Klingáč, 12.11.1995
Iveta Novotná, 27.2.2001
\end{code}
\end{tabular}

\vspace{-2em}
\begin{tabular}{@{}p{0.2\linewidth}p{0.7\linewidth}}
\textbf{\small Vstup:} &
\vspace{-3em}
\begin{code}
Zobraz narodeniny pre mesiac v roku: @\fbox{3.2019}@
\end{code}
\end{tabular}

\vspace{-2em}
\begin{tabular}{@{}p{0.2\linewidth}p{0.7\linewidth}}
\textbf{\small Výstup:} &
\vspace{-3em}
\begin{code}
Narodeniny: @\fbox{Marec 2019}@
@\fbox{15.3. - Jožko Mrkvička - 17 rokov}@
\end{code}
\end{tabular}
\vspace{-2em}


\subsubsection*{5. Cestovné poriadky}
Z celoštátneho rýchlika prestupujú cestujúci v okresných mestách na miestne autobusy. Podľa času odchodu a trvania cesty zisti, ktorý autobus stihnú. Vypíš najbližší spoj s najmenším čakaním medzi vlakom a autobusom. Daj pozor! Prvý časový údaj v riadku s odchodom autobusu je trvanie cesty vlakom  do stanice, odkiaľ odchádza ten autobus.

\begin{tabular}{@{}p{0.2\linewidth}p{0.7\linewidth}}
\textbf{\small Obsah súboru:} &
\vspace{-3em}
\begin{code}
vlak,9:15,10:45,12:15,14:30,16:15,18:20
bus,1:00,11:00,13:00,15:00,17:00
bus,1:45,9:30,12:08,16:33
\end{code}
\end{tabular}

\vspace{-2em}
\begin{tabular}{@{}p{0.2\linewidth}p{0.7\linewidth}}
\textbf{\small Vstup:} &
\vspace{-3em}
\begin{code}
Čas: @\fbox{10:00}@
Trvanie cesty vlakom: @\fbox{1:00}@
\end{code}
\end{tabular}

\vspace{-2em}
\begin{tabular}{@{}p{0.2\linewidth}p{0.7\linewidth}}
\textbf{\small Výstup:} &
\vspace{-3em}
\begin{code}
Najbližší spoje (vlak, autobus):
@\fbox{12:15 - 13:15, 15:00 -}@
\end{code}
\end{tabular}
\vspace{-2em}


\subsection{Funkcie}
\textbf{Funkcia} je pomenovaná časť programu, ktorá vykonáva špecifickú činnosť. Hovorí sa im preto tiež \textit{procedúry} alebo \textit{podprogramy}. Predstavuje súvislú časť kód, obsahujúcu sled na seba nadväzujúcich príkazov, tvoriacich jeden logický celok.  Takto umožňuje zložitejší program rozdeliť na viacero samostatných častí.


\subsubsection*{1. Vraky}
V šírich vodách Atlantiku sa stále ukrýka nepreberné bohatstvo vo vrakoch potopených lodí. V tejto minhre bude tvojou úlohou odkryť tajomstvo skrývajúce sa pod hladinou, nájdením parníku vytvoreného na náhodnej pozícii. Do programu napíš funkciu \verb|vzdialenost(x, y)|, ktorá na základe zadaných súradníc vypočíta ako ďaleko si od vraku.

\begin{code}
Sonar hlási potopený parník na dohľad!
Tvoje súradnice?: ___,___
Od vraku si _____ námorných míľ.
...
Našiel si vrak. Dobrá práca!
\end{code}


\subsubsection*{2. Cézarová šifra}
Pri tvojich cestách po lodných pokladoch ťa odpočúvajú piráti, ktorí ťa chcú predbehnúť a obohatiť sa. Na utajenie svojej polohy a správ s pevninou musíš svoje informácie šifrovať. 

Funkcia \verb|sifruj(sprava, kluc)| zašifruje text správy tak, že posunie každé písmeno abecedy podľa písmena \verb|kluc|, čiže napríklad správa \emph{"ABC"} sa kľúčom \emph{"B"} zmení na \emph{"BCD"}. 

Funkcia \verb|desifruj(sifra, kluc)| bude fungovať spätne.  Pre lepšiu bezpečnosť podporuj aj dlhšie kľúče. 

Každé písmeno bude vyjadrovať posun od začiatku abecedy písmena, s ktorým sa stretne. Potom správa \emph{"AVE CEZAR"} s kľúčom \emph{"BCD"} bude \emph{"BXH DGCBT"}.


\subsubsection*{3. Pascalov trojuholník}
Vytvor funkciu \verb|pascalov_trojuholnik(n)|, ktorá vypíšte súčtovú pyramídu s $n$ riadkami, ktorá má po okrajoch jednotky a nasledujúce riadky sa tvoria ako súčet dvoch čísel v predchádzajúcom riadku.

\begin{code}
Počet riadkov: 5

    1
   1 1
  1 2 1
 1 3 3 1
1 4 6 4 1
\end{code}

\subsubsection*{4. Štatistika}
Pre investora je dôležité poznať podmienky trhu a potenciálnu konkurenciu predtým, než si naplánuje stratégiu investovania. Rozbiehaš realitnú kanceláriu a skôr než nastaviš ceny pre konkrétne byty, zisti v akom vzťahu je výmera bytu k jeho cene v lokalite. Pre každú štatistickú funkciu si napíš zodpovedajúcu procedúru. Údaje o bytoch načítaj zo súboru.

\begin{code}
Súbor s bytmi v lokalite: ______

                    :   Cena (E)  :   Výmera(m^2) :
Priemer             :             :               :
Medián              :             :               :
Modus               :             :               :
Smerodajná odchýlka :             :               :
\end{code}


\subsubsection*{5. Lietadlo}
Pilotov v kokpite lietadlo by počas letu zaujímalo, ako ďaleko sú ešte od prístatia. Zo zemepisných súradníc aktuálnej polohy a súradníc cieľa vypočíataj vo funkcii `letime(x, y)` najkratšiu vzdialenosť medzi týmito bodmi na sférickom povrchu zemegule (\textit{Ortodróma}).
$$ \theta = \arccos(\sin(90° - \alpha_1) \cdot \sin(90° - \alpha_2) + \cos(90° - \alpha_1) \cdot \cos(90° - \alpha_2) \cdot \cos(|\beta_1 - \beta_2|)) $$
$$ d = (r \cdot \theta \cdot \pi)\;/\;180 $$

\begin{code}
Pozícia: 42.990967 -71.463767
Cieľ: 48.53682 -13.855231

Vzdialenosť: 4416.21 km
\end{code}


\subsubsection*{5. Bublikové triedenie}
Pre prehľadnosť údajov je užitočné vedieť ich utriediť podľa rôznych kritérií. Napíš program, ktorý vypíše študentov zo súboru zoradených podľa zadaného názvu stĺpčeka vzostupne.  Na začiatok použi algoritmus bublinkového triedenia, neskôr proces zefektívni využitím algoritmom triedenia zlučovaním alebo rýchlym triedením.

\paragraph{Obsah súboru (ziaci.csv):}

\begin{code}
meno, priezvisko, vek, datum narodenia, bydlisko, priemer, trieda
Milan, Peterka, 15, 2004-09-18, Bratislava, 1.6, I.B.
...
\end{code}


\subsubsection*{7. Rímske čísla}
Od archeológov si dostal dlhý zoznam rímskych čísel, ktoré boli nájdené v novobjavených podzemených historických pamiatkach. Tažko sa v nich dá vyznať a je na tebe, aby si ich premenil na "normálne" arabské čísla. Pre zhrnutie ti poslali aj zoznam pravidiel prevodu týchto číselných systémov. Napíš funkciu \verb|rimske_na_arabske(rimske)|, ktorá premení rímske na arabské číslo.

\begin{code}
I = 1
V = 5
X = 10
L = 50
C = 100
D = 500
M = 1000
\end{code}


\subsubsection*{8. Základný tvar zlomku}
Zlomky sú vhodné na presné výpočty s častami z celku. Vytvor jednoduchú kalkulačku, ktorá umožňuje dva zlomky sčítať, odčítať, násobiť a deliť. Výsledok vždy zjednoduš na základný tvar (\emph{Euklidov algoritmus pre NSD a NSN}).

\begin{code}
Kalkulačka zlomkov
a = 3/4
b = 1/2
Vypočítaj (+, -, *, /): +

Výsledok:
3/4 + 1/2 = 5/4
\end{code}


\subsubsection*{9. Hra Poklad}
Povráva sa, že na strašidelnom hrade v Karpatoch je bludisko so siedmimi tajomnými komnatami. Každá má meno a je v nej truhlica s pokladom. Mapa bludiska je náhodne poskladaná, uložená v pamäti počítača, ale nie je nakreslená na obrazovke. Hráč musí zistiť, ako sú komnaty navzájom pospájané. Na začiatku hry sa ocitne v náhodne vybranej komnate. Jeho úlohou je zhromaždiť všetky truhlice v jednej komnate, pričom môže vykonať iba ohraničený počet krokov.

\paragraph{Komnaty v mriežke s uloženým pokladom:}
\begin{enumerate}
\itemsep0pt
\item Purpurová a pekelná - Drahokamy
\item Červená a čudná - Žuvačky
\item Sivá a studená - Nanuky
\item Žltá a žeravá - Zlatky
\item Čierna a čarodejná - Smeti
\item Hnedá a hrozivá - Kalkulačky
\item Zelená a záhadná - Medeňáky
\end{enumerate}
   
\paragraph{Vzorová časť hrania hry:}

\begin{code}
Počítač rozumie týmto príkazom
S, V, J, Z   : Pohyb na sever, východ, juh, západ
ZDVIHNI		 : Zdvihne truhlicu
POLOZ		 : Položí truhlicu
KDE			 : Informuje o polohe truhlíc
SOS			 : Vypíše pravidlá hry

Si v 4.komnate
Je žltá a žeravá
Sú v nej: ZLATKY
Čo chceš robiť?
? ZDVIHNI
Zdvihol si truhlicu, v ktorej sú zlatky.

Ešte stále si 4.komnate
Čo chceš robiť?
? Z
...
\end{code}

\subsubsection*{10. Databáza} - Na školu za siedmimi horami a dolinami si objednali počítač na uloženie a prehliadanie záznamov o študentoch. Keďže rok, čo rok odchádzajú maturanti a prichádzajú prváci, potrebujú tabuľky i upravovať. Napíš databázový systém, ktorý bude umožňovať vytvárať a mazať tabuľky, kde každá bude vo vlastnom csv súbore. Budú sa dať vkladať a mazať aj riadky, či upravovať jednotlivé políčka. Ulož do databázy napríklad aj informácie o knihách zo školskej knižnice.

Pre nápady na rozšírenie pozri: \textbf{Postavme si databázu[EN]}: \url{https://cstack.github.io/db_tutorial/}

\paragraph{Ukážka možností systému:}

\begin{code}
DATABÁZA> NOVÁ TABUĽKA žiaci: meno, priezvisko, dátum narodenia
DATABÁZA> TABUĽKY
žiaci
DATABÁZA> OTVOR TABUĽKU žiaci
ŽIACI> VLOŽ Ružena, Kvetinková, 1998-11-15
ŽIACI> ZOBRAZ
   +----+---------+-------------+-----------------+
   | id |  meno   |  priezvisko | dátum narodenia |
   +----+---------+-------------+-----------------+
   | 1  |  Ružena | Kvetinková  |  1998-11-15     |
   +----+---------+-------------+-----------------+
ŽIACI> UPRAV 1 NASTAV priezvisko: Sedmokrásková
ŽIACI> ZOBRAZ: ZORAĎ PODĽA priezvisko
...
ŽIACI> ZOBRAZ: HĽADAJ PODĽA priezvisko: Sedmokrásková
...
ŽIACI> ZMAŽ 1
ŽIACI> ZMAŽ TABUĽKU žiaci
DATABÁZA> SKONČI
\end{code}


\subsubsection*{11. Kalkulačka}
Moderné vedecké kalkulačky sú takmer zázrakom. Buď tým, že sa mimo akademickej pôdy skoro vôbec nepoužívajú, alebo samotnou zložitosťou ich fungovania. Dokážu rozlíšiť, či má prednosť násobenie alebo sčítanie, zatiaľ čo vezmú do úvahy zátvorky. Nemôže byť pre nich nič jednoduchšie ako prijsť na to, čo je číslo a čo operátor v dlhom posuvnom texte displeja. Vytvor program kalkuačky, ktorá sa bude správať ako vrecková vedecká kalkulačka (s infixovým zápisom)(\textit{Algoritmus posunovacej stanice (Shunting yard algorithm)}).

\begin{code}
> 5 * (1589 - 2 * 74) / 2 + (33 * 8)
> 3866.5
> ...
\end{code} 


\section{Vzorové riešenia}
Riešenia úloh v učebniciach spravidla zachytávajú jedinú správnu odpoveď v podobe čísla, oznamovacej vety alebo nákresu bez vysvetlenia postupu. Možností korektného usporiadania príkazov v programe však existuje v princípe neobmedzené množstvo, ktoré len vzrastá so znalosťami pokročilejších vyjadrovacích prostriedkov programovacieho jazyka. Postupnosť elementárnych krokov na dospenie k výsledku je nevyhnuté aj v matematike, ale rozdiel s programovaním je vo formuláciach otázok. 

Podľa nášho názoru by ideálne riešenie úlohy podporujúce samoštúdium nemalo byť odhalené okamžite v celku, ale navádzať na postup cez čiastkové nápovede vo forme otázok. Riešiteľovi je za takých okolností jasnejšie, prečo si autor zvolili práve odkrytú programovú konštrukciu. Alternatívne môže byť predostretý výpis programu zohľadňujúci požiadavky na vstup a výstup a oboznámenosť žiaka s témou v danom momente na základe predošlých úloh. V prípade ukázania už hotového riešenia slúži to iba na kontrolu. Pokladanie regulujúcich otázok spočíva na učiteľovi. 

V \textbf{prílohe \ref{chapter:appendix-riesenia}} sa nachádzajú \textbf{exemplárne riešenia} takmer všetkých úloh zo zbierky okrem zameraných na opakovanie predošlého celku a na divergentné myslenie. U týchto cvičení je riešenie nanajvýš naznačené výpustkami (\dots), ktoré sú vyznačené v úlohe ,,6.1. Prepisovanie''. Spôsobom zachytenia riešení sa prikláňame v návrhu nášho učebného textu k použitiu na hodinách pod dohľadom učiteľa. 

\begin{tabular}{@{}p{0.3\linewidth}p{0.6\linewidth}}
\textbf{\small Náznak riešenia:} &
\vspace{-3em}
\begin{code}
nazov_suboru = input("Názov súboru")
subor = open(nazov_suboru, "r")
for riadok in subor:
    riadok = riadok.strip()
    ...
subor.close()
\end{code}
\end{tabular}

Na vzorovom programe úlohy ,,3.3. Pyramída'' ukážeme prispôsobenie riešenia odpozorovaním chybám žiakov pri vypracovaní a miesta, kde o zvolenom zápise rozhodujú predchádzajúce skúsenosti. Hneď prvý riadok načítavajúci vstup sa dá napísať troma variantami rastúcej zložitosti na porozumenie. 

\begin{tabular}{@{}p{0.3\linewidth}p{0.6\linewidth}}
\textbf{\small Varianta č.1:} &
\vspace{-3em}
\begin{code}
print("Výška pyramídy: ")
vyska = input()
vyska = int(vyska)
\end{code}
\end{tabular}

\vspace{-2em}
\begin{tabular}{@{}p{0.3\linewidth}p{0.6\linewidth}}
\textbf{\small Varianta č.2:} &
\vspace{-3em}
\begin{code}
vyska = input("Výška pyramídy: ")
vyska = int(vyska)
\end{code}
\end{tabular}

\vspace{-2em}
\begin{tabular}{@{}p{0.3\linewidth}p{0.6\linewidth}}
\textbf{\small Varianta č.3:} &
\vspace{-3em}
\begin{code}
vyska = int(input("Výška pyramídy: "))
\end{code}
\end{tabular}

Varianta č.1 prísne odlišuje medzi výstupom cez príkaz \verb|print|, načítaní vstupu z klávesnice cez \verb|input| a premenou písmen na typ číslo \verb|int|. Vo variante č.2 už operujeme s viacúčelovosťou príkazu \verb|input|, ktorá sa v skorších fázach pletie s priradením konštantného reťazca do premennej. Najsofistikovanejší spôsob spolieha zápis zloženej funkcie a predstavuje idióm. Žiadna syntax vzorového programu nie je univerzálna, ale vyvíjaja postupom kapitolami. V témach premenné a podmienky sa najčastejšie ukazuje varianta č.2. Od téme cykly sa varianta č.3 považuje za dostatočne známu.

\begin{tabular}{@{}p{0.3\linewidth}p{0.6\linewidth}}
\textbf{\small Násobenie reťazcov:} &
\vspace{-3em}
\begin{code}
for riadok in range(vyska):
	...
	print(" " * medzery + "*" * hviezdy)
\end{code}
\end{tabular}

\vspace{-2em}
\begin{tabular}{@{}p{0.3\linewidth}p{0.6\linewidth}}
\textbf{\small Vnorené cykly:} &
\vspace{-3em}
\begin{code}
for riadok in range(vyska):
	...
	for i in range(medzery):
		print(" ", end="")
	print()
	for i in range(hviezdy):
		print("*", end="")
	print()
\end{code}
\end{tabular}

Náročnosť úlohy a tým umiestnenie v systéme je ovplyvnené aj typmi príkazov vo vzorovom riešení. Tretia úloha v kapitole cykly má za cieľ precvičiť for cyklus s pevne určeným počtom opakovaní, preto sa medzery a hviezdičky vypisujú násobením reťazcov. Napriek tomu, že reťazce sa dosiaľ neprebrali do hĺbky je jednoduchšie na pochopenie idea, že: ,,6 krát hviezdička vypíše 6 hviezedičiek'' než mechanizmus za vnorenými for cyklami. Na tomto príklade sa ukazuje závislosť zaradenia úlohy od použitého jazyka, pretože nie všetky umožňujú násobenie reťazcov, keďže sa jedná o zložený dátový typ.

\section{Diskusia}


Štruktúra úlohy (zadanie - scenéria, uvedenie do problémovej situácie, čo sa vyžaduje (priamo / nepriamo), rozvíja čítanie s porozumením

Porozumenie textu: primerne 50 slov / úlohu, 4 - 5 viet, fog index 18

\begin{figure}[h]
\centering
\begin{subfigure}[b]{0.32\textwidth}
\centering
\includegraphics[width=\textwidth]{assets/words.png}
\end{subfigure}
\hfill
\begin{subfigure}[b]{0.32\textwidth}
\centering
\includegraphics[width=\textwidth]{assets/sentences.png}
\end{subfigure}
\hfill
\begin{subfigure}[b]{0.32\textwidth}
\centering
\includegraphics[width=\textwidth]{assets/fog.png}
\end{subfigure}
\hfill
\caption{Kvantifikatívne charakteristiky textu úloh v zbierke}
\end{figure}

textom znižujeme extra kognitívnu záťaž - je rozdiel porozumieť textu a schopnosť vymyslieť riešenie

Zoradenie úloh je podľa náročnosti, do tém a podtém podľa toho ktorý najzložitejší pojem alebo postup sa v nich vyskytuje

Tabuľky na didaktické funkcie, kognitívnu úroveň a kvantitatívne vlastnosti textu (fog, počet slov, počet viet)
zaradie úloh do kategórii závisí od ich poradia v zbierke

Pozorovanie ako to žiaci zvládali

Do záveru
Zhrnutie
Možné rozšírenia: otázky a odpovede (programové vyučovanie) pri prezentovaní riešení, implementácia do webového prostredia (segmentácia úloh na stránky), zhodnotie úspešnosti rozšíriteľnosti zbierky



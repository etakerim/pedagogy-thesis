\chapter{Úvod}
Rýchly vývoj informatiky ako vedy za posledné desaťročia, rapídny technologický rozvoj vedúci k zvýšeniu dostupnosti prostriedkov výpočtovej techniky a nástup inovatívnych prístupov vo výučbe umocňuje nedostatok moderných kvalitných vzdelávacích materiálov pre stredné školy. Na vzdelávanie informatiky na úrovni vyššieho sekundárneho vzdelania nevplýva ani tak posun v podstatných princípoch odboru, ale skôr výskyt nových informačných a komunikačných technológií prinášajúcich predtým nepoznané výzvy do spoločnosti. Nemenej podstatný dopad majú schopnosti nástrojov, čiže vlastností softvérového vybavenia a jeho používateľského rozhrania

Pre niektoré oblasti, ako sú kybernetická bezpečnosť alebo programovanie vychádzajú v súčasnosti už aktuálne učebnice, ale častokrát sú školám nedostupné najmä z ekonomických dôvodov. Poznatky v tlačených učebniciach zároveň zvyknú rýchlo zastarávať, preto sa preferujú elektronické knihy. E-knihy majú zasa z pohľadu vyučovacieho procesu úskalia ohľadom ich prístupnosti a prevládajúceho tradicionalizmu.

Význačné zmeny sa udiali v programovacích jazykoch ako nástrojoch na formálny zápis algoritmov. Ovplyvnili výber prostredí na programovanie a jazykov pre použitie vo vzdelávaní a postupnosť v osvojovaní prvkov jazyka. Nové verzie neustále prinášajú úpravy syntaxe a údajových štruktúr.

V záujme udržania kroku s najnovšími relevantnými poznatkami v odbore a stavom technológií, sú učitelia často nútení pripravovať si vlastné učebné texty. Napomáhajú im k tomu internetové zdroje a e-learningové knižnice. Ponechávajú však časovo náročný výber adekvátnych článkov a vyváženú skladbu cvičení na kreativite učiteľa. Na prvý pohľad sa to môže javiť prospešné pre individualizáciu výučby. Realizácia nebýva nikdy ideálna, tak aby žiakom poskytla rozmanité úlohy na samostatnú domácu prípravu.

Na vyučovacích hodinách by mala dobrá učebnica vhodne podnecovať a podporovať problémové vyučovanie, ktoré aktivizuje žiakov k hlbšiemu ovládaniu preberanej témy. Okrem iných časových a priestorových obmedzení vplývajúcich na výber učebnej metódy, vedie učiteľa nedostupnosť usporiadaného učebného textu na sprevádzanie učivom, hlavne k využitiu frontálneho výkladu.

V svojpomocnej tvorbe učebníc všeobecnovzdelávacieho učebného predmetu informatika sa zameriavame na vzdelávací štandard algoritmické riešenie problémov podľa štátneho vzdelávacieho programu. V snahe preniesť úsilie v triede pri osvojovaní učiva vo väčšej miere na žiaka, vychádzame z potreby zostavenia zbierky úloh z programovania pre stredné školy k použitiu v škole aj na doma. Nevyhnutné zložky hodné komplexného posúdenia sú obsahová a formálna rovina.

Na obsah zbierky kladieme nároky na vyvážené pokrytie dôležitých typov úloh na viacerých kognitívnych úrovniach v súlade s princípmi formulácie úlohy aj systematickým usporiadaním úloh. Spôsob začlenenia riešení do zbierky a ich fáza konfrontovania s postupmi žiakov je tiež neoddeliteľnou súčasťou učebného textu ako celku. Na predchádzanie straty aktuálnosti textov budeme požadovať čo najväčšiu nezávislosť úloh na programovacom prostredí a jazyku.

Po formálnej stránke máme záujem navrhnúť ,,živú učebnicu'', kde by mohli učitelia v online prostredí postupne dopĺňať nové úlohy do jednotlivých častí zbierky. Na tento účel prispôsobujeme existujúce metódy hodnotenia náročnosti textu a zaraďovania úloh do systému úloh podľa príslušných kritérií. Prihliada sa pritom na obsahové zameranie v rámci tematických okruhov podľa znenia zadania. Namiesto opakujúcich sa typov cvičení sa nájde náhrada vhodným preformulovaním.

V hlavnej časti práce preskúmame teóriu tvorby učebníc s ohľadom na funkcie, skvalitňovanie učebného textu a na nástrahy pri presune do elektronickej podoby. Ďalej sa venujeme zostrojeniu adekvátneho a rozšíriteľného systému úloh podľa zbierok príkladov z matematiky. Nasleduje prehľad a rozbor súčasných učebníc programovania a zbierok úloh. V praktickej časti predstavíme konkrétne úlohy so vzorovými riešeniami a hodnotením zaradenia či náročnosti úloh. Nakoniec prediskutujeme typické vlastnosti textu úloh po obsahovej a grafickej stránke.
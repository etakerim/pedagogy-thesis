\subsection{Funkcie}

\subsubsection*{1. Vraky}
\begin{solution}
import random
import math
MIERKA = 15
VRAK_X = random.randint(0, MIERKA)
VRAK_Y = random.randint(0, MIERKA)

def vzdialenost(x, y):
    return math.hypot(x - VRAK_X, y - VRAK_Y)
  
def nasiel(x, y):
    return x == VRAK_X and y == VRAK_Y

print("Sonar hlási potopený parník na dohľad!")
while True:
    suradnice = input("Tvoje súradnice?: ")
    suradnice = suradnice.split(",")
    x = int(suradnice[0])
    y = int(suradnice[1])
    if nasiel(x, y):
        print("Našiel si vrak. Dobrá práca!")
        break

    print(f"Od vraku si {vzdialenost(x, y):.3f} námornych míľ")
\end{solution}


\subsubsection*{2. Lietadlo}
\begin{solution}
from math import sin, cos, acos, radians
def letime(x, y):
    POLOMER_ZEME = 6371.11
    uhol = acos(sin(x[0]) * sin(y[0]) +
                cos(x[0]) * cos(y[0]) * cos(abs(x[1] - y[1])))
    obluk = POLOMER_ZEME * uhol
    return obluk

start = input("Pozícia: ")
ciel = input("Cieľ: ")

start = start.split(" ")
start = [
    radians(float(start[0])),
    radians(float(start[1]))
]
ciel = ciel.split(" ")
ciel = [
    radians(float(ciel[0])),
    radians(float(ciel[1]))
]
vzdielenost = letime(start, ciel)
print(f"\nVzdialenosť: {vzdielenost:.2f} km")
\end{solution}


\subsubsection*{3. Cézarová šifra}

\begin{solution}
def sifruj(sprava, kluc):
    sifra = ""
    A = ord("A")
    Z = ord("Z")
    ABECEDA = Z - A + 1
    sprava = sprava.upper()

    for i in range(len(sprava)):
        pismeno = sprava[i]
        k = kluc[i % len(kluc)]

        if A <= ord(pismeno) <= Z:
            poradie = ord(pismeno) - A
            posun = ord(k) - A

            poradie = (poradie + posun) % ABECEDA
            sifra += chr(poradie + A)

    return sifra

def desifruj(sifra, kluc):
    sprava = ""
    A = ord("A")
    Z = ord("Z")
    ABECEDA = Z - A + 1
    sifra = sifra.upper()

    for i in range(len(sifra)):
        pismeno = sifra[i]
        k = kluc[i % len(kluc)]

        if A <= ord(pismeno) <= Z:
            poradie = ord(pismeno) - A
            posun = ord(k) - A

            poradie = (poradie - posun) % ABECEDA
            sprava += chr(poradie + A)

    return sprava

retazec = input("Zadaj správu: ")
kluc = input("Vlož tajný kľúč: ")
akcia = input("Čo spraviť (šifruj / dešifruj): ")
s = ""
if akcia == "šifruj":
    print("Zašifrovaná správa: ", end="")
    s = sifruj(retazec, kluc)
elif akcia == "dešifruj":
    print("Dešifrovaná správa: ", end="")
    s = desifruj(retazec, kluc)
print(s)
\end{solution}

\subsubsection*{4. Pascalov trojuholník}
\begin{solution}
def pascalov_trojuholnik(n):
    row = [1, 1]
    medzery = n
    pocet = 0

    for i in range(n):
        pocet += 1
        medzery -= 1

        print(" " * medzery, end="")
        for cislo in row[:pocet]:
            print(cislo, end=" ")
        print()

        for j in range(pocet - 1,  0, -1):
            row[j] = row[j] + row[j - 1]
        row.append(1)

vyska = int(input("Zadajte výšku Pascalovho trojuholníka: "))
pascalov_trojuholnik(vyska)
\end{solution}

\subsubsection*{5. Štatistika}
\begin{solution}
import math
def priemer(zoznam):
    sucet = 0
    for prvok in zoznam:
        sucet += prvok
    return sucet / len(zoznam)

def modus(zoznam):
    nazvy = []
    vyskyty = []
    # Zisti koľkokrát sa čo vyskytuje
    for prvok in zoznam:
        index = -1
        for i in range(len(nazvy)):
            if prvok == nazvy[i]:
                index = i

        if index != -1:
            vyskyty[index] += 1
        else:
            nazvy.append(prvok)
            vyskyty.append(0)

    # Pozri sa po najväčšom počte objavení sa a prehlás ho za modus
    najviac = None
    rekorder = -1

    for i in range(len(vyskyty)):
        if najviac == None or vyskyty[i] > najviac:
            najviac = vyskyty[i]
            rekorder = i

    return nazvy[rekorder]

def utried(zoznam):
    for i in range(len(zoznam) - 1):
        for j in range(len(zoznam) - i - 1):
            if zoznam[j] > zoznam[j + 1]:
                x = zoznam[j]
                zoznam[j] = zoznam[j + 1]
                zoznam[j + 1] = x

def median(zoznam):
    utried(zoznam)
    stred = (len(zoznam) + 1) // 2
    return zoznam[stred - 1]

def smerodajna_odchylka(zoznam):
    average = priemer(zoznam)

    sucet = 0
    for prvok in zoznam:
        sucet += (prvok - average) ** 2

    return math.sqrt(sucet / len(zoznam))

subor = input("Súbor s bytmi v lokalite: ")
ceny = []
vymery = []

byty = open(subor, "r")

for byt in byty:
    zaznam = byt.split(",")
    ceny.append(int(zaznam[0]))
    vymery.append(int(zaznam[1]))

byty.close()

# Pozri tiež modul "statistics" - https://docs.python.org/3/library/statistics.html
print(f"{'':25s}:{'Cena (eur)':15s}:{'Výmera(m^2)':15s}:")
print(f"{'Priemer':25s}:{priemer(ceny):15.2f}:{priemer(vymery):15.2f}:")
print(f"{'Medián':25s}:{median(ceny):15.2f}:{median(vymery):15.2f}:")
print(f"{'Modus':25s}:{modus(ceny):15.2f}:{modus(vymery):15.2f}:")
print(f"{'Smerodajná odchýlka':25s}:{smerodajna_odchylka(ceny):15.2f}:{smerodajna_odchylka(vymery):15.2f}:")
\end{solution}

\subsubsection*{6. Rímske čísla}
\begin{solution}
def rimske_na_arabske(rimske):
    TABULKA = {"I": 1, "V": 5, "X": 10, "L": 50, "C": 100, "D": 500, "M": 1000}
    arabske = []
    vysledok = 0
    for symbol in rimske:
        arabske.append(TABULKA[symbol])

    i = 0
    while i < len(arabske):
        if i + 1 != len(arabske) and arabske[i] < arabske[i + 1]:
            vysledok += arabske[i + 1] - arabske[i]
            i += 2
        else:
            vysledok += arabske[i]
            i += 1
    return vysledok

cislo = input("Zadaj rímske číslo: ")
print(rimske_na_arabske(cislo))
\end{solution}

\subsubsection*{7. Základný tvar zlomku}
\begin{solution}
def nsd(a, b):
    # Najväčší spoločný deliteľ
    # alebo: math.gcd(a, b)
    while b > 0:
        a, b = b, a \% b
    return a

def nsn(a, b):
    # Najmenší spoločný násobok
    return a * b // nsd(a, b)

def zakladny_tvar(zlomok):
    delitel = nsd(zlomok[0], zlomok[1])
    return [
        zlomok[0] // delitel,
        zlomok[1] // delitel
    ]

def vytvor_zlomok(retazec):
    # alebo: map(int, retazec.split("/"))
    zlomok = retazec.split("/")
    for i in range(len(zlomok)):
        zlomok[i] = int(zlomok[i])
    return zlomok

def nasobit(x, y):
    citatel = x[0] * y[0]
    menovatel = x[1] * y[1]
    zlomok = [citatel, menovatel]
    return zakladny_tvar(zlomok)

def delit(x, y):
    citatel = x[0] * y[1]
    menovatel = x[1] * y[0]
    zlomok = [citatel, menovatel]
    return zakladny_tvar(zlomok)

def scitat(x, y):
    menovatel = nsn(x[1], y[1])
    x_citatel = x[0] * (menovatel // x[1])
    y_citatel = y[0] * (menovatel // y[1])
    citatel = x_citatel + y_citatel
    zlomok = [citatel, menovatel]
    return zakladny_tvar(zlomok)

def odcitat(x, y):
    menovatel = nsn(x[1], y[1])
    x_citatel = x[0] * (menovatel // x[1])
    y_citatel = y[0] * (menovatel // y[1])
    citatel = x_citatel - y_citatel
    zlomok = [citatel, menovatel]
    return zakladny_tvar(zlomok)

def vypis(zlomok):
    return f"{zlomok[0]}/{zlomok[1]}"


print("Kalkulačka zlomkov")
a = input("a = ")
b = input("b = ")
akcia = input("Vypočítaj (+, -, *, /): ")

a = zakladny_tvar(vytvor_zlomok(a))
b = zakladny_tvar(vytvor_zlomok(b))

print("\nVýsledok:")
if akcia == '+':
    print(f"{vypis(a)} + {vypis(b)} = {vypis(scitat(a, b))}")
elif akcia == '-':
    print(f"{vypis(a)} - {vypis(b)} = {vypis(odcitat(a, b))}")
elif akcia == '*':
    print(f"{vypis(a)} * {vypis(b)} = {vypis(nasobit(a, b))}")
elif akcia == '/':
    print(f"{vypis(a)} / {vypis(b)} = {vypis(delit(a, b))}")
\end{solution}

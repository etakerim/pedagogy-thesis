\subsection{Podmienky}

\subsubsection*{1. Heslo}
\begin{solution}
print("Stoj! Povedz Heslo!")
pokus = input("? ")
if pokus == "tajne heslo":
    print("Vstúp, priateľ")
else:
    print("Zmizni kade ľahšie")
\end{solution}

\subsubsection*{2. Najväčšie číslo}
\begin{solution}
x = input("1.skóre: ")
y = input("2.skóre: ")
z = input("3.skóre: ")
x = int(x)
y = int(y)
z = int(z)
najviac = x
poradie = 1
if y > najviac:
    najviac = y
    poradie = 2
if z > najviac:
    najviac = z
    poradie = 3

print(f"Najväčie skóre {najviac} bodov má {poradie} hráč.")
\end{solution}

\subsubsection*{3. Vhodné oblečenie}

\begin{solution}
pocasie = input("Ako je vonku?: "))
miesto = input("Kam ideš?: "))

if pocasie == "slnečno"
	povinne = "šiltovka"
if pocasie == "zamračené":
	povinne = "mikina"
if počasie == "dážď":
	povinne = "vetrovka"
	
if miesto == "ihrisko":
	odporucanie = "tepláky"
if miesto == "škola"
	odporucanie = "košela"

print("Určite si nezabudni", povinne, "a tiež si vezmi", odporucanie, ".")
\end{solution}

\subsubsection*{4. Morský vánok}
\begin{solution}
stupen = input("Sila vetra na Beaufortovej stupnici: ")
stupen = int(stupen)

if stupen == 0:
	nazov = "bezvetrie"
	rychlost = 0
	vlny = 0
elif stupen == 1:
	nazov = "vánok"
	rychlost = 2
	vlny = 0.1
elif stupen == 2:
	nazov = "slabý vietor"
	rychlost = 5
	vlny = 0.2
# Doplň ostatné stupne podľa Beafortovej stupnice

print(f"Vietor sa nazýva {nazov}.")
print(f"Vietor má rýchlosť {rychlost} kt.")
print(f"Očakávaná výška vĺn je {vlny} m.")
\end{solution}

\subsubsection*{5. Pokazený rozpis}
\begin{solution}
min = input("Trvanie (min.): ")
min = int(min)
hod = min // 60
dni = hod // 24
hod -= dni * 24
min -= (hod * 60) + (dni * 24 * 60)

print("=", end=" ")
if dni > 0:
    print(f"{dni} d.", end=" ")
if hod > 0:
    print(f"{hod} hod.", end=" ")

print(f"{min} min.")
\end{solution}

\subsubsection*{6. Hovoriaca kalkulačka}
\begin{solution}
print("Som hovorica kalkulačka a rada počítam!")
a = int(input("Povedz mi prvé číslo: "))
b = int(input("Potrebujem ďašie číslo: "))
cinnost = input("Chceš ich sčítať alebo odčítať: ")

if cinnost == "sčítať":
    print(f"Výsledok tvojho príkladu: {a} plus {b} je {a + b}")
elif cinnost == "odčítať":
    print(f"Výsledok tvojho príkladu: {a} mínus {b} je {a - b}")
else:
    print(f"Neviem čo znamená '{cinnost}'")
\end{solution}

\subsubsection*{7. Chaos v lístkoch}
\begin{solution}
print("Popíš mi svoju cestu s MHD")
zony = int(input("Koľko zón prejdeš?:"))
minuty = int(input("Koľko minút má trvať cesta?:"))

if zony == 2 and minuty <= 30:
	cena = 0.55
elif zony == 3 and minuty <= 60:
	cena = 0.80
elif zony == 4 and minuty <= 60:
	cena = 1.00
elif zony == 5 and minuty <= 90:
	cena = 1.25
elif zony == 6 and minuty <= 90:
	cena = 1.50
elif zony == 7 and minuty <= 120:
	cena = 1.65

print("Zlavnený lístok stojí {cena:.2f} eur.")
\end{solution} 

\subsubsection*{8. Kvadratická rovnica}
\begin{solution}
import math
print("Koeficienty kvadratickej rovnice:")
a = float(input("a = "))
b = float(input("b = "))
c = float(input("c = "))

if a == 0:
	print("Ide o lineárnu rovnicu")
else:
	print(f"{a:g}x^2 + {b:g}x + {c:g} = 0")
	D = b ** 2 - 4 * a * c
	if D < 0:
		print("Kvadratická rovnica nemá riešenie v R")
	elif D > 0:
		x1 = (-b - math.sqrt(D)) / (2 * a)
		x2 = (-b + math.sqrt(D)) / (2 * a)
		print(f"x1 = {x1}")
		print(f"x2 = {x2}")
	elif D == 0:
		x = -b / (2 * a)
		print(f"x = {x}")
		Vx = -b / (2 * a)
		Vy = c - ((b ** 2) / (4 * a))
		print(f"V[{Vx}; {Vy}]")
\end{solution}

\subsubsection*{9. Trojuholníky}
\begin{solution}
import math

print("Zadajte strany ľubovolného trojuholníka:")
a = input("a = ")
b = input("b = ")
c = input("c = ")

a = float(a)
b = float(b)
c = float(c)

if a + b <= c:
	print("Pre trojuholník neplatí trojuholníková nerovnosť")
	print("a + b <= c")
	print(f"{a} + {b} <= {c}")
elif a + c <= b:
	print("Pre trojuholník neplatí trojuholníková nerovnosť")
	print("a + c <= b")
	print(f"{a} + {c} <= {b}")
elif b + c <= a:
	print("Pre trojuholník neplatí trojuholníková nerovnosť")
	print("b + c <= a")
	print(f"{b} + {c} <= {a}")
else:
	alpha = math.acos((a**2 - b**2 - c**2) / (-2*b*c))
	beta = math.acos((b**2 - a**2 - c**2) / (-2*a*c))
	gamma = math.acos((c**2 - a**2 - b**2) / (-2*a*b))

	va = c * math.sin(beta)
	vb = a * math.sin(gamma)
	vc = b * math.sin(alpha)

	alpha = math.degrees(alpha)
	beta = math.degrees(beta)
	gamma = math.degrees(gamma)

    print(f"\nStrany: a = {a}; b = {b}; c = {c}")
    print(f"Uhly: alpha = {alpha}°; beta = {beta}°; gamma = {gamma}°")
    print(f"Výšky: v(a) = {va}; v(b) = {vb}; v(c) = {vc}")
    print(f"O = {a + b + c}")
    print(f"S = {a * va * 0.5}")

        print("Trojuholník je:", end=" ")
        if a == b == c:
            print("Rovnostranný", end=", ")
        elif a == b or b == c or c == a:
            print("Rovnoramenný", end=", ")
        else:
            print("Rôznostranný", end=", ")

	if alpha < 90 and beta < 90 and gamma > 90:
		print("Ostrouhlý")
	elif alpha > 90 or beta > 90 or gamma > 90:
		print("Tupouhlý")
	else:
        print("Pravouhlý")
\end{solution}

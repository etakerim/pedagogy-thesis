\subsection{Premenné}
\subsubsection*{1. Pozdrav}
\begin{solution}
meno = input("Ako sa voláš?: ")
print("Ahoj ", meno)
print("Dovidenia ", meno)
\end{solution}

\subsubsection*{2. Básnik}
\begin{solution}
slovo = input("Napíš slovo, ktoré sa rýmuje so slovom strach: ")
print("Tu je báseň:")
print("Z počítačov mával som vždy strach")
print("teraz som však šťastný ako", slovo, ".")
\end{solution}

\subsubsection*{3. Pozvánka}
\begin{solution}
meno = input("Meno kamaráta: ")
cas = input("Čas oslavy: ")
vec = input("Prinesie okrem darčeku: ")

print(f"Ahoj {meno},")
print(f"pozývam ťa na moju narodeninovú párty.")
print(f"Bude sa konať 12.4. o {cas}.")
print("Nezabudni priniesť {vec} a pekný darček.")
print("Teším sa na teba! :)")
\end{solution}

\subsubsection*{4. Teplota vo Farenheitoch}
\begin{solution}
f = input("Vonku je °F: ")
f = float(f)
c = (5 / 9) * (f - 32)
print(f"Doma by bolo na teplomeri {c:.2f}°C.")
\end{solution}

\subsubsection*{5. Hlboká roklina}
\begin{solution}
g = 9.81
t = input("Čas do dopadu kameňa: ")
t = int(t)
h = (g * (t ** 2)) / 2
print("Hĺbka rokliny je", h, "metrov")
\end{solution}

\subsubsection*{6. Vedro s vodou}
\begin{solution}
pi = 3.14159
v = input("Výška vedra (cm): ")
d = input("Priemer dna (cm): ")
v = int(v)
d = int(d)
V = pi * ((d / 2) ** 2)
V = V / 1000
print("Do vedra sa zmestí", V, "litrov vody.")
\end{solution}

\subsubsection*{7. Cesta autom}
\begin{solution}
km = input("Dĺžka cesty (km): ")
odchod = input("Odchod z domu (hodina): ")
prichod = input("Príchod do hotela (hodina): ")

km = float(km)
odchod = int(odchod)
prichod = int(prichod)
hod = prichod - odchod

print(f"Auto pôjde priemernou rýchlosťou {km / hod:.2f} km/h.")
\end{solution}

\subsubsection*{8. Kúpalisko}
\begin{solution}
dlzka = input("Dĺžka bazéna (m): ")
sirka = input("Šírka bazéna (m): ")
hlbka = input("Hĺbka bazéna (m): ")
okraj = input("Hĺbka hladiny od okraja (cm): ")
cena = input("Cena za m^3 vody v eurách: ")

dlzka = float(dlzka)
sirka = float(sirka)
hlbka = float(hlbka)
okraj = int(okraj)
cena = float(cena)
V = dlzka * sirka * (hlbka - (okraj / 100))
V *= 1000
cena = cena * V

print(f"Na bazén sa minie {V} litrov vody
print(f"Voda bude to stáť {cena} eur.")
\end{solution}

\subsubsection*{9. Maľovanie}
\begin{solution}
# Získaj z klávesnice rozmery miestnosti
print("Rozmery miestnosti")
dlzka = input("Dĺžka (cm): ")
sirka = input("Širka (cm): ")
vyska = input("Výška (cm): ")

# Premeň z písmen na čísla
dlzka = int(dlzka)
sirka = int(sirka)
vyska = int(vyska)

# Získaj z klávesnice rozmery okna a výdatnosť farby
print("Rozmery okna")
sirkaOkna = input("Širka (cm): ")
vyskaOkna = input("Výška (cm): ")
vydatnost = input("Výdatnosť farby (m^2/kg): ")

# Premeň z písmen na čísla
sirkaOkna = int(sirkaOkna)
vyskaOkna = int(sirkaOkna)
vydatnost = float(vydatnost)

# Spočítaj plochy stien, stropu a odpočítaj plochu okna
PlochaMiestnost = \
(dlzka * sirka) + 2 * (vyska * sirka) + 2 * (vyska * dlzka)
PlochaOkno = sirkaOkna * vyskaOkna
S = (PlochaMiestnost - PlochaOkno) / 10000
farbaKg = S / vydatnost

print(f"Maľovať budeš plochu {S:.2f} m2."
print(f"Kúp {farbaKg:.2f} kg farby.")
\end{solution}

\subsubsection*{10. Chemikálie}
\begin{solution}
m1 = int(input("Hmotnosť roztoku č.1 (m1)?"))
w1 = int(input("Hmotnostný zlomok roztoku č.1 (w1)?"))
m2 = int(input("Hmotnosť roztoku č.2 (m2)?"))
w2 = int(input("Hmotnostný zlomok roztoku č.2 (w2)?"))

m3 = m1 + m2
w3 = (m1 * w1 + m2 * w_2) / m3

print(f"Výsledný roztok má hmotnosť", m3, "g")
print(f"Hmotnostný zlomok rozpustenej látky je", w3 * 100, "%")
\end{solution}

\subsubsection*{11. Brzdenie}
\begin{solution}
import math

print("Vlaková súprava")
v = int(input("- Rýchlosť (km/h): "))
lokomotiva = float(input("- Hmotnosť lokomotívy (t): "))
vagon = float(input("- Hmotnosť vagóna (t): "))
pocet_vagonov = int(input("- Počet vagónov: "))
F_b = int(input("- Brzdná sila (N/t): "))

# Premeň jednotky na základné SI
v /= 3.6
lokomotiva *= 1000
vagon *= 1000
F_b /= 1000

# Hmotnosť súpravy je hmotnosť lokomotívy a všetkých vagónov
m = lokomotiva + (pocet_vagonov * vagon)
# Vypočítaj celkovú kinetickú energiu, tá je rovnaká ako práca
# ktorú musia brzdy vykonať na zabrzdenie.
W = 0.5 * m * (v ** 2)
# Celková sila pôsobiaca proti pohybu vlaku
F = F_b * m
# Z definície práce W = F * s, vypočítaj dráhu potrebnú na zastavenie
s = W / F
# Vypočítaj čas potrebný na zastavenie pre rovnomerný spomalený pohyb
a = F / m
t = math.sqrt(2 * s / a)

print(f"Vlaková súprava má hmotnosť {int(m / 1000)} ton.")
print(f"V rýchlosti {int(v * 3.6)} km/h zabrzdí na vzdialnosť {int(s)} metrov.")
print(f"Brzdenie bude trvať {int(t)} sekúnd.")
\end{solution}
\section{Reťazce a zoznamy}

\subsection{Vymeň písmeno}
\begin{solution}
text = input("Správa: ")
chyba = input("Za chybné písmeno: ")
nahrada = input("Vymeň: ")
upravene = ""
for pismeno in text:
    if pismeno == chyba:
        upravene += nahrada
    else:
        upravene += pismeno
print("\nOpravené!")
print(upravene)
\end{solution}


\subsection{Cenzúra}
\begin{solution}
vstup = input("Správa: ")
prepis = input("Samohlásku nahraď: ")
vystup = ""
samohlasky = "aeiouyáéíóúý"
najdene = False

for i in range(len(vstup)):
    for j in range(len(samohlasky)):
        if vstup[i] == samohlasky[j]:
            vystup += prepis
            najdene = True
            break
    if not najdene:
        vystup += vstup[i]
    najdene = False
 
print("Cenzurované", vystup)
\end{solution}

\subsection{Počítanie slov}
\begin{solution}
clanok = input("Článok: ")
pocet_znakov = 0
pocet_slov = 0
pocet_viet = 0
je_medzera = True

for znak in clanok:
    pocet_znakov += 1
    if znak == ".":
        pocet_viet += 1
    if znak.isspace():
        je_medzera = True
    elif je_medzera and not znak.isspace():
        pocet_slov += 1
        je_medzera = False

print(f"Znaky: {pocet_znakov}")
print(f"Slová: {pocet_slov}")
print(f"Vety: {pocet_viet}")
print(f"Normostany: {int(pocet_znakov / 1800)}")
\end{solution}


\subsection{Najdlhšie slovo}
\begin{solution}
prejav = input("Rečnícky prejav: ")
slovo = ""
najdlhsie = ""

for znak in prejav:
    if znak.isalpha():
        slovo += znak
    else:
        if len(slovo) > len(najdlhsie):
            najdlhsie = slovo
        slovo = ""

print(f"Najdlhšie slovo v ňom: {najdlhsie}")
\end{solution}

\subsection{Frekvencia písmen}
\begin{solution}
clanok = input("Článok: ")
abeceda = [0] * 26
pismena = 0

for pismeno in clanok:
    if pismeno.isalpha():
        pozicia = ord(pismeno.upper()) - ord("A")
        if pozicia >= 0 and pozicia <= 26:
            abeceda[pozicia] += 1
            pismena += 1

for i in range(len(abecedaReťazce a zoznamy - Riešenia)):
    pismeno = chr(ord("A") + i)
    vyskyt = 100 * (abeceda[i] / pismena)
    print(f"{pismeno}: {vyskyt:.2f}%")
\end{solution}

\subsection{Histogram}
\begin{solution}
clanok = input("Článok: ")

STO_PERCENT = 100
abeceda = [0] * 26
pismena = 0

for pismeno in clanok:
    if pismeno.isalpha():
        pozicia = ord(pismeno.upper()) - ord("A")
        if pozicia >= 0 and pozicia <= 26:
            abeceda[pozicia] += 1
            pismena += 1
for i in range(len(abeceda)):
	pismeno = chr(ord("A") + i)
	vyskyt = int(STO_PERCENT * (abeceda[i] / pismena))
	print(f"{pismeno}: {'*' * vyskyt}")
\end{solution}

\subsection{Nákupný košík}
\begin{solution}
nakup = []
while True:
    tovar = input("Čo kúpiť?: ")
    if tovar == "HOTOVO":
        break
    cena = float(input(f"Cena {tovar}?: "))
    nakup.append([tovar, cena])

riadok = "+" + 20 * "-" + "+" + 15 * "-" + "+" + 15 * "-" + "+"
print(riadok)
print(f"|{'Tovar':20s}|{'DPH':15s}|{'Cena s DPH':15s}|")

celkom = 0
for polozka in nakup:
    tovar = polozka[0]
    cena = polozka[1]
    celkom += cena
    print(riadok)
    print(f"|{tovar:20s}|{cena * 0.2:15.2f}|{cena:15.2f}|")

print(riadok)
print(f"|{'CELKOM':20s}|{celkom * 0.2:15.2f}|{celkom:15.2f}|")
print(riadok)
\end{solution}


\subsection{Akronym}
\begin{solution}
veta = input("Slovné spojenie: ")
skratka = ""
je_medzera = True
for znak in veta:
	if znak.isspace():
		je_medzera = True
	elif je_medzera and znak.isalpha():
		je_medzera = False
		skratka += znak.upper()
print(f"Skratka: {skratka}")
\end{solution}


\subsection{Veľa opakovania}
\begin{solution}
cesta = input("Cesta robota: ")
skratene = ""
smer = ""
n = 0
for krok in cesta:
	if krok.isalpha():
		if smer == "":
			smer = krok
			n = 1
		elif krok != smer:
			skratene += f"{n}{smer}"
			smer = krok
			n = 1
		else:
			n += 1
skratene += f"{n}{smer}"
print(f"Skomprimované: {skratene}")
\end{solution}
